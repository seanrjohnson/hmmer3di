\begin{sreapi}
\hypertarget{func:esl_keyhash_Create()}
{\item[ESL\_KEYHASH * esl\_keyhash\_Create(void)]}

Create a new hash table for key indexing, and returns
a pointer to it.

Throws \ccode{NULL} on allocation failure.



\hypertarget{func:esl_keyhash_CreateCustom()}
{\item[ESL\_KEYHASH * esl\_keyhash\_CreateCustom(uint32\_t hashsize, int kalloc, int salloc)]}

Create a new hash table, initially allocating for
a hash table of size \ccode{hashsize} entries, \ccode{kalloc} 
keys, and a total key string length of \ccode{salloc}.
\ccode{hashsize} must be a power of 2, and all allocations
must be $\geq 0$. 

The object will still expand as needed, so the reason to
use a customized allocation is when you're trying to
minimize memory footprint and you expect your keyhash to
be smaller than the default (of up to 128 keys, of total
length up to 2048).

Throws \ccode{NULL} on allocation failure.


\hypertarget{func:esl_keyhash_Clone()}
{\item[ESL\_KEYHASH * esl\_keyhash\_Clone(const ESL\_KEYHASH *kh)]}

Allocates and duplicates a keyhash \ccode{kh}. Returns a
pointer to the duplicate.

Throws \ccode{NULL} on allocation failure.


\hypertarget{func:esl_keyhash_Get()}
{\item[char * esl\_keyhash\_Get(const ESL\_KEYHASH *kh, int idx)]}

Returns a pointer to the key name associated
with index \ccode{idx}. The key name is a \ccode{NUL}-terminated 
string whose memory is managed internally in
the keyhash \ccode{kh}.


\hypertarget{func:esl_keyhash_GetNumber()}
{\item[int esl\_keyhash\_GetNumber(const ESL\_KEYHASH *kh)]}

Returns the total number of keys currently stored in the
keyhash \ccode{kh}.


\hypertarget{func:esl_keyhash_Reuse()}
{\item[int esl\_keyhash\_Reuse(ESL\_KEYHASH *kh)]}

Empties keyhash \ccode{kh} so it can be reused without
creating a new one. 

Returns \ccode{eslOK} on success.


\hypertarget{func:esl_keyhash_Destroy()}
{\item[void esl\_keyhash\_Destroy(ESL\_KEYHASH *kh)]}

Destroys \ccode{kh}.

Returns (void)


\hypertarget{func:esl_keyhash_Dump()}
{\item[void esl\_keyhash\_Dump(FILE *fp, const ESL\_KEYHASH *kh)]}

Mainly for debugging purposes. Dump 
some information about the hash table \ccode{kh}
to the stream \ccode{fp}, which might be stderr
or stdout.


\hypertarget{func:esl_keyhash_Store()}
{\item[int esl\_keyhash\_Store(ESL\_KEYHASH *kh, const char *key, esl\_pos\_t n, int *opt\_index)]}

Store a string (or mem) \ccode{key} of length \ccode{n} in the key
index hash table \ccode{kh}.  Associate it with a unique key
index, counting from 0. This index maps hashed keys to
integer-indexed C arrays, clumsily emulating hashes or
associative arrays. Optionally returns the index through
\ccode{opt\_index}.

\ccode{key}, \ccode{n} follow the standard idiom for strings and
unterminated buffers. If \ccode{key} is raw memory, \ccode{n} must
be provided; if \ccode{key} is a \0-terminated string, \ccode{n}
may be -1.

Returns \ccode{eslOK} on success; stores \ccode{key} in \ccode{kh}; \ccode{opt\_index} is 
returned, set to the next higher index value.
Returns \ccode{eslEDUP} if \ccode{key} was already stored in the table;
\ccode{opt\_index} is set to the existing index for \ccode{key}.

Throws \ccode{eslEMEM} on allocation failure, and sets \ccode{opt\_index} to -1.


\hypertarget{func:esl_keyhash_Lookup()}
{\item[int esl\_keyhash\_Lookup(const ESL\_KEYHASH *kh, const char *key, esl\_pos\_t n, int *opt\_index)]}

Look up string or mem \ccode{key} of length \ccode{n} in hash table \ccode{kh}.
If \ccode{key} is found, return \ccode{eslOK}, and optionally set \ccode{*opt\_index}
to its array index (0..nkeys-1).
If \ccode{key} is not found, return \ccode{eslENOTFOUND}, and
optionally set \ccode{*opt\_index} to -1.

If \ccode{key} is a \0-terminated string, \ccode{n} may be -1. 


\end{sreapi}

