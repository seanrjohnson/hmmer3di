\begin{sreapi}
\hypertarget{func:esl_msaweight_PB()}
{\item[int esl\_msaweight\_PB(ESL\_MSA *msa)]}

Given a multiple alignment \ccode{msa}, calculate sequence
weights according to the position-based weighting
algorithm (Henikoff and Henikoff, JMB 243:574-578,
1994). These weights are stored internally in the \ccode{msa}
object, replacing any weights that may have already been
there. Weights are $\geq 0$ and they sum to \ccode{msa->nseq}.

The Henikoffs' algorithm is defined for ungapped
alignments. It does not give rules for dealing with
gaps, nor for degenerate residue symbols. The rule here
is to ignore these, and moreover (in digital mode
alignments, in order to deal with deep gappy alignments)
to only consider alignment columns that are "consensus"
(defined below).  This means that longer sequences
initially get more weight; hence we do a "double
normalization" in which the weights are first divided by
sequence length in canonical residues (to get the
average weight per residue, to compensate for that
effect), then normalized to sum to nseq.

The \ccode{msa} may be in either digitized or text mode. The
weighting algorithm is subtly different depending on
this mode. Digital mode is preferred. The digital mode
algorithm handles deep gappy alignments better (it only
considers consensus columns, not all columns), it deals
with degenerate residue symbols (by ignoring them, like
it ignores gaps), and it's much faster. In text mode,
the algorithm simply uses the 26 letters as "residues"
(case-insensitively), and ignores all other residues as
gaps.

PB weights require $O(NL)$ time and $O(LK)$ memory, for
an alignment of $L$ columns, $N$ sequences, and alphabet
size $K$.

Consensus columns (for the digital alignment version)
are determined as follows:

1. if the MSA has RF consensus column annotation, use it.

2. else, use HMMER's rules (fragthresh, symfrac):
- define sequence fragments as those with aligned
lengths where fractional span (l-r+1)/L \ccode{ fragthresh
for leftmost residue position l and rightmost r;
- count \% residue occupancy per column ignoring
external gaps for fragments; 
- define consensus columns as those with $\geq$
symfrac residue occupancy.
Defaults are fragthresh = 0.5, symfrac = 0.5.

3. if all else fails, use all columns.

Additionally, at step 2, if the alignment is very deep
(}50000 sequences by default), we attempt to define
consensus columns from a statistical sample (of 10,000
seqs by default). This is a speed optimization. It can
fail in a pathological case where the alignment contains
a ridiculous number of fragments piled up on one
subsequence. (Some DNA repeats can be this way.)  If we
get >5000 fragments in the sample (by default), we use
all sequences, not a subsample.

All the parameters mentioned with default parameters can
be changed using the \ccode{esl\_msaweight\_PB\_adv()} "advanced"
version.

Using RF-annotated consensus columns by default is not
always desirable. In HMMER and Infernal, default "fast"
model construction may define a different set of
consensus columns than Easel does, which means that the
same MSA can give different parameterizations, depending
on whether that MSA has RF annotation or not. We decided
this is undesirable [HMMER iss #180]. Both HMMER and
Infernal use esl_msaweight_PB_adv(), using a
ESL_MSAWEIGHT_CFG that sets ignore_rf to TRUE in
default, FALSE with --hand.

Returns \ccode{eslOK} on success. \ccode{msa->wgt[]} now contains weights for
each sequence. \ccode{eslMSA\_HASWGTS} flag is raised in
\ccode{msa->flags}. 

Throws \ccode{eslEMEM} on allocation error, in which case \ccode{msa} is
returned unmodified.



\hypertarget{func:esl_msaweight_PB_adv()}
{\item[int esl\_msaweight\_PB\_adv(const ESL\_MSAWEIGHT\_CFG *cfg, ESL\_MSA *msa, ESL\_MSAWEIGHT\_DAT *dat)]}

Same as \ccode{esl\_msaweight\_PB()} but only takes digital-mode
MSAs, and can optionally take customized parameters
\ccode{cfg}, and optionally collect data about the computation
in \ccode{dat}.

Returns \ccode{eslOK} on success. \ccode{msa->wgt[]} now contains weights for
each sequence. \ccode{eslMSA\_HASWGTS} flag is raised in
\ccode{msa->flags}.  \ccode{dat}, if provided, contains data about
stuff that happened during the weight computation.

Throws (no abnormal error conditions)


\hypertarget{func:esl_msaweight_GSC()}
{\item[int esl\_msaweight\_GSC(ESL\_MSA *msa)]}

Given a multiple sequence alignment \ccode{msa}, calculate
sequence weights according to the
Gerstein/Sonnhammer/Chothia algorithm. These weights
are stored internally in the \ccode{msa} object, replacing
any weights that may have already been there. Weights
are $\geq 0$ and they sum to \ccode{msa->nseq}.

The \ccode{msa} may be in either digitized or text mode.
Digital mode is preferred, so that distance calculations
used by the GSC algorithm are robust against degenerate
residue symbols.

This is an implementation of Gerstein et al., "A method to
weight protein sequences to correct for unequal
representation", JMB 236:1067-1078, 1994.

The algorithm is $O(N^2)$ memory (it requires a pairwise
distance matrix) and $O(N^3 + LN^2)$ time ($N^3$ for a UPGMA
tree building step, $LN^2$ for distance matrix construction)
for an alignment of N sequences and L columns. 

In the current implementation, the actual memory
requirement is dominated by two full NxN distance
matrices (one tmp copy in UPGMA, and one here): for
8-byte doubles, that's $16N^2$ bytes. To keep the
calculation under memory limits, don't process large
alignments: max 1400 sequences for 32 MB, max 4000
sequences for 256 MB, max 8000 seqs for 1 GB. Watch
out, because Pfam alignments can easily blow this up.

Returns \ccode{eslOK} on success, and the weights inside \ccode{msa} have been
modified.  

Throws \ccode{eslEINVAL} if the alignment data are somehow invalid and
distance matrices can't be calculated. \ccode{eslEMEM} on an
allocation error. In either case, the original \ccode{msa} is
left unmodified.



\hypertarget{func:esl_msaweight_BLOSUM()}
{\item[int esl\_msaweight\_BLOSUM(ESL\_MSA *msa, double maxid)]}

Given a multiple sequence alignment \ccode{msa} and an identity
threshold \ccode{maxid}, calculate sequence weights using the
BLOSUM algorithm (Henikoff and Henikoff, PNAS
89:10915-10919, 1992). These weights are stored
internally in the \ccode{msa} object, replacing any weights
that may have already been there. Weights are $\geq 0$
and they sum to \ccode{msa->nseq}.

The algorithm does a single linkage clustering by
fractional id, defines clusters such that no two clusters
have a pairwise link $\geq$ \ccode{maxid}), and assigns
weights of $\frac{1}{M_i}$ to each of the $M_i$
sequences in each cluster $i$. The \ccode{maxid} threshold
is a fractional pairwise identity, in the range
$0..1$.

The \ccode{msa} may be in either digitized or text mode.
Digital mode is preferred, so that the pairwise identity
calculations deal with degenerate residue symbols
properly.

Returns \ccode{eslOK} on success, and the weights inside \ccode{msa} have been
modified. 

Throws \ccode{eslEMEM} on allocation error. \ccode{eslEINVAL} if a pairwise
identity calculation fails because of corrupted sequence 
data. In either case, the \ccode{msa} is unmodified.



\hypertarget{func:esl_msaweight_IDFilter()}
{\item[int esl\_msaweight\_IDFilter(const ESL\_MSA *msa, double maxid, ESL\_MSA **ret\_newmsa)]}

Remove sequences in \ccode{msa} that are $\geq$ \ccode{maxid}
fractional identity to another aligned sequence.  Return
the filtered alignment in \ccode{*ret\_newmsa}.  The original
\ccode{msa} is not changed.

When a pair of sequences has $\geq$ \ccode{maxid} fractional
identity, the default rule for which of them we keep
differs depending on whether the \ccode{msa} is in digital or
text mode. We typically manipulate biosequence
alignments in digital mode. In digital mode, the default
rule (called "conscover") is to prefer the sequence with
an alignment span that covers more consensus columns.
(Where "span" is defined by the column coordinates of
the leftmost and rightmost aligned residues of the
sequence; fragment definition rules also use the concept
of "span".)  In the simpler text mode, the rule is to
keep the earlier sequence and discard the later, in the
order the seqs come in the alignment. The conscover rule
is designed to avoid keeping fragments, while trying not
to bias the insertion/deletion statistics of the
filtered alignment. To use different preference rules in
digital mode alignments, see the more customizable
\ccode{esl\_msaweight\_IDFilter\_adv()}.

Fractional identity is calculated by
\ccode{esl\_dst\_{CX}PairID()}, which defines the denominator of
the calculation as MIN(rlen1, rlen2).

Unparsed Stockholm markup is not propagated into the new
filtered alignment.

Throws \ccode{eslEMEM} on allocation error. \ccode{eslEINVAL} if a pairwise
identity calculation fails because of corrupted sequence 
data. In either case, the \ccode{msa} is unmodified.



\hypertarget{func:esl_msaweight_IDFilter_adv()}
{\item[int esl\_msaweight\_IDFilter\_adv(const ESL\_MSAWEIGHT\_CFG *cfg, const ESL\_MSA *msa, double maxid, ESL\_MSA **ret\_newmsa)]}

Same as \ccode{esl\_msaweight\_IDFilter()} but customizable with
optional custom parameters in \ccode{cfg}. Requires digital
mode alignment.

In particular, \ccode{cfg} can be used to select two other
rules for sequence preference, besides the default
"conscover" rule: random preference, or preferring the
lower-index sequence (same as the rule used in text mode
alignments).

For the "conscover" rule, consensus column determination
can be customized the same way as in PB weighting.


\end{sreapi}

