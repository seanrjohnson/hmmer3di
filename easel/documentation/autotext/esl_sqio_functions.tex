\begin{sreapi}
\hypertarget{func:esl_sqfile_Open()}
{\item[int esl\_sqfile\_Open(const char *filename, int format, const char *env, ESL\_SQFILE **ret\_sqfp)]}

Open a sequence file \ccode{filename} for reading. 
The opened \ccode{ESL\_SQFILE} is returned through \ccode{ret\_sqfp}.

The format of the file is asserted to be \ccode{format} (for
example, \ccode{eslSQFILE\_FASTA}). If \ccode{format} is
\ccode{eslSQFILE\_UNKNOWN} then the routine attempts to
autodetect the file format.

If \ccode{env} is non-NULL, it is the name of an environment
variable that contains a colon-delimited list of
directories in which we may find this \ccode{filename}.
For example, if we had 
\ccode{setenv BLASTDB /nfs/db/blast-db:/nfs/db/genomes/}
in the environment, a database search application
could pass "BLASTDB" as \ccode{env}.

Returns \ccode{eslOK} on success, and \ccode{*ret\_sqfp} points to a new
open \ccode{ESL\_SQFILE}. Caller deallocates this object with
\ccode{esl\_sqfile\_Close()}. 

Returns \ccode{eslENOTFOUND} if \ccode{filename} can't be opened.

Returns \ccode{eslEFORMAT} if the file is empty, or
if autodetection is attempted and the format can't be
determined.  

On any error condition, \ccode{*ret\_sqfp} is returned NULL.

Throws \ccode{eslEMEM} on allocation failure.


\hypertarget{func:esl_sqfile_Close()}
{\item[void esl\_sqfile\_Close(ESL\_SQFILE *sqfp)]}

Closes an open \ccode{sqfp}.

Returns (void).


\hypertarget{func:esl_sqfile_OpenDigital()}
{\item[int esl\_sqfile\_OpenDigital(const ESL\_ALPHABET *abc, const char *filename, int format, const char *env, ESL\_SQFILE **ret\_sqfp)]}

Same as \ccode{esl\_sqfile\_Open()}, but we will expect all
sequence input to conform to the digital alphabet \ccode{abc}.

Normally, after opening the sequence file in digital
mode, you'd read sequence into a digital \ccode{ESL\_SQ}.
However, you don't actually have to. The state of the
\ccode{ESL\_SQ} controls how the input is stored; the state of
the \ccode{ESL\_SQFILE} controls how the input is validated.

Returns \ccode{eslOK} on success, and \ccode{*ret\_sqfp} points to a new
open \ccode{ESL\_SQFILE}.

Returns \ccode{eslENOTFOUND} if \ccode{filename} can't be opened.
Returns \ccode{eslEFORMAT} if the file is empty, or if
autodetection is attempted and the format can't be
determined.  On any error conditions, \ccode{*ret\_sqfp} is
returned NULL.

Throws \ccode{eslEMEM} on allocation failure.


\hypertarget{func:esl_sqfile_SetDigital()}
{\item[int esl\_sqfile\_SetDigital(ESL\_SQFILE *sqfp, const ESL\_ALPHABET *abc)]}

Given an \ccode{ESL\_SQFILE} that's already been opened,
configure it to expect subsequent input to conform
to the digital alphabet \ccode{abc}.

Calling \ccode{esl\_sqfile\_Open(); esl\_sqfile\_SetDigital()} is
equivalent to \ccode{esl\_sqfile\_OpenDigital()}. The two-step
version is useful when you need a
\ccode{esl\_sqfile\_GuessAlphabet()} call in between, guessing
the file's alphabet in text mode before you set it to
digital mode.

Returns \ccode{eslOK} on success.


\hypertarget{func:esl_sqfile_GuessAlphabet()}
{\item[int esl\_sqfile\_GuessAlphabet(ESL\_SQFILE *sqfp, int *ret\_type)]}

After opening \ccode{sqfp}, attempt to guess what alphabet
its sequences are in, by inspecting the first sequence
in the file, and return this alphabet type in \ccode{*ret\_type}.

Returns \ccode{eslOK} on success, and \ccode{*ret\_type} is set to \ccode{eslDNA},
\ccode{eslRNA}, or \ccode{eslAMINO}.

Returns \ccode{eslENOALPHABET} if the alphabet can't be
reliably guessed.

Returns \ccode{eslEFORMAT} if a parse error is encountered.
Call \ccode{esl\_sqfile\_GetErrorBuf()} to get a ptr to a
user-directed error message describing the problem,
including the line number on which it was found.

Returns \ccode{eslENODATA} if the file appears to be empty.

On any error, \ccode{*ret\_type} is \ccode{eslSQFILE\_UNKNOWN}.

Throws \ccode{eslEMEM} on allocation error;
\ccode{eslEINCONCEIVABLE} on unimaginable internal errors.


\hypertarget{func:esl_sqio_Read()}
{\item[int esl\_sqio\_Read(ESL\_SQFILE *sqfp, ESL\_SQ *sq)]}

Reads the next sequence from open sequence file \ccode{sqfp} into 
\ccode{sq}. Caller provides an allocated and initialized \ccode{sq}, which
will be internally reallocated if its space is insufficient.

Returns \ccode{eslOK} on success; the new sequence is stored in \ccode{sq}.

Returns \ccode{eslEOF} when there is no sequence left in the
file (including first attempt to read an empty file).

Returns \ccode{eslEFORMAT} if a parse error is encountered.
Call \ccode{esl\_sqfile\_GetErrorBuf()} to get a ptr to a
user-directed error message describing the problem,
including the line number on which it was found.

Throws \ccode{eslEMEM} on allocation failure;
\ccode{eslEINCONCEIVABLE} on internal error.


\hypertarget{func:esl_sqio_ReadInfo()}
{\item[int esl\_sqio\_ReadInfo(ESL\_SQFILE *sqfp, ESL\_SQ *sq)]}

Read the next sequence from open sequence file \ccode{sqfp},
but don't store the sequence (or secondary structure).
Upon successful return, \ccode{s} holds all the available 
information about the sequence -- its name, accession,
description, and overall length \ccode{sq->L}. 

This is useful for indexing sequence files, where
individual sequences might be ginormous, and we'd rather
avoid reading complete seqs into memory.

Returns \ccode{eslOK} on success.


\hypertarget{func:esl_sqio_ReadSequence()}
{\item[int esl\_sqio\_ReadSequence(ESL\_SQFILE *sqfp, ESL\_SQ *sq)]}

Read the next sequence from open sequence file \ccode{sqfp},
skipping over the header data.  Upon successful return, 
\ccode{s} holds just the sequece data.  File offsets will be
filled in.

This is useful fast reads of binary formats where the
header information and sequences are stored in different
files.

Returns \ccode{eslOK} on success.


\hypertarget{func:esl_sqio_ReadWindow()}
{\item[int esl\_sqio\_ReadWindow(ESL\_SQFILE *sqfp, int C, int W, ESL\_SQ *sq)]}

Read a next window of \ccode{W} residues from open file \ccode{sqfp},
keeping \ccode{C} residues from the previous window as
context, and keeping previous annotation in the \ccode{sq}
as before. 

If this is the first window of a new sequence record,
\ccode{C} is ignored (there's no previous context yet), and
the annotation fields of the \ccode{sq} (name, accession, and
description) are initialized by reading the sequence
record's header. This is the only time the annotation
fields are initialized.

On return, \ccode{sq->dsq[]} contains the window and its
context; residues \ccode{1..sq->C} are the previous context,
and residues \ccode{sq->C+1..sq->n} are the new window.  The
start and end coordinates of the whole \ccode{dsq[1..n]}
(including context) in the original source sequence are
\ccode{sq->start..sq->end}. (Or, for text mode sequences,
\ccode{sq->seq[0..sq->C-1,sq->C..sq->n-1]}, while \ccode{start} and
\ccode{end} coords are still \ccode{1..L}.)

When a sequence record is completed and no more data
remain, \ccode{eslEOD} is returned, with an ``info'' \ccode{sq}
structure (containing the annotation and the total
sequence length \ccode{L}, but no sequence). (The total
sequence length \ccode{L} is unknown in \ccode{sq} until this
\ccode{eslEOD} return.)

The caller may then do one of two things before calling
\ccode{esl\_sq\_ReadWindow()} again; it can reset the sequence
with \ccode{esl\_sq\_Reuse()} to continue reading the next
sequence in the file, or it can set a negative \ccode{W} as a
signal to read windows from the reverse complement
(Crick) strand. Reverse complement reading only works
for nucleic acid sequence. 

If you read the reverse complement strand, you must read
the whole thing, calling \ccode{esl\_sqio\_ReadWindow()} with
negative \ccode{W} windows until \ccode{eslEOD} is returned again
with an empty (info-only) \ccode{sq} structure. When that
\ccode{EOD} is reached, the \ccode{sqfp} is repositioned at the
start of the next sequence record; the caller should now
\ccode{Reuse()} the \ccode{sq}, and the next \ccode{esl\_sqio\_ReadWindow()}
call must have a positive \ccode{W}, corresponding to starting
to read the Watson strand of the next sequence.

Note that the \ccode{ReadWindow()} interface is designed for
an idiom of sequential reading of complete sequences in
overlapping windows, possibly on both strands; if you
want more freedom to move around in the sequence
grabbing windows in another order, you can use the
\ccode{FetchSubseq()} interface.

Reading the reverse complement strand requires file
repositioning, so it will not work on non-repositionable
streams like gzipped files or a stdin pipe. Moreover,
for reverse complement input to be efficient, the
sequence file should have consistent line lengths, 
suitable for SSI's fast subsequence indexing.

Returns \ccode{eslOK} on success; \ccode{sq} now contains next window of
sequence, with at least 1 new residue. The number
of new residues is \ccode{sq->W}; \ccode{sq->C} residues are 
saved from the previous window. Caller may now
process residues \ccode{sq->dsq[sq->C+1]..sq->dsq[sq->n]}.

\ccode{eslEOD} if no new residues were read for this sequence
and strand, and \ccode{sq} now contains an empty info-only
structure (annotation and \ccode{L} are valid). Before calling
\ccode{esl\_sqio\_ReadWindow()} again, caller will either want
to make \ccode{W} negative (to start reading the Crick strand
of the current sequence), or it will want to reset the
\ccode{sq} (with \ccode{esl\_sq\_Reuse()}) to go on the next sequence.

\ccode{eslEOF} if we've already returned \ccode{eslEOD} before to
signal the end of the previous seq record, and moreover,
there's no more sequence records in the file.

\ccode{eslEINVAL} if an invalid residue is found in the
sequence, or if you attempt to take the reverse
complement of a sequence that can't be reverse
complemented.

Throws \ccode{eslESYNTAX} if you try to read a reverse window before
you've read forward strand.

\ccode{eslECORRUPT} if something goes awry internally in the
coordinate system.

\ccode{eslEMEM} on allocation error.


\hypertarget{func:esl_sqio_ReadBlock()}
{\item[int esl\_sqio\_ReadBlock(ESL\_SQFILE *sqfp, ESL\_SQ\_BLOCK *sqBlock, int max\_residues, int max\_sequences, int max\_init\_window, int long\_target)]}

Reads a block of sequences from open sequence file \ccode{sqfp} into 
\ccode{sqBlock}.

Returns \ccode{eslOK} on success; the new sequence is stored in \ccode{sqBlock}.

Returns \ccode{eslEOF} when there is no sequence left in the
file (including first attempt to read an empty file).

Returns \ccode{eslEFORMAT} if a parse error is encountered.
Call \ccode{esl\_sqfile\_GetErrorBuf()} to get a ptr to a
user-directed error message describing the problem,
including the line number on which it was found.

Throws \ccode{eslEMEM} on allocation failure;
\ccode{eslEINCONCEIVABLE} on internal error.


\hypertarget{func:esl_sqio_Parse()}
{\item[int esl\_sqio\_Parse(char *buf, int size, ESL\_SQ *s, int format)]}

Parse the buffer \ccode{buf} for a sequence \ccode{s} of type
\ccode{format}.  The buffer must contain the entire sequence.

Returns \ccode{eslOK} on success.

Throws \ccode{eslEMEM}  on allocation error.
\ccode{eslEFORMAT}  on parsing error.
\ccode{eslEINVAL} on unsupported format.


\hypertarget{func:esl_sqio_Write()}
{\item[int esl\_sqio\_Write(FILE *fp, ESL\_SQ *s, int format, int update)]}

Write sequence \ccode{s} to an open FILE \ccode{fp} in file format
\ccode{format}.

If \ccode{update} is \ccode{TRUE}, set the offsets for sequence \ccode{s}.

Returns \ccode{eslOK} on success.

Throws \ccode{eslEMEM} on allocation error.
\ccode{eslEWRITE} on system write error, such as filled disk.


\hypertarget{func:esl_sqio_Echo()}
{\item[int esl\_sqio\_Echo(ESL\_SQFILE *sqfp, const ESL\_SQ *sq, FILE *ofp)]}

Given a complete \ccode{sq} that we have read by some means
from an open \ccode{sqfp}; echo that sequence's record
onto the output stream \ccode{ofp}. 

This allows records to be regurgitated exactly as they
appear, rather than writing the subset of information
stored in an \ccode{ESL\_SQ}. \ccode{esl-sfetch} in the miniapps uses
this, for example.

Because this relies on repositioning the \ccode{sqfp}, it
cannot be called on non-positionable streams (stdin or
gzipped files). Because it relies on the sequence lying
in a contiguous sequence of bytes in the file, it cannot
be called on a sequence in a multiple alignment file.
Trying to do so throws an \ccode{eslEINVAL} exception.

Returns \ccode{eslOK} on success.

Throws \ccode{eslEINVAL}   if \ccode{sqfp} isn't a repositionable sequence file.
\ccode{eslECORRUPT} if we run out of data, probably from bad offsets
\ccode{eslEMEM}     on allocation failure.
\ccode{eslESYS}     on system call failures.




\hypertarget{func:esl_sqfile_GetErrorBuf()}
{\item[const char * esl\_sqfile\_GetErrorBuf(const ESL\_SQFILE *sqfp)]}

Returns the pointer to the error buffer.
Each parser is responsible for formatting
a zero terminated string describing the
error condition.

Returns A pointer the error message.


\hypertarget{func:esl_sqfile_IsRewindable()}
{\item[int esl\_sqfile\_IsRewindable(const ESL\_SQFILE *sqfp)]}

Returns \ccode{TRUE} if \ccode{sqfp} can be rewound (positioned 
to an offset of zero), in order to read it a second
time.


\hypertarget{func:esl_sqio_IsAlignment()}
{\item[int esl\_sqio\_IsAlignment(int fmt)]}

Returns TRUE if \ccode{fmt} is an alignment file
format code; else returns FALSE.

This function only checks the convention
that \ccode{fmt} codes $<$100 are unaligned formats,
and $\geq$100 are aligned formats. It does
not check that \ccode{fmt} is a recognized format
code.


\hypertarget{func:esl_sqio_EncodeFormat()}
{\item[int esl\_sqio\_EncodeFormat(char *fmtstring)]}

Given \ccode{fmtstring}, return format code.  For example, if
\ccode{fmtstring} is "fasta", returns \ccode{eslSQFILE\_FASTA}. Returns 
\ccode{eslSQFILE\_UNKNOWN} if \ccode{fmtstring} doesn't exactly match a 
known format.

Matching is case insensitive; fasta, FASTA, and FastA
all return \ccode{eslSQFILE\_FASTA}, for example.


\hypertarget{func:esl_sqio_DecodeFormat()}
{\item[char * esl\_sqio\_DecodeFormat(int fmt)]}

Given a format code \ccode{fmt}, returns a string label for
that format. For example, if \ccode{fmt} is \ccode{eslSQFILE\_FASTA},
returns "FASTA". 


\hypertarget{func:esl_sqfile_Position()}
{\item[int esl\_sqfile\_Position(ESL\_SQFILE *sqfp, off\_t offset)]}

Reposition an open \ccode{sqfp} to offset \ccode{offset}.
\ccode{offset} would usually be the first byte of a
desired sequence record.

Only normal sequence files can be positioned to a
nonzero offset. If \ccode{sqfp} corresponds to a standard
input stream or gzip -dc stream, it may not be
repositioned. If \ccode{sqfp} corresponds to a multiple
sequence alignment file, the only legal \ccode{offset}
is 0, to rewind the file to the beginning and 
be able to read the entire thing again.

After \ccode{esl\_sqfile\_Position()} is called on a nonzero
\ccode{offset}, and other bookkeeping information is unknown.
If caller knows it, it should set it explicitly.

See the SSI module for manipulating offsets and indices.

Returns \ccode{eslOK}     on success;
\ccode{eslEOF}    if no data can be read from this position.

Throws \ccode{eslESYS} if the fseeko() or fread() call fails.
\ccode{eslEMEM} on (re-)allocation failure.
\ccode{eslEINVAL} if the \ccode{sqfp} is not positionable.
\ccode{eslENOTFOUND} if in trying to rewind an alignment file  
by closing and reopening it, the open fails.
On errors, the state of \ccode{sqfp} is indeterminate, and
it should not be used again.


\hypertarget{func:esl_sqio_Ignore()}
{\item[int esl\_sqio\_Ignore(ESL\_SQFILE *sqfp, const char *ignoredchars)]}

Set the input map of the open \ccode{sqfp} to allow
the characters in the string \ccode{ignoredchars} to appear
in input sequences. These characters will be ignored.

For example, an application might want to ignore '*'
characters in its sequence input, because some translated
peptide files use '*' to indicate stop codons.

Returns \ccode{eslOK} on success.


\hypertarget{func:esl_sqio_AcceptAs()}
{\item[int esl\_sqio\_AcceptAs(ESL\_SQFILE *sqfp, char *xchars, char readas)]}

Set the input map of the open \ccode{sqfp} to allow the 
characters in the string \ccode{xchars} to appear in 
input sequences. These characters will all be 
mapped to the character \ccode{readas} (or, for digital
sequence input, to the digitized representation 
of the text character \ccode{readas} in the \ccode{sqfp}'s
digital alphabet).

For example, an application might want to read
'*' as 'X' when reading translated peptide files
that use '*' to indicate a stop codon.

Returns \ccode{eslOK} on success.


\hypertarget{func:esl_sqfile_OpenSSI()}
{\item[int esl\_sqfile\_OpenSSI(ESL\_SQFILE *sqfp, const char *ssifile\_hint)]}

Opens an SSI index file associated with the already open
sequence file \ccode{sqfp}. If successful, the necessary
information about the open SSI file is stored internally
in \ccode{sqfp}.

The SSI index file name is determined in one of two
ways, depending on whether a non-\ccode{NULL} \ccode{ssifile\_hint}
is provided.

If \ccode{ssifile\_hint} is \ccode{NULL}, the default for
constructing the SSI filename from the sequence
filename, by using exactly the same path (if any) for
the sequence filename, and appending the suffix \ccode{.ssi}.
For example, the SSI index for \ccode{foo} is \ccode{foo.ssi}, for
\ccode{./foo.fa} is \ccode{./foo.fa.ssi}, and for
\ccode{/my/path/to/foo.1.fa} is \ccode{/my/path/to/foo.1.fa.ssi}.

If \ccode{ssifile\_hint} is \ccode{non-NULL}, this exact fully
qualified path is used as the SSI file name.

Returns \ccode{eslOK} on success, and \ccode{sqfp->ssi} is now internally
valid.

\ccode{eslENOTFOUND} if no SSI index file is found;
\ccode{eslEFORMAT} if it's found, but appears to be in incorrect format;
\ccode{eslERANGE} if the SSI file uses 64-bit offsets but we're on
a system that doesn't support 64-bit file offsets.

Throws \ccode{eslEINVAL} if the open sequence file \ccode{sqfp} doesn't
correspond to a normal sequence flatfile -- we can't
random access in .gz compressed files, standard input,
or multiple alignment files that we're reading
sequentially.

Throws \ccode{eslEMEM} on allocation error.


\hypertarget{func:esl_sqfile_PositionByKey()}
{\item[int esl\_sqfile\_PositionByKey(ESL\_SQFILE *sqfp, const char *key)]}

Reposition \ccode{sqfp} so that the next sequence we read will
be the one named (or accessioned) \ccode{key}.

\ccode{sqfp->linenumber} is reset to be relative to the start
of the record named \ccode{key}, rather than the start of the
file.

Returns \ccode{eslOK} on success, and the file \ccode{sqfp} is repositioned
so that the next \ccode{esl\_sqio\_Read()} call will read the
sequence named \ccode{key}.

Returns \ccode{eslENOTFOUND} if \ccode{key} isn't found in the
index; in this case, the position of \ccode{sqfp} in the file
is unchanged.

Returns \ccode{eslEFORMAT} if something goes wrong trying to
read the index, almost certainly indicating a format
problem in the SSI file.

Returns \ccode{eslEOF} if, after repositioning, we fail to
load the next line or buffer from the sequence file;
this probably also indicates a format problem in the SSI
file.

Throws \ccode{eslEMEM}   on allocation error;
\ccode{eslEINVAL} if there's no open SSI index in \ccode{sqfp};
\ccode{eslESYS}   if the \ccode{fseek()} fails.

In all these cases, the state of \ccode{sqfp} becomes
undefined, and the caller should not use it again.


\hypertarget{func:esl_sqfile_PositionByNumber()}
{\item[int esl\_sqfile\_PositionByNumber(ESL\_SQFILE *sqfp, int which)]}

Reposition \ccode{sqfp} so that the next sequence we 
read will be the \ccode{which}'th sequence, where \ccode{which}
is \ccode{0..sqfp->ssi->nprimary-1}. 

\ccode{sqfp->linenumber} is reset to be relative to the start
of the record named \ccode{key}, rather than the start of the
file.

Returns \ccode{eslOK} on success, and the file \ccode{sqfp} is repositioned.

Returns \ccode{eslENOTFOUND} if there is no sequence number
\ccode{which} in the index; in this case, the position of
\ccode{sqfp} in the file is unchanged.

Returns \ccode{eslEFORMAT} if something goes wrong trying to
read the index, almost certainly indicating a format
problem in the SSI file.

Returns \ccode{eslEOF} if, after repositioning, we fail to
load the next line or buffer from the sequence file;
this probably also indicates a format problem in the SSI
file.

Throws \ccode{eslEMEM}   on allocation error;
\ccode{eslEINVAL} if there's no open SSI index in \ccode{sqfp};
\ccode{eslESYS}   if the \ccode{fseek()} fails.

In all these cases, the state of \ccode{sqfp} becomes
undefined, and the caller should not use it again.


\hypertarget{func:esl_sqio_Fetch()}
{\item[int esl\_sqio\_Fetch(ESL\_SQFILE *sqfp, const char *key, ESL\_SQ *sq)]}

Fetch a sequence named (or accessioned) \ccode{key} from
the repositionable, open sequence file \ccode{sqfp}.
The open \ccode{sqfp} must have an open SSI index.
The sequence is returned in \ccode{sq}.

Returns \ccode{eslOK} on soccess.
\ccode{eslEINVAL} if no SSI index is present, or if \ccode{sqfp} can't
be repositioned.
\ccode{eslENOTFOUND} if \ccode{source} isn't found in the file.
\ccode{eslEFORMAT} if either the index file or the sequence file
can't be parsed, because of unexpected format issues.

Throws \ccode{eslEMEM} on allocation error.


\hypertarget{func:esl_sqio_FetchInfo()}
{\item[int esl\_sqio\_FetchInfo(ESL\_SQFILE *sqfp, const char *key, ESL\_SQ *sq)]}

Fetch a sequence named (or accessioned) \ccode{key} from
the repositionable, open sequence file \ccode{sqfp}, reading
all info except the sequence (and secondary structure).
The open \ccode{sqfp} must have an open SSI index.
The sequence info is returned in \ccode{sq}.

Returns \ccode{eslOK} on soccess.
\ccode{eslEINVAL} if no SSI index is present, or if \ccode{sqfp} can't
be repositioned.
\ccode{eslENOTFOUND} if \ccode{source} isn't found in the file.
\ccode{eslEFORMAT} if either the index file or the sequence file
can't be parsed, because of unexpected format issues.

Throws \ccode{eslEMEM} on allocation error.


\hypertarget{func:esl_sqio_FetchSubseq()}
{\item[int esl\_sqio\_FetchSubseq(ESL\_SQFILE *sqfp, const char *source, int64\_t start, int64\_t end, ESL\_SQ *sq)]}

Fetch subsequence \ccode{start..end} from a sequence named (or
accessioned) \ccode{source}, in the repositionable, open sequence file \ccode{sqfp}.
The open \ccode{sqfp} must have an SSI index. Put the
subsequence in \ccode{sq}. 

As a special case, if \ccode{end} is 0, the subsequence is
fetched all the way to the end, so you don't need to
look up the sequence length \ccode{L} to fetch a suffix.

The caller may want to rename/reaccession/reannotate the
subsequence.  Upon successful return, \ccode{sq->name} is set
to \ccode{source/start-end}, and \ccode{sq->source} is set to
\ccode{source} The accession and description \ccode{sq->acc} and
\ccode{sq->desc} are set to the accession and description of
the source sequence.

Returns \ccode{eslOK} on success.
\ccode{eslEINVAL} if no SSI index is present, or if \ccode{sqfp} can't
be repositioned.
\ccode{eslENOTFOUND} if \ccode{source} isn't found in the file.
\ccode{eslEFORMAT} if either the index file or the sequence file
can't be parsed, because of unexpected format issues.
\ccode{eslERANGE} if the \ccode{start..end} coords don't lie entirely
within the \ccode{source} sequence.

Throws \ccode{eslEMEM} on allocation errors.


\hypertarget{func:esl_sqfile_Cache()}
{\item[int esl\_sqfile\_Cache(const ESL\_ALPHABET *abc, const char *seqfile, int fmt, const char *env, ESL\_SQCACHE **ret\_sqcache)]}

Read an entire database into memory building a cached
structure \ccode{ESL\_SQCACHE}.  The cache structure has basic
information about the database, ie number of sequences
number of residues, etc.

All sequences \ccode{ESL\_SQ} are in a memory array \ccode{sq\_list}.
The number of elements in the list is \ccode{seq\_count}.  The
header pointers, ie name, acc and desc are pointers into
the \ccode{header\_mem} buffer.  All digitized sequences are pointers
into the \ccode{residue\_mem} buffer.

Returns \ccode{eslOK} on success.

Returns \ccode{eslEFORMAT} if a parse error is encountered in
trying to read the sequence file.

Returns \ccode{eslENODATA} if the file appears to be empty.

Throws \ccode{eslEMEM} on allocation error;


\hypertarget{func:esl_sqfile_Free()}
{\item[void esl\_sqfile\_Free(ESL\_SQCACHE *sqcache)]}

Frees all the memory used to cache the sequence database.

Returns none.


\end{sreapi}

