\begin{sreapi}
\hypertarget{func:esl_ssi_Open()}
{\item[int esl\_ssi\_Open(const char *filename, ESL\_SSI **ret\_ssi)]}

Open the SSI index file \ccode{filename}, and returns a pointer
to the new \ccode{ESL\_SSI} object in \ccode{ret\_ssi}.

Caller is responsible for closing the SSI file with
\ccode{esl\_ssi\_Close()}.

Returns \ccode{eslOK}        on success;
\ccode{eslENOTFOUND} if \ccode{filename} cannot be opened for reading;
\ccode{eslEFORMAT}   if it's not in correct SSI file format;
\ccode{eslERANGE}    if it uses 64-bit file offsets, and we're on a system
that doesn't support 64-bit file offsets.

Throws \ccode{eslEMEM} on allocation error.


\hypertarget{func:esl_ssi_FindName()}
{\item[int esl\_ssi\_FindName(ESL\_SSI *ssi, const char *key, uint16\_t *ret\_fh, off\_t *ret\_roff, off\_t *opt\_doff, int64\_t *opt\_L)]}

Looks up the string \ccode{key} in index \ccode{ssi}.
\ccode{key} can be either a primary or secondary key. If \ccode{key}
is found, \ccode{ret\_fh} contains a unique handle on
the file that contains \ccode{key} (suitable for an \ccode{esl\_ssi\_FileInfo()}
call, or for comparison to the handle of the last file
that was opened for retrieval), and \ccode{ret\_offset} contains
the offset of the sequence record in that file.

Returns \ccode{eslOK}        on success;
\ccode{eslENOTFOUND} if no such key is in the index;
\ccode{eslEFORMAT}   if an fread() or fseeko() fails, which almost
certainly reflects some kind of misformatting of
the index.

Throws \ccode{eslEMEM}      on allocation error.


\hypertarget{func:esl_ssi_FindNumber()}
{\item[int esl\_ssi\_FindNumber(ESL\_SSI *ssi, int64\_t nkey, uint16\_t *opt\_fh, off\_t *opt\_roff, off\_t *opt\_doff, int64\_t *opt\_L, char **opt\_pkey)]}

Looks up primary key number \ccode{nkey} in the open index
\ccode{ssi}.  \ccode{nkey} ranges from \ccode{0..ssi->nprimary-1}. When
key \ccode{nkey} is found, any/all of several optional
arguments point to results. \ccode{*opt\_fh} contains a unique
handle on the file that contains that key (suitable for
an \ccode{esl\_ssi\_FileInfo()} call, or for comparison to the
handle of the last file that was opened for retrieval).
\ccode{*opt\_roff} contains the record offset; \ccode{*opt\_doff}
contains the data offset; \ccode{*opt\_L} contains the record
length; and \ccode{*opt\_pkey} points to the primary key name
(a string, allocated here, that the caller becomes
responsible for free'ing).

Returns \ccode{eslOK}        on success;
\ccode{eslENOTFOUND} if there is no sequence record \ccode{nkey};
\ccode{eslEFORMAT}   if a read or a seek fails, probably indicating
some kind of file misformatting.

Throws \ccode{eslEMEM} on allocation error.


\hypertarget{func:esl_ssi_FindSubseq()}
{\item[int esl\_ssi\_FindSubseq(ESL\_SSI *ssi, const char *key, int64\_t requested\_start,
		   uint16\_t *ret\_fh, off\_t *ret\_roff, off\_t *ret\_doff, int64\_t *ret\_L, int64\_t *ret\_actual\_start)]}

Fast subsequence retrieval: look up a primary or secondary
\ccode{key} in the open index \ccode{ssi}, and ask for the nearest data
offset to a subsequence starting at residue
\ccode{requested\_start} in the sequence (numbering the sequence
\ccode{1..L}).  If \ccode{key} is found, on return, \ccode{ret\_fh} contains
a unique handle on the file that contains \ccode{key};
\ccode{ret\_roff} contains the disk offset to the start of the
sequence record; \ccode{ret\_doff} contains the disk offset
(see below); and \ccode{ret\_actual\_start) contains the coordinate
(1..L) of the first valid residue at or after
<data\_offset}. \ccode{ret\_actual\_start} is $\leq$
\ccode{requested\_start}.

Depending on the file's characteristics, there are four
possible outcomes.

If the file has the \ccode{eslSSI\_FASTSUBSEQ} flag set, a data
offset was indexed for this key, and the data can be
indexed at single residue resolution (because the file's
lines contain only residues, no spaces), then \ccode{ret\_doff}
is exactly the position of residue \ccode{requested\_start} on
disk, and \ccode{ret\_actual\_start} is \ccode{requested\_start}.

If the file has the \ccode{eslSSI\_FASTSUBSEQ} flag set, a data
offset was indexed for this key, but the data can only be
indexed at line resolution (because at least some of the
file's lines contain spaces), then \ccode{ret\_doff} is the
position of the start of the line that \ccode{requested\_start}
is on, and \ccode{ret\_actual\_start} is the coord \ccode{1..L} of the
first residue on that line.

If the file does not have the \ccode{eslSSI\_FASTSUBSEQ} flag
set (because lines contain a variable number of residues
and/or bytes), but a data offset was indexed for this
key, then we can still at least return that data offset,
but the caller is going to have to start from the
beginning of the data and read residues until it reaches
the desired \ccode{requested\_start}. Now \ccode{ret\_doff} is the
offset to the start of the first line of the sequence
data, and \ccode{ret\_actual\_start} is 1.

If the key does not have a data offset indexed at all,
then regardless of the file's \ccode{eslSSI\_FASTSUBSEQ}
setting, we can't calculate even the position of the
first line. In this case, \ccode{ret\_doff} is 0 (for
unset/unknown), and \ccode{ret\_actual\_start} is \ccode{1}.

A caller that's going to position the disk and read a
subseq must check for all four possible outcomes (pardon
redundancy with the above, but just to be clear, from the
caller's perspective now):

If \ccode{ret\_doff} is 0, no data offset information can be
calculated; the caller can still use \ccode{ret\_roff} to
position the disk to the start of \ccode{key}'s record, but it
will need to parse the header to find the start of the
sequence data; then it will need to parse the sequence
data, skipping to residue \ccode{requested start}.

If \ccode{ret\_doff} is valid ($>0$), and \ccode{ret\_actual\_start} is
1, then caller may use \ccode{ret\_doff} to position the disk to
the start of the first sequence data line, but will still
need to parse all the sequence data, counting and
skipping to residue \ccode{requested start}. This is equivalent
to (and in practice, not much more efficient than)
positioning to the record start and parsing the header to
locate the sequence data start. 

If \ccode{ret\_doff} is valid ($>0$), and \ccode{ret\_actual\_start} is
$>1$ but $<$ \ccode{requested\_start}, then \ccode{ret\_doff} is the
offset to the first byte of a line on which the
subsequence begins. The caller can position the disk
there, then start parsing, skipping \ccode{requested\_start -
*ret\_actual\_start} residues to reach the
\ccode{requested\_start}. (In the case where the subsequence
begins on the first line, then \ccode{ret\_actual\_start} will be
1, and the caller will have to handle this as the case
above.)

If \ccode{<ret\_doff} is valid ($>0$), and \ccode{ret\_actual\_start} is
$=$ \ccode{requested\_start}, then \ccode{ret\_doff} is the offset to a
byte in the file, such that the requested subsequence
starts at the next valid residue at or after that
position.  (The \ccode{ret\_doff} would usually be exactly the
first residue of the subsequence, because we used single
residue resolution arithmetic to find it, but there's a
case where \ccode{requested\_start} happens to be the first
residue of a line and we calculated \ccode{ret\_doff} using
line-resolution arithmetic; in this latter case,
\ccode{ret\_doff} could be pointing at a space before the first
subseq residue.) The caller may position the disk there
and start parsing immediately; the first valid residue
will be the start of the subsequence.

Returns \ccode{eslOK}         on any of the four successful outcomes.
\ccode{eslENOTFOUND}  if no such key is found in the index;
\ccode{eslEFORMAT} on a read or seek failure, presumably meaning that
the file is misformatted somehow;
\ccode{eslERANGE}  if \ccode{requested\_start} isn't somewhere in the range
\ccode{1..len} for the target sequence.

Throws \ccode{eslEMEM} on allocation error.                       


\hypertarget{func:esl_ssi_FileInfo()}
{\item[int esl\_ssi\_FileInfo(ESL\_SSI *ssi, uint16\_t fh, char **ret\_filename, int *ret\_format)]}

Given a file number \ccode{fh} in an open index file
\ccode{ssi}, retrieve file name \ccode{ret\_filename} and
the file format \ccode{ret\_format}. 

\ccode{ret\_filename} is a pointer to a string maintained
internally by \ccode{ssi}. It should not be free'd; 
\ccode{esl\_ssi\_Close(ssi)} will take care of it.

Returns \ccode{eslOK} on success.

Throws \ccode{eslEINVAL} if there is no such file number \ccode{fh}.


\hypertarget{func:esl_ssi_Close()}
{\item[void esl\_ssi\_Close(ESL\_SSI *ssi)]}

Close an open SSI index \ccode{ssi}.



\hypertarget{func:esl_newssi_Open()}
{\item[int esl\_newssi\_Open(const char *ssifile, int allow\_overwrite, ESL\_NEWSSI **ret\_newssi)]}

Creates and returns a \ccode{ESL\_NEWSSI}, in order to create a 
new SSI index file.

Returns \ccode{eslOK} on success, and \ccode{*ret\_newssi} is a pointer to a
new \ccode{ESL\_NEWSSI} structure.

Returns \ccode{eslENOTFOUND} if \ccode{ssifile} can't be opened.

Returns \ccode{eslEOVERWRITE} if \ccode{allow\_overwrite} is \ccode{FALSE}
and \ccode{ssifile} (or any necessary tmp files) already
exist, to block overwriting of an existing SSI file.

Throws \ccode{eslEMEM} on allocation error.


\hypertarget{func:esl_newssi_AddFile()}
{\item[int esl\_newssi\_AddFile(ESL\_NEWSSI *ns, const char *filename, int fmt, uint16\_t *ret\_fh)]}

Registers the file \ccode{filename} into the new index \ccode{ns},
along with its format code \ccode{fmt}. The index assigns it
a unique handle, which it returns in \ccode{ret\_fh}. This
handle is needed when registering primary keys.

Caller should make sure that the same file isn't registered
twice; this function doesn't check.

Returns \ccode{eslOK} on success;
\ccode{eslERANGE} if registering this file would exceed the
maximum number of indexed files.

Throws \ccode{eslEMEM} on allocation or reallocation error.


\hypertarget{func:esl_newssi_SetSubseq()}
{\item[int esl\_newssi\_SetSubseq(ESL\_NEWSSI *ns, uint16\_t fh, uint32\_t bpl, uint32\_t rpl)]}

Declare that the file associated with handle \ccode{fh} is
suitable for fast subsequence lookup, because it has
a constant number of residues and bytes per (nonterminal)
data line, \ccode{rpl} and \ccode{bpl}, respectively.

Caller is responsible for this being true: \ccode{rpl} and
\ccode{bpl} must be constant for every nonterminal line of 
every sequence in this file.

Returns \ccode{eslOK} on success.

Throws \ccode{eslEINVAL} on invalid argument(s).


\hypertarget{func:esl_newssi_AddKey()}
{\item[int esl\_newssi\_AddKey(ESL\_NEWSSI *ns, const char *key, uint16\_t fh, 
		  off\_t r\_off, off\_t d\_off, int64\_t L)]}

Register primary key \ccode{key} in new index \ccode{ns}, while telling
the index that this primary key is in the file associated
with filehandle \ccode{fh} (the handle returned by a previous call
to \ccode{esl\_newssi\_AddFile()}); that its record starts at 
offset \ccode{r\_off} in the file; that its data (usually
sequence data) starts at offset \ccode{d\_off} in the file (i.e.
after any record header); and that the record's data is
of length \ccode{L} (usually, the record is a sequence, and \ccode{L} 
is its length in residues).

The data length \ccode{L} is technically optional as far as SSI
is concerned; \ccode{L} may be passed as 0 to leave it
unset. However, functions in the \ccode{sqio} module that use
SSI indices will assume that \ccode{L} is available.

\ccode{d\_off} is also optional; it may be passed as \ccode{0} to
leave it unset. If provided, \ccode{d\_off} gives an offset to
the data portion of the record. The interpretation of
this data offset may be implementation-defined and may
depend on the format of the datafile; for example, in how
\ccode{sqio} uses SSI indices, \ccode{d\_off} is the offset to the
start of the first sequence line.

Both \ccode{d\_off} and \ccode{L} must be provided, and additionally
\ccode{eslSSI\_FASTSUBSEQ} must be set for this file, for fast
subsequence lookup to work.

Returns \ccode{eslOK}        on success;
\ccode{eslERANGE}    if registering this key would exceed the maximum
number of primary keys;
\ccode{eslENOTFOUND} if we needed to open external tmp files, but
the attempt to open them failed.

Throws \ccode{eslEINVAL} on an invalid argument;
\ccode{eslEMEM}   on allocation failure;
\ccode{eslEWRITE} on any system error writing to tmp file, such
as filling the filesystem.


\hypertarget{func:esl_newssi_AddAlias()}
{\item[int esl\_newssi\_AddAlias(ESL\_NEWSSI *ns, const char *alias, const char *key)]}

Registers secondary key \ccode{alias} in index \ccode{ns}, and 
map it to the primary key \ccode{key}. \ccode{key} must already
have been registered. That is, when someone looks up \ccode{alias},
we'll retrieve record \ccode{key}. 

Returns \ccode{eslOK}        on success;
\ccode{eslERANGE}    if registering this key would exceed the maximum
number of secondary keys that can be stored;
\ccode{eslENOTFOUND} if we needed to open external tmp files, but
the attempt to open them failed.

Throws \ccode{eslEWRITE}   on any system error writing to tmp file, such 
as running out of space on the device.


\hypertarget{func:esl_newssi_Write()}
{\item[int esl\_newssi\_Write(ESL\_NEWSSI *ns)]}

Writes the complete index \ccode{ns} in SSI format to its file,
and closes the file.

Handles all necessary overhead of sorting the primary and
secondary keys, including any externally sorted tmpfiles that
may have been needed for large indices.

You only \ccode{\_Write()} once. The open SSI file is closed.
After calling \ccode{\_Write()}, you should \ccode{\_Close()} the
\ccode{ESL\_NEWSSI}.

Verifies that all primary and secondary keys are unique.
If not, returns \ccode{eslEDUP}.

On any error, the SSI file \ccode{ns->ssifile} is deleted.

Returns \ccode{eslOK}       on success;
\ccode{eslEDUP}     if primary or secondary keys aren't all unique
\ccode{eslERANGE}   if index size exceeds system's maximum file size;
\ccode{eslESYS}     if any of the steps of an external sort fail.

Throws \ccode{eslEINVAL} on invalid argument, including too-long tmpfile names,
or trying to _Write() the \ccode{ESL\_NEWSSI} more than once;
\ccode{eslEMEM}   on buffer allocation failure;
\ccode{eslEWRITE} on any system write failure, including filled disk.  



\hypertarget{func:esl_newssi_Close()}
{\item[void esl\_newssi\_Close(ESL\_NEWSSI *ns)]}

Frees a \ccode{ESL\_NEWSSI}.


\hypertarget{func:esl_byteswap()}
{\item[void esl\_byteswap(char *swap, int nbytes)]}

Swap between big-endian and little-endian, in place.


\hypertarget{func:esl_ntoh16()}
{\item[uint16\_t esl\_ntoh16(uint16\_t netshort)]}

Convert a 2-byte integer from network-order to host-order,
and return it.

\ccode{esl\_ntoh32()} and \ccode{esl\_ntoh64()} do the same, but for 4-byte
and 8-byte integers, respectively.


\hypertarget{func:esl_hton16()}
{\item[uint16\_t esl\_hton16(uint16\_t hostshort)]}

Convert a 2-byte integer from host-order to network-order, and
return it.

\ccode{esl\_hton32()} and \ccode{esl\_hton64()} do the same, but for 4-byte
and 8-byte integers, respectively.


\hypertarget{func:esl_fread_u16()}
{\item[int esl\_fread\_u16(FILE *fp, uint16\_t *ret\_result)]}

Read a 2-byte network-order integer from \ccode{fp}, convert to
host order, leave it in \ccode{ret\_result}.

\ccode{esl\_fread\_u32()} and \ccode{esl\_fread\_u64()} do the same, but
for 4-byte and 8-byte integers, respectively.

Returns \ccode{eslOK} on success, and \ccode{eslFAIL} on \ccode{fread()} failure.


\hypertarget{func:esl_fwrite_u16()}
{\item[int esl\_fwrite\_u16(FILE *fp, uint16\_t n)]}

Write a 2-byte host-order integer \ccode{n} to stream \ccode{fp}
in network order.

\ccode{esl\_fwrite\_u32()} and \ccode{esl\_fwrite\_u64()} do the same, but
for 4-byte and 8-byte integers, respectively.

Returns \ccode{eslOK} on success, and \ccode{eslFAIL} on \ccode{fwrite()} failure.


\hypertarget{func:esl_fread_offset()}
{\item[int esl\_fread\_offset(FILE *fp, int sz, off\_t *ret\_offset)]}

Read a file offset from the stream \ccode{fp} (which would usually
be a save file), and store it in \ccode{ret\_offset}.

Offsets may have been saved by a different machine
than the machine that reads them. The writer and the reader
may differ in byte order and in width (\ccode{sizeof(off\_t)}). 

Byte order is dealt with by saving offsets in 
network byte order, and converting them to host byte order
when they are read (if necessary). 

Width is dealt with by the \ccode{sz} argument, which must be
either 4 or 8, specifying that the saved offset is a
32-bit versus 64-bit \ccode{off\_t}. If the reading host
\ccode{off\_t} width matches the \ccode{sz} of the writer, no
problem. If \ccode{sz} is 4 but the reading host has 64-bit
\ccode{off\_t}'s, this is also no problem; the conversion
always works. If \ccode{sz} is 64 but the reading host has
only 32-bit \ccode{off\_t}, we cannot guarantee that we have
sufficient dynamic range to represent the offset; if
the stored offset is too large to represent in a 32-bit
offset, we throw a fatal \ccode{eslEINCOMPAT} error.

Returns \ccode{eslOK} on success; \ccode{eslFAIL} on a read failure.

Throws \ccode{eslEINVAL} if \ccode{sz} is something other than 4 or 8;
\ccode{eslEINCOMPAT} if the stored offset is too large for
the reader to represent (the machine that wrote the
SSI file used 64 bit offsets, the reader uses 32
bit offsets, and this offset is too large to represent
in a 32 bit offset).


\hypertarget{func:esl_fwrite_offset()}
{\item[int esl\_fwrite\_offset(FILE *fp, off\_t offset)]}

Portably write (save) \ccode{offset} to the stream \ccode{fp}, in network
byte order. 

Returns \ccode{eslOK} on success; \ccode{eslFAIL} on write failure.

Throws \ccode{eslEINVAL} if \ccode{off\_t} is something other than a 32-bit or
64-bit integer on this machine, in which case we don't know
how to deal with it portably.


\end{sreapi}

