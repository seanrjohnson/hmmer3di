\begin{sreapi}
\hypertarget{func:esl_rsq_Sample()}
{\item[int esl\_rsq\_Sample(ESL\_RANDOMNESS *rng, int allowed\_chars, int L, char **ret\_s)]}

Sample a random character string of length \ccode{L}, 
consisting of characters in the set defined by
an integer flag \ccode{allowed\_chars}, using 
random number generator \ccode{rng}. 

Return the new NUL-terminated string in \ccode{*ret\_s}.  This
may either be a new allocation, or in pre-allocated
storage provided by the caller. If caller passes
\ccode{*ret\_s} as \ccode{NULL}, new space is allocated, and the
caller is responsible for freeing it. That is: 
\begin{cchunk}
char *s  = NULL; 
esl_rsq_Sample(..., &s); 
free(s);
\end{cchunk}

If caller passes a non-\ccode{NULL} \ccode{*ret\_s}, it is assumed to
be a preallocated space of at least \ccode{L+1} characters,
and caller is (of course) responsible for freeing
it. That is: 
\begin{cchunk}
char *s = malloc(L+1);
esl_rsq_Sample(...,L, &s);
free(s);
\end{cchunk}

Allowed values of the flag \ccode{allowed\_char\_flag} mirror
the standard C99 character set functions in \ccode{ctype.h}:

| \ccode{eslRSQ\_SAMPLE\_ALNUM}  |  isalnum()  | isalpha() or isdigit() |
| \ccode{eslRSQ\_SAMPLE\_ALPHA}  |  isalpha()  | islower() or isupper() |
| \ccode{eslRSQ\_SAMPLE\_LOWER}  |  islower()  | [a-z] |
| \ccode{eslRSQ\_SAMPLE\_UPPER}  |  isupper()  | [A-Z] |
| \ccode{eslRSQ\_SAMPLE\_DIGIT}  |  isdigit()  | [0-9] |
| \ccode{eslRSQ\_SAMPLE\_XDIGIT} |  isxdigit() | [0-9] or [a-f] or [A-F] |
| \ccode{eslRSQ\_SAMPLE\_CNTRL}  |  iscntrl()  | ASCII control characters |
| \ccode{eslRSQ\_SAMPLE\_GRAPH}  |  isgraph()  | any printing char except space |
| \ccode{eslRSQ\_SAMPLE\_SPACE}  |  isspace()  | space, and other whitespace such as tab, newline |
| \ccode{eslRSQ\_SAMPLE\_BLANK}  |  isblank()  | space or tab |
| \ccode{eslRSQ\_SAMPLE\_PRINT}  |  isprint()  | any printing char including space |
| \ccode{eslRSQ\_SAMPLE\_PUNCT}  |  ispunct()  | punctuation |

Note that with \ccode{eslRSQ\_SAMPLE\_CNTRL}, your string
may sample NUL control characters (\ccode{0}), in addition to
the string-terminating one at \ccode{(*ret\_s)[L]}, so \ccode{strlen(*ret\_s)}
may not equal \ccode{L} in this case. 

These values are exclusive: you use one and only one of
them as \ccode{allowed\_chars}, you can't logically OR or NOT
them together.

Returns \ccode{eslOK} on success; \ccode{*ret\_s} is the sampled string, which was
newly allocated here, and caller becomes responsible for free'ing.

Throws \ccode{eslEMEM} on allocation error; \ccode{eslEINVAL} if caller 
passes an invalid value of \ccode{allowed\_chars}. Now \ccode{*ret\_s} 
is \ccode{NULL}.


\hypertarget{func:esl_rsq_IID()}
{\item[int esl\_rsq\_IID(ESL\_RANDOMNESS *r, const char *alphabet, const double *p, int K, int L, char *s)]}

Generate a \ccode{NUL}-terminated i.i.d. symbol string of length \ccode{L},
$0..L-1$, and leave it in \ccode{s}. The symbol alphabet is given
as a string \ccode{alphabet} of \ccode{K} total symbols, and the iid
probability of each residue is given in \ccode{p}. The caller
must provide an \ccode{s} that is allocated for at least
\ccode{(L+1)*sizeof(char)}, room for \ccode{L} residues and the \ccode{NUL} terminator.

\ccode{esl\_rsq\_fIID()} does the same, but for a floating point
probability vector \ccode{p}, rather than a double precision
vector.



\hypertarget{func:esl_rsq_CShuffle()}
{\item[int esl\_rsq\_CShuffle(ESL\_RANDOMNESS *r, const char  *s, char *shuffled)]}

Returns a shuffled version of \ccode{s} in \ccode{shuffled}, given
a source of randomness \ccode{r}.

Caller provides allocated storage for \ccode{shuffled}, for at
least the same length as \ccode{s}.

\ccode{shuffled} may also point to the same storage as \ccode{s},
in which case \ccode{s} is shuffled in place.

Returns \ccode{eslOK} on success.


\hypertarget{func:esl_rsq_CShuffleDP()}
{\item[int esl\_rsq\_CShuffleDP(ESL\_RANDOMNESS *r, const char *s, char *shuffled)]}

Given string \ccode{s}, and a source of randomness \ccode{r},
returns shuffled version in \ccode{shuffled}. The shuffle
is a "doublet-preserving" (DP) shuffle which
shuffles a sequence while exactly preserving both mono-
and di-symbol composition. 

\ccode{s} may only consist of alphabetic characters [a-zA-Z].
The shuffle is done case-insensitively. The shuffled
string result is all upper case.

Caller provides storage in \ccode{shuffled} of at least the
same length as \ccode{s}.

\ccode{shuffled} may also point to the same storage as \ccode{s},
in which case \ccode{s} is shuffled in place.

The algorithm does an internal allocation of a
substantial amount of temporary storage, on the order of
\ccode{26 * strlen(s)}, so an allocation failure is possible
if \ccode{s} is long enough.

The algorithm is a search for a random Eulerian walk on
a directed multigraph \citep{AltschulErickson85}.

If \ccode{s} is of length 2 or less, this is a no-op, and
\ccode{shuffled} is a copy of \ccode{s}.

Returns \ccode{eslOK} on success.

Throws \ccode{eslEINVAL} if \ccode{s} contains nonalphabetic characters.
\ccode{eslEMEM} on allocation failure.


\hypertarget{func:esl_rsq_CShuffleKmers()}
{\item[int esl\_rsq\_CShuffleKmers(ESL\_RANDOMNESS *r, const char *s, int K, char *shuffled)]}

Consider a text sequence \ccode{s} as a string of nonoverlapping
k-mers of length \ccode{K}. Shuffle the k-mers, given a random
number generator \ccode{r}. Put the shuffled sequence in
\ccode{shuffled}.

If the length of \ccode{s} is not evenly divisible by \ccode{K}, the
remaining residues are left (unshuffled) as a prefix to
the shuffled k-mers.

For example, shuffling ABCDEFGHIJK as k=3-mers might
result in ABFIJKFGHCDE.

Caller provides allocated storage for \ccode{shuffled},
for at least the same length as \ccode{s}. 

\ccode{shuffled} may also point to the same storage as \ccode{s},
in which case \ccode{s} is shuffled in place.

There is almost no formally justifiable reason why you'd
use this shuffle -- it's not like it preserves any
particularly well-defined statistical properties of the
sequence -- but it's a quick and dirty way to sort of
maybe possibly preserve some higher-than-monomer
statistics.

Returns \ccode{eslOK} on success.

Throws \ccode{eslEMEM} on allocation error.


\hypertarget{func:esl_rsq_CReverse()}
{\item[int esl\_rsq\_CReverse(const char *s, char *rev)]}

Returns a reversed version of \ccode{s} in \ccode{rev}. 

There are no restrictions on the symbols that \ccode{s}
might contain.

Caller provides storage in \ccode{rev} for at least
\ccode{(strlen(s)+1)*sizeof(char)}.

\ccode{s} and \ccode{rev} can point to the same storage, in which
case \ccode{s} is reversed in place.

Returns \ccode{eslOK} on success.


\hypertarget{func:esl_rsq_CShuffleWindows()}
{\item[int esl\_rsq\_CShuffleWindows(ESL\_RANDOMNESS *r, const char *s, int w, char *shuffled)]}

Given string \ccode{s}, shuffle residues in nonoverlapping
windows of width \ccode{w}, and put the result in \ccode{shuffled}.
See [Pearson88].

\ccode{s} and \ccode{shuffled} can be identical to shuffle in place.

Caller provides storage in \ccode{shuffled} for at least
\ccode{(strlen(s)+1)*sizeof(char)}.



\hypertarget{func:esl_rsq_CMarkov0()}
{\item[int esl\_rsq\_CMarkov0(ESL\_RANDOMNESS *r, const char *s, char *markoved)]}

Makes a random string \ccode{markoved} with the same length and
0-th order Markov properties as \ccode{s}, given randomness
source \ccode{r}.

\ccode{s} and \ccode{markoved} can be point to the same storage, in which
case \ccode{s} is randomized in place, destroying the original
string.

\ccode{s} must consist only of alphabetic characters [a-zA-Z].
Statistics are collected case-insensitively over 26 possible
residues. The random string is generated all upper case.

Returns \ccode{eslOK} on success.

Throws \ccode{eslEINVAL} if \ccode{s} contains nonalphabetic characters.


\hypertarget{func:esl_rsq_CMarkov1()}
{\item[int esl\_rsq\_CMarkov1(ESL\_RANDOMNESS *r, const char *s, char *markoved)]}

Makes a random string \ccode{markoved} with the same length and
1st order (di-residue) Markov properties as \ccode{s}, given
randomness source \ccode{r}.

\ccode{s} and \ccode{markoved} can be point to the same storage, in which
case \ccode{s} is randomized in place, destroying the original
string.

\ccode{s} must consist only of alphabetic characters [a-zA-Z].
Statistics are collected case-insensitively over 26 possible
residues. The random string is generated all upper case.

If \ccode{s} is of length 2 or less, this is a no-op, and
\ccode{markoved} is a copy of \ccode{s}.

Returns \ccode{eslOK} on success.

Throws \ccode{eslEINVAL} if \ccode{s} contains nonalphabetic characters.


\hypertarget{func:esl_rsq_xIID()}
{\item[int esl\_rsq\_xIID(ESL\_RANDOMNESS *r, const double *p, int K, int L, ESL\_DSQ *dsq)]}

Generate an i.i.d. digital sequence of length \ccode{L} (1..L) and
leave it in \ccode{dsq}. The i.i.d. probability of each residue is
given in the probability vector \ccode{p}, and the number of
possible residues (the alphabet size) is given by \ccode{K}.
(Only the alphabet size \ccode{K} is needed here, as opposed to
a digital \ccode{ESL\_ALPHABET}, but the caller presumably
has a digital alphabet.) The caller must provide a \ccode{dsq}
allocated for at least \ccode{L+2} residues of type \ccode{ESL\_DSQ},
room for \ccode{L} residues and leading/trailing digital sentinel bytes.

\ccode{esl\_rsq\_xfIID()} does the same, but for a
single-precision float vector \ccode{p} rather than a
double-precision vector \ccode{p}.

As a special case, if \ccode{p} is \ccode{NULL}, sample residues uniformly.



\hypertarget{func:esl_rsq_SampleDirty()}
{\item[int esl\_rsq\_SampleDirty(ESL\_RANDOMNESS *rng, ESL\_ALPHABET *abc, double **byp\_p, int L, ESL\_DSQ *dsq)]}

Using random number generator \ccode{rng}, use probability
vector \ccode{p} to sample an iid digital sequence in alphabet
\ccode{abc} of length \ccode{L}. Store it in \ccode{dsq}. 

The \ccode{dsq} space, allocated by the caller, has room for
at least \ccode{L+2} residues, counting the digital
sentinels. 

Probability vector \ccode{p} has \ccode{Kp} terms, and sums to 1.0
over them. The probabilities in \ccode{p} for residues \ccode{K},
\ccode{Kp-2}, and \ccode{Kp-1} (gap, nonresidue, missing) are
typically zero, to generate a standard unaligned digital
sequence with degenerate residues. To sample a random
"alignment", \ccode{p[K]} is nonzero.

If \ccode{p} is \ccode{NULL}, then we sample a probability vector
according to the following rules, which generates
*ungapped* dirtied random sequences:
1. Sample pc, the probability of canonical
vs. noncanonical residues, uniformly on [0,1).
2. Sample a p[] uniformly for canonical residues
\ccode{0..K-1}, and renormalize by multiplying by pc.
Sample a different p[] uniformly for noncanonical
residues \ccode{K+1..Kp-3}, and renormalize by (1-pc).
3. p[] = 0 for gap residue K, nonresidue Kp-2, 
missing residue Kp-1.
This usage is mainly intended to make it easy to
sample dirty edge cases for automated tests.

Returns \ccode{eslOK} on success

Throws \ccode{eslEMEM} on allocation failure.


\hypertarget{func:esl_rsq_XShuffle()}
{\item[int esl\_rsq\_XShuffle(ESL\_RANDOMNESS *r, const ESL\_DSQ *dsq, int L, ESL\_DSQ *shuffled)]}

Given a digital sequence \ccode{dsq} of length \ccode{L} residues,
shuffle it, and leave the shuffled version in \ccode{shuffled}.

Caller provides allocated storage for \ccode{shuffled} for at
least the same length as \ccode{dsq}. 

\ccode{shuffled} may also point to the same storage as \ccode{dsq},
in which case \ccode{dsq} is shuffled in place.

Returns \ccode{eslOK} on success.


\hypertarget{func:esl_rsq_XShuffleDP()}
{\item[int esl\_rsq\_XShuffleDP(ESL\_RANDOMNESS *r, const ESL\_DSQ *dsq, int L, int K, ESL\_DSQ *shuffled)]}

Same as \ccode{esl\_rsq\_CShuffleDP()}, except for a digital
sequence \ccode{dsq} of length \ccode{L}, encoded in a digital alphabet
of \ccode{K} residues. 

\ccode{dsq} may only consist of residue codes \ccode{0..K-1}; if it
contains gaps, degeneracies, or missing data, pass the alphabet's
\ccode{Kp} size, not its canonical \ccode{K}.

If \ccode{L} $\leq 2$, this is a no-op; \ccode{shuffled} is a copy of \ccode{dsq}.

Returns \ccode{eslOK} on success.

Throws \ccode{eslEINVAL} if \ccode{s} contains digital residue codes
outside the range \ccode{0..K-1}.
\ccode{eslEMEM} on allocation failure.


\hypertarget{func:esl_rsq_XShuffleKmers()}
{\item[int esl\_rsq\_XShuffleKmers(ESL\_RANDOMNESS *r, const ESL\_DSQ *dsq, int L, int K, ESL\_DSQ *shuffled)]}

Same as \ccode{esl\_rsq\_CShuffleKmers()}, but shuffle digital 
sequence \ccode{dsq} of length \ccode{L} into digital result \ccode{shuffled}.

Returns \ccode{eslOK} on success.

Throws \ccode{eslEMEM} on allocation error.


\hypertarget{func:esl_rsq_XReverse()}
{\item[int esl\_rsq\_XReverse(const ESL\_DSQ *dsq, int L, ESL\_DSQ *rev)]}

Given a digital sequence \ccode{dsq} of length \ccode{L}, return
reversed version of it in \ccode{rev}. 

Caller provides storage in \ccode{rev} for at least
\ccode{(L+2)*sizeof(ESL\_DSQ)}.

\ccode{s} and \ccode{rev} can point to the same storage, in which
case \ccode{s} is reversed in place.

Returns \ccode{eslOK} on success.


\hypertarget{func:esl_rsq_XShuffleWindows()}
{\item[int esl\_rsq\_XShuffleWindows(ESL\_RANDOMNESS *r, const ESL\_DSQ *dsq, int L, int w, ESL\_DSQ *shuffled)]}

Given a digital sequence \ccode{dsq} of length \ccode{L}, shuffle
residues in nonoverlapping windows of width \ccode{w}, and put
the result in \ccode{shuffled}.  See [Pearson88].

Caller provides storage in \ccode{shuffled} for at least
\ccode{(L+2)*sizeof(ESL\_DSQ)}.

\ccode{dsq} and \ccode{shuffled} can be identical to shuffle in place.



\hypertarget{func:esl_rsq_XMarkov0()}
{\item[int esl\_rsq\_XMarkov0(ESL\_RANDOMNESS *r, const ESL\_DSQ *dsq, int L, int K, ESL\_DSQ *markoved)]}

Same as \ccode{esl\_rsq\_CMarkov0()}, except for a digital
sequence \ccode{dsq} of length \ccode{L}, encoded in a digital 
alphabet of \ccode{K} residues; caller provides storage
for the randomized sequence \ccode{markoved} for at least 
\ccode{L+2} \ccode{ESL\_DSQ} residues, including the two flanking
sentinel bytes.

\ccode{dsq} therefore may only consist of residue codes
in the range \ccode{0..K-1}. If it contains gaps,
degeneracies, or missing data, pass the alphabet's
\ccode{Kp} size, not its canonical \ccode{K}.

Returns \ccode{eslOK} on success.

Throws \ccode{eslEINVAL} if \ccode{s} contains digital residue codes outside
the range \ccode{0..K-1}.
\ccode{eslEMEM} on allocation failure.


\hypertarget{func:esl_rsq_XMarkov1()}
{\item[int esl\_rsq\_XMarkov1(ESL\_RANDOMNESS *r, const ESL\_DSQ *dsq, int L, int K, ESL\_DSQ *markoved)]}

Same as \ccode{esl\_rsq\_CMarkov1()}, except for a digital
sequence \ccode{dsq} of length \ccode{L}, encoded in a digital 
alphabet of \ccode{K} residues. Caller provides storage
for the randomized sequence \ccode{markoved} for at least 
\ccode{L+2} \ccode{ESL\_DSQ} residues, including the two flanking
sentinel bytes.

\ccode{dsq} and \ccode{markoved} can be point to the same storage, in which
case \ccode{dsq} is randomized in place, destroying the original
string.

\ccode{dsq} therefore may only consist of residue codes
in the range \ccode{0..K-1}. If it contains gaps,
degeneracies, or missing data, pass the alphabet's
\ccode{Kp} size, not its canonical \ccode{K}.

If \ccode{L} $\leq 2$, this is a no-op; \ccode{markoved} is a copy of \ccode{dsq}.

Returns \ccode{eslOK} on success.

Throws \ccode{eslEINVAL} if \ccode{s} contains digital residue codes outside
the range \ccode{0..K-1}.
\ccode{eslEMEM} on allocation failure.


\end{sreapi}

