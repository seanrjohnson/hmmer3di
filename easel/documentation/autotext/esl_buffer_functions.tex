\begin{sreapi}
\hypertarget{func:esl_buffer_Open()}
{\item[int esl\_buffer\_Open(const char *filename, const char *envvar, ESL\_BUFFER **ret\_bf)]}

Open \ccode{filename} for parsing. Return an open
\ccode{ESL\_BUFFER} for it.

The standard Easel idiom allows reading from standard
input (pass \ccode{filename} as '-'), allows reading gzip'ed
files automatically (any \ccode{filename} ending in \ccode{.gz} is
opened as a pipe from \ccode{gzip -dc}), and allows using an
environment variable to specify a colon-delimited list
of directories in which \ccode{filename} may be found. Normal
files are memory mapped (if \ccode{mmap()} is available) when
they are large, and slurped into memory if they are
small.

If \ccode{filename} is '-' (a single dash character), 
capture the standard input stream rather than 
opening a file; return \ccode{eslOK}.

Else, try to find \ccode{filename} in a directory \ccode{d},
starting with the current working directory. If
\ccode{./filename} is found (note that \ccode{filename} may include
a relative path), directory \ccode{d} is \ccode{.}.  Else, if
\ccode{envvar} is non-\ccode{NULL}, check the environment variable
\ccode{envvar} for a colon-delimited list of directories, and
for each directory \ccode{d} in that list, try to find
\ccode{d/filename}. Use the first \ccode{d} that succeeds. If
none succeed, return \ccode{eslENOTFOUND}.

Now open the file. If \ccode{filename} ends in \ccode{.gz}, 'open'
it by running \ccode{gzip -dc d/filename 2}/dev/null>,
capturing the standard output from gunzip decompression
in the \ccode{ESL\_BUFFER}. Otherwise, open \ccode{d/filename} as a
normal file. If its size is not more than
\ccode{eslBUFFER\_SLURPSIZE} (default 4 MB), it is slurped into
memory; else, if \ccode{mmap()} is available, it is memory
mapped; else, it is opened as a read-only binary stream
with \ccode{fopen()} in mode \ccode{"rb"}.

Returns \ccode{eslOK} on success; \ccode{*ret\_bf} is the new \ccode{ESL\_BUFFER}.

\ccode{eslENOTFOUND} if file isn't found or isn't readable.
\ccode{eslFAIL} if gzip -dc fails on a .gz file, probably 
because a gzip executable isn't found in PATH. 

On any normal error, \ccode{*ret\_bf} is still returned,
in an unset state, with a user-directed error message
in \ccode{*ret\_bf->errmsg}.

Throws \ccode{eslESYS} on system call failures (such as fread()).
\ccode{eslEMEM} on allocation failure.
Now \ccode{*ret\_bf} is \ccode{NULL}.


\hypertarget{func:esl_buffer_OpenFile()}
{\item[int esl\_buffer\_OpenFile(const char *filename, ESL\_BUFFER **ret\_bf)]}

Open \ccode{filename} for reading. Return an open \ccode{ESL\_BUFFER} in
\ccode{*ret\_bf}.

\ccode{filename} may be a relative path such as \ccode{subdir/foo}
or a full path such as \ccode{/my/dir/foo}.

On a POSIX-compliant system, large files are memory 
mapped, and small files are just slurped into memory.

On non-POSIX systems, the file is opened as a stream.
On a short initial read (if the file size is smaller than
the buffer page size), the file is considered to be
completely slurped.

Returns \ccode{eslOK} on success; \ccode{*ret\_bf} is new \ccode{ESL\_BUFFER}.

\ccode{eslENOTFOUND} if \ccode{filename} isn't found or isn't readable.

On normal errors, a new \ccode{*ret\_bf} is still returned, in
an unset state, with a user-directed error message in
\ccode{*ret\_bf->errmsg}.

Throws \ccode{eslEMEM} on allocation failure.


\hypertarget{func:esl_buffer_OpenPipe()}
{\item[int esl\_buffer\_OpenPipe(const char *filename, const char *cmdfmt, ESL\_BUFFER **ret\_bf)]}

Run the command \ccode{cmdfmt} on \ccode{filename} and capture its \ccode{stdout}
stream for parsing. Return the open \ccode{ESL\_BUFFER} in
\ccode{*ret\_bf}.

\ccode{cmdfmt} has a restricted format; it is a \ccode{printf()}-style
format string with a single \ccode{\%s}, where \ccode{filename} is to
be substituted. An example \ccode{cmdfmt} is "gzip -dc %s
2>/dev/null".

\ccode{filename} is checked for existence and read permission
before a command line is constructed.

\ccode{filename} may be \ccode{NULL}. In this case, \ccode{cmdfmt} is
assumed to be be the complete command, and (obviously)
the diagnostic check for \ccode{filename}
existence/readability is skipped. This gives you some
ability to skip the restricted single-argument format of
\ccode{cmdfmt}.  If you need to do something fancier with a
pipe, you can always open and manage it yourself and use
\ccode{esl\_buffer\_OpenStream()}.

\ccode{popen()} executes the command under \ccode{/bin/sh}.

The \ccode{stderr} stream of the command should almost
certainly be redirected (else it will appear on output
of your program). In general it should be discarded
to \ccode{/dev/null}. One of the only signs of a command
failure is that the command produces a "short read", of
less than \ccode{bf->pagesize} (and often 0, on a complete
failure, if \ccode{stderr} has been discarded).  If \ccode{stderr}
is longer than the buffer's \ccode{pagesize}, we may not
accurately detect error conditions. If you must capture
\ccode{stderr} (for example with a \ccode{cmdfmt} like
"gzip -dc %s 2>&1") be aware that the parser may
see that output as "successful" execution, if it's long
enough.

The reason to pass \ccode{cmdfmt} and \ccode{filename} separately is
to enable better error diagnostics. \ccode{popen()} itself
tends to "succeed" whether the command or the file exist
or not.  By having \ccode{filename}, we can check for its
existence/readability first.

The reason that error checking \ccode{popen()} isn't entirely
straightforward is that we don't see the exit status of
the command until we \ccode{pclose()}. We can only \ccode{pclose()}
when we're done loading data from the file, and that
only happens here on a short initial read. If we do get
a short read, we \ccode{pclose()}, get and check the command's
exit status, and return the \ccode{ESL\_BUFFER} in an
\ccode{eslBUFFER\_ALLFILE} state with \ccode{bf->cmdline} set.

Returns \ccode{eslOK} on success, and \ccode{*ret\_bf} is the new \ccode{ESL\_BUFFER}.

\ccode{eslENOTFOUND} if \ccode{filename} isn't found or isn't readable.

\ccode{eslFAIL} if the constructed command fails - which
usually means that the program isn't found or isn't
executable, or that the command returned nonzero
(quickly, i.e. with zero or little output and a 'short
read').

On any normal error, the \ccode{*ret\_bf} is returned (in an
\ccode{eslBUFFER\_UNSET} state) and \ccode{bf->errmsg} contains a
user-directed error message.

Throws \ccode{eslESYS} on \ccode{*sprintf()} or \ccode{fread()} failure.
\ccode{eslEMEM} on allocation failure.

On any exception, \ccode{*ret\_bf} is NULL.


\hypertarget{func:esl_buffer_OpenMem()}
{\item[int esl\_buffer\_OpenMem(const char *p, esl\_pos\_t n, ESL\_BUFFER **ret\_bf)]}

Given a buffer or string \ccode{p} of length \ccode{n}, turn it into
an \ccode{ESL\_BUFFER}. Return the new buffer in \ccode{*ret\_bf}.

The memory for \ccode{p} is still managed by the caller. 
Caller should free it, if necessary, only after the 
\ccode{ESL\_BUFFER} has been closed. 

As a special case, if \ccode{n} is -1, \ccode{p} is assumed to be a
\verb+\0+-terminated string and its length is calculated with
\ccode{strlen()}.

Returns \ccode{eslOK} on success, and \ccode{*ret\_bf} points to new buffer.

Throws \ccode{eslEMEM} on allocation failure.
On any exception, \ccode{*ret\_bf} is \ccode{NULL}.


\hypertarget{func:esl_buffer_OpenStream()}
{\item[int esl\_buffer\_OpenStream(FILE *fp, ESL\_BUFFER **ret\_bf)]}

Given an open stream \ccode{fp} for reading, create an
\ccode{ESL\_BUFFER} around it.

\ccode{fp} is often \ccode{stdin}, for example.

The caller remains responsible for closing \ccode{fp}, if it
opened it. 

Returns \ccode{eslOK} on success, and \ccode{*ret\_bf} points to a new \ccode{ESL\_BUFFER}.

Throws \ccode{eslEINVAL}: \ccode{fp} is NULL, in error state, or already at eof before any reading occurs.
\ccode{eslESYS} : fread() failed
\ccode{eslEMEM} : an allocation failed


\hypertarget{func:esl_buffer_Close()}
{\item[int esl\_buffer\_Close(ESL\_BUFFER *bf)]}

Close the input buffer \ccode{bf}, freeing all resources that it
was responsible for.

Returns \ccode{eslOK} on success.

Throws \ccode{eslESYS} on a system call failure such as \ccode{munmap()}, \ccode{pclose()}, or \ccode{fclose()}.



\hypertarget{func:esl_buffer_GetOffset()}
{\item[esl\_pos\_t esl\_buffer\_GetOffset(ESL\_BUFFER *bf)]}

Returns the current offset position of the parser
in the input buffer: \ccode{bf->baseoffset + bf->pos}.


\hypertarget{func:esl_buffer_SetOffset()}
{\item[int esl\_buffer\_SetOffset(ESL\_BUFFER *bf, esl\_pos\_t offset)]}

Set the buffer's internal state (\ccode{bf->pos}) to position
\ccode{offset} in the input. Load new data into the buffer if
necessary.

In modes where \ccode{bf->mem} contains the whole input
(ALLFILE, MMAP, STRING), this always works.

In modes where we're reading a
nonrewindable/nonpositionable stream (STREAM, CMDPIPE),
\ccode{offset} may be at or ahead of the current position, but
rewinding to an offset behind the current position only
works if \ccode{offset} is within the current buffer
window. If the caller knows it wants to return to some
\ccode{offset} later, it should set an anchor to make sure it
stays in the buffer. New data may need to be read into
\ccode{bf->mem} to assure \ccode{pagesize} bytes are available. If
an anchor is set, this may require reoffset and/or
reallocation of \ccode{bf->mem}.

FILE mode is handled as above, but additionally, if no
anchor is set and \ccode{offset} is not in the current buffer,
\ccode{fseeko()} is used to reposition in the open file. If
\ccode{fseeko()} is unavailable (non-POSIX compliant systems),
FILE mode is handled like other streams, with limited
rewind ability.

Returns \ccode{eslOK} on success.

Throws \ccode{eslEINVAL} if \ccode{offset} is invalid, either because it 
would require rewinding the (nonrewindable) stream, 
or because it's beyond the end.
\ccode{eslESYS} if a system call fails, such as fread().
\ccode{eslEMEM} on allocation failure.
\ccode{eslEINCONCEIVABLE} if \ccode{bf} internal state is corrupt.


\hypertarget{func:esl_buffer_SetAnchor()}
{\item[int esl\_buffer\_SetAnchor(ESL\_BUFFER *bf, esl\_pos\_t offset)]}

Set an anchor at byte \ccode{offset} (in input coords) in
input \ccode{bf}: which means, keep everything from this byte
on in buffer memory, until anchor is raised.

The presence of an anchor affects new reads from \ccode{fp};
\ccode{mem[r..n-1]} are protected from overwrite, and may be
moved to \ccode{mem[0..n-r-1]} as new data is read from the
stream.  Anchors are only needed for input streams that
we read chunkwise.  If entire input is already in \ccode{bf},
setting an anchor is a no-op.

In general, the caller should remember what anchor(s) it
sets, so it can raise them later with
\ccode{esl\_buffer\_RaiseAnchor()}.

Byte \ccode{offset} must be in the current buffer window. If
not, an \ccode{eslEINVAL} exception is thrown.

Only one anchor is active at a time. If an anchor is
already set for \ccode{bf}, the most upstream one is used.

Returns \ccode{eslOK} on success.

Throws \ccode{eslEINVAL} if \ccode{offset} is not in current buffer window.


\hypertarget{func:esl_buffer_SetStableAnchor()}
{\item[int esl\_buffer\_SetStableAnchor(ESL\_BUFFER *bf, esl\_pos\_t offset)]}

Same as \ccode{esl\_buffer\_SetAnchor()}, except the anchor is
such that all pointers returned by \ccode{\_Get*()} functions
(i.e. as opposed to just the last \ccode{\_Get*} function)
will remain valid at least until the anchor is raised.

The main use of this is when the caller wants to get
multiple lines or tokens in the input before parsing 
them. 

A stable anchor prevents buffer refills/reloads from
moving the internal memory around while the anchor is
in place.

Returns \ccode{eslOK} on success.

Throws \ccode{eslEINVAL} if \ccode{offset} is not in current buffer window.



\hypertarget{func:esl_buffer_RaiseAnchor()}
{\item[int esl\_buffer\_RaiseAnchor(ESL\_BUFFER *bf, esl\_pos\_t offset)]}

Declare that an anchor previously set at \ccode{offset}
in buffer \ccode{bf} may be raised. 

\ccode{offset} is in absolute input coordinates (\ccode{0..len-1} for
an input of length \ccode{len}). Because it's supposed to be
anchored, this position ought to be in the current
buffer window. If an anchor is in effect in \ccode{bf}, 
\ccode{offset} should be at or distal to that anchor.

The buffer's memory and position are not changed yet.  A
caller can raise an anchor and still assume that the
buffer contains all data from that anchor, until the
next call to something that would alter the buffer.

Returns \ccode{eslOK} on success.

Throws (none)


\hypertarget{func:esl_buffer_Get()}
{\item[int esl\_buffer\_Get(ESL\_BUFFER *bf, char **ret\_p, esl\_pos\_t *ret\_n)]}

Given a buffer \ccode{bf}, return a pointer to the current
parsing position in \ccode{*ret\_p}, and the number of valid
bytes from that position in \ccode{*ret\_n}.

If buffer is at EOF (no valid bytes remain), returns
\ccode{eslEOF} with \ccode{NULL} in \ccode{*ret\_p} and 0 in \ccode{*ret\_n}.

The buffer's parsing position \ccode{bf->pos} is NOT 
changed. Another \ccode{Get()} call will return exactly
the same \ccode{p} and \ccode{n}. Each \ccode{Get()} call is generally
followed by a \ccode{Set()} call. It's the \ccode{Set()} call
that moves \ccode{bf->pos} and refills the buffer.

Assumes that the buffer \ccode{bf} is correctly loaded,
with either at least \ccode{pagesize} bytes after the 
parser position, or near/at EOF.

Returns \ccode{eslOK} on success;
\ccode{eslEOF} if no valid bytes remain in the input, or if
\ccode{*ret\_n} is less than \ccode{nrequest}. 

Throws \ccode{eslEMEM} on allocation failure. 
\ccode{eslESYS} if fread() fails mysteriously.
\ccode{eslEINCONCEIVABLE} if internal state of \ccode{bf} is corrupt.


\hypertarget{func:esl_buffer_Set()}
{\item[int esl\_buffer\_Set(ESL\_BUFFER *bf, char *p, esl\_pos\_t nused)]}

Reset the state of buffer \ccode{bf}: we were recently
given a pointer \ccode{p} by an \ccode{esl\_buffer\_Get()} call
and we parsed \ccode{nused} bytes starting at \ccode{p[0]}. 

\ccode{bf->pos} is set to point at \ccode{p+nused}, and we
reload the buffer (if necessary) to try to have at
least \ccode{bf->pagesize} bytes of input following that
position.

One use is in raw parsing, where we stop parsing
somewhere in the buffer:
\begin{cchunk}
esl_buffer_Get(bf, &p, &n);
(do some stuff on p[0..n-1], using up \ccode{nused} bytes)
esl_buffer_Set(bf, p, nused);
\end{cchunk}
This includes the case of nused=n, where we parse the
whole buffer that Get() gave us, and the Set() call may
be needed to load new input data before the next Get().

Another use is an idiom for peeking at a token, line, or
a number of bytes without moving the parser position:
\begin{cchunk}
esl_buffer_GetLine(bf, &p, &n);
(do we like what we see in p[0..n-1]? no? then put it back)
esl_buffer_Set(bf, p, 0);
\end{cchunk}

Because it is responsible for loading new input as
needed, Set() may reoffset and reallocate \ccode{mem}. If the
caller wants an anchor respected, it must make sure that
anchor is still in effect; i.e., a caller that is
restoring state to an \ccode{ESL\_BUFFER} should call Set()
BEFORE calling RaiseAnchor().

As a special case, if \ccode{p} is NULL, then \ccode{nused} is
ignored, \ccode{bf->pos} is left whereever it was, and the
only thing the \ccode{Set()} attempts to do is to fulfill the
pagesize guarantee from the current position. If a
\ccode{NULL} \ccode{p} has been returned by a Get*() call because we
reached EOF, for example in some parsing loop that the
EOF has broken us out of, it is safe to call
\ccode{esl\_buffer\_Set(bf, NULL, 0)}: this is a no-op on a
buffer that is at EOF.

Returns \ccode{eslOK} on success.

Throws \ccode{eslEMEM} on allocation failure. 
\ccode{eslESYS} if fread() fails mysteriously.
\ccode{eslEINCONCEIVABLE} if internal state of \ccode{bf} is corrupt.


\hypertarget{func:esl_buffer_GetLine()}
{\item[int esl\_buffer\_GetLine(ESL\_BUFFER *bf, char **opt\_p, esl\_pos\_t *opt\_n)]}

Get a pointer \ccode{*opt\_p} to the next line in buffer \ccode{bf},
and the length of the line in \ccode{*opt\_n} (in bytes, and
exclusive of newline bytes). Advance buffer position
past (one) newline, putting it on the next valid data
byte. Thus \ccode{p[0..n-1]} is one data line. It is not
NUL-terminated.

\ccode{bf}'s buffer may be re(al)located as needed, to get
the whole line into the current window.

Because the caller only gets a pointer into \ccode{bf}'s 
internal state, no other \ccode{esl\_buffer} functions 
should be called until the caller is done with \ccode{p}.

To peek at next line, use Set to restore \ccode{bf}'s state:
\begin{cchunk}
esl_buffer_GetLine(bf, &p, &n);
esl_buffer_Set(bf, p, 0);           
\end{cchunk}

Returns \ccode{eslOK} on success.  \ccode{*opt\_p} is a valid pointer into \ccode{bf}'s buffer,
and \ccode{*opt\_n} is >=0. (0 would be an empty line.)

\ccode{eslEOF} if there's no line (even blank).
On EOF, \ccode{*opt\_p} is NULL and \ccode{*opt\_n} is 0.

Throws \ccode{eslEMEM} if allocation fails.
\ccode{eslESYS} if a system call such as fread() fails unexpectedly
\ccode{eslEINCONCEIVABLE} if \ccode{bf} internal state is corrupt.


\hypertarget{func:esl_buffer_FetchLine()}
{\item[int esl\_buffer\_FetchLine(ESL\_BUFFER *bf, char **opt\_p, esl\_pos\_t *opt\_n)]}

Get the next line from the buffer \ccode{bf}, starting from its
current position.  Return an allocated copy of it in
\ccode{*opt\_p}, and its length in bytes in \ccode{*opt\_n}.  Advance
the buffer position past (one) newline, putting it on
the next valid byte. The last line in a file does not
need to be terminated by a newline. The returned memory is not
NUL-terminated.

If the next line is empty (solely a newline character),
returns \ccode{eslOK}, but with \ccode{*opt\_p} as \ccode{NULL} and
\ccode{*opt\_n} as 0.

Caller is responsible for free'ing \ccode{*opt\_p}. 

Because \ccode{*ret\_p} is a copy of \ccode{bf}'s internal buffer,
caller may continue to manipulate \ccode{bf}, unlike
\ccode{esl\_buffer\_GetLine()}.

Returns \ccode{eslOK} on success.  Either \ccode{*opt\_p} is an allocated copy
of next line and \ccode{*opt\_n} is $>0$, or \ccode{*opt\_p} is \ccode{NULL}
and \ccode{*opt\_n} is 0 (in the case where the line is empty, 
 8            immediately terminated by newline, such as \verb+"\n"+.).

\ccode{eslEOF} if there's no line (even blank).
On EOF, \ccode{*opt\_p} is NULL and \ccode{*opt\_n} is 0.

Throws \ccode{eslEMEM} if allocation fails.
\ccode{eslESYS} if a system call such as fread() fails unexpectedly
\ccode{eslEINCONCEIVABLE} if \ccode{bf} internal state is corrupt.


\hypertarget{func:esl_buffer_FetchLineAsStr()}
{\item[int esl\_buffer\_FetchLineAsStr(ESL\_BUFFER *bf, char **opt\_s, esl\_pos\_t *opt\_n)]}

Same as \ccode{esl\_buffer\_FetchLine()} except the
returned line is \ccode{NUL}-terminated and can be treated
as a string.

Returns \ccode{eslOK} on success.  \ccode{*opt\_p} is an allocated copy
of next line and \ccode{*opt\_n} is >=0. (0 would be an empty line
terminated by newline, such as \verb+\n+.)

\ccode{eslEOF} if there's no line (even blank).
On EOF, \ccode{*opt\_p} is NULL and \ccode{*opt\_n} is 0.

Throws \ccode{eslEMEM} if allocation fails.
\ccode{eslEINVAL} if an anchoring attempt is invalid
\ccode{eslESYS} if a system call such as fread() fails unexpectedly
\ccode{eslEINCONCEIVABLE} if \ccode{bf} internal state is corrupt.


\hypertarget{func:esl_buffer_GetToken()}
{\item[int esl\_buffer\_GetToken(ESL\_BUFFER *bf, const char *sep, char **opt\_tok, esl\_pos\_t *opt\_n)]}

Find the next token in \ccode{bf} delimited by the separator
characters in \ccode{sep} or newline. Return a pointer
to that token in \ccode{*opt\_tok}, and its length in \ccode{*opt\_n}.
A 'token' consists of one or more characters that are
neither in \ccode{sep} nor a newline (verb+\r+ or \verb+\n+).

Because the caller only gets a pointer into the buffer's
current memory, it should not call another
\ccode{esl\_buffer\_*} function until it's done with the token
or made a copy of it.

In detail, starting from the buffer \ccode{bf}'s current
point; first skip past any leading characters in \ccode{sep}. If EOF
is reached, return \ccode{eslEOF}. If the point is now on a
newline, skip past it and return \ccode{eslEOL}. Set an anchor
at the current point. Count how many non-separator
characters \ccode{n} occur from the current point \ccode{p}
(expand/refill buffer as needed), and define the token
as \ccode{p[0..n-1]}. Skip any trailing characters in \ccode{sep}. 
Set the point to the start of the next token (a char not
in \ccode{sep}.) Release the anchor and return.

If caller knows how many tokens it expects on each line,
it should not include \verb+"\r\n"+ in its \ccode{sep}. This way,
hitting a newline will cause a \ccode{eslEOL} return. The
caller can check for expected or unexpected \ccode{EOL}'s.

If the caller doesn't care how many tokens it allows per
line, it should include \verb+"\r\n"+ in its \ccode{sep}. Now
newlines will be skipped like any other separator
character, and the only normal returns are \ccode{eslEOF} and
\ccode{eslOK}.

Returns \ccode{eslOK} if a token is found; \ccode{*ret\_p} points to it,
and \ccode{*opt\_n} is its length in chars (> 0). The current
point is at the start of the next token.

\ccode{eslEOF} if the input ends before any token is found.
Now \ccode{*ret\_p} is \ccode{NULL} and \ccode{*opt\_n} is 0. The current
point is at EOF.

\ccode{eslEOL} if a line ends before a token is found.  (This
case only arises if *sep does not contain newline.)
Now \ccode{*ret\_p} is \ccode{NULL} and \ccode{*opt\_n} is 0. The current
point is at the next character following the newline.

\ccode{bf->mem} may be modified and/or reallocated, if new
input reads are required to find the entire token.

Throws \ccode{eslEMEM} if an allocation fails.
Now \ccode{*ret\_p} is \ccode{NULL} and \ccode{*opt\_n} is 0. The
current point is undefined.


\hypertarget{func:esl_buffer_FetchToken()}
{\item[int esl\_buffer\_FetchToken(ESL\_BUFFER *bf, const char *sep, char **opt\_tok, esl\_pos\_t *opt\_n)]}

Essentially the same as \ccode{esl\_buffer\_GetToken()}, except a
copy of the token is made into newly allocated memory,
and a pointer to this memory is returned in \ccode{*opt\_tok}.
The caller is responsible for freeing the memory. 

The token is raw memory, not a \ccode{NUL}-terminated string. 
To fetch tokens as \ccode{NUL}-terminated strings, see 
\ccode{esl\_buffer\_GetTokenAsStr()}.

Returns \ccode{eslOK} if a token is found; \ccode{*ret\_p} points to it,
and \ccode{*opt\_n} is its length in chars (> 0). The current
point is at the start of the next token.

\ccode{eslEOF} if the input ends before any token is found.
Now \ccode{*ret\_p} is \ccode{NULL} and \ccode{*opt\_n} is 0. The current
point is at EOF.

\ccode{eslEOL} if a line ends before a token is found.  (This
case only arises if *sep does not contain newline.)
Now \ccode{*ret\_p} is \ccode{NULL} and \ccode{*opt\_n} is 0. The current
point is at the next character following the newline.

\ccode{bf->mem} may be modified and/or reallocated, if new
input reads are required to find the entire token.

Throws \ccode{eslEMEM} if an allocation fails.
Now \ccode{*ret\_p} is \ccode{NULL} and \ccode{*opt\_n} is 0. The
current point is undefined.


\hypertarget{func:esl_buffer_FetchTokenAsStr()}
{\item[int esl\_buffer\_FetchTokenAsStr(ESL\_BUFFER *bf, const char *sep, char **opt\_tok, esl\_pos\_t *opt\_n)]}

Essentially the same as \ccode{esl\_buffer\_FetchToken()} 
except the copied token is \verb+\0+-terminated so it
can be treated as a string.

Returns \ccode{eslOK} if a token is found; \ccode{*ret\_p} points to it,
and \ccode{*opt\_n} is its length in chars (> 0). The current
point is at the start of the next token.

\ccode{eslEOF} if the input ends before any token is found.
Now \ccode{*ret\_p} is \ccode{NULL} and \ccode{*opt\_n} is 0. The current
point is at EOF.

\ccode{eslEOL} if a line ends before a token is found.  (This
case only arises if *sep does not contain newline.)
Now \ccode{*ret\_p} is \ccode{NULL} and \ccode{*opt\_n} is 0. The current
point is at the next character following the newline.

\ccode{bf->mem} may be modified and/or reallocated, if new
input reads are required to find the entire token.

Throws \ccode{eslEMEM} if an allocation fails.
Now \ccode{*ret\_p} is \ccode{NULL} and \ccode{*opt\_n} is 0. The
current point is undefined.


\hypertarget{func:esl_buffer_Read()}
{\item[int esl\_buffer\_Read(ESL\_BUFFER *bf, size\_t nbytes, void *p)]}

Given an input buffer \ccode{bf}, read exactly \ccode{nbytes}
characters into memory \ccode{p} provided by the caller.

Suitable for copying known-width scalars from
binary files, as in:
\begin{cchunk}
char c;
int  n;
esl_buffer_Read(bf, sizeof(char), c);
esl_buffer_Read(bf, sizeof(int),  n);
\end{cchunk}

Returns \ccode{eslOK} on success; \ccode{p} contains exactly \ccode{nbytes}
of data from \ccode{bf}, and the point is advanced by \ccode{nbytes}.

\ccode{eslEOF} if less than \ccode{nbytes} characters remain 
in \ccode{bf}. Point is unchanged.

Throws \ccode{eslEMEM} if an allocation fails.
\ccode{eslESYS} if an fread() fails mysteriously.
\ccode{eslEINCONCEIVABLE} if internal state of \ccode{bf} is corrupted.


\end{sreapi}

