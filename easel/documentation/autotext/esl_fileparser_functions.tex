\begin{sreapi}
\hypertarget{func:esl_fileparser_Open()}
{\item[int esl\_fileparser\_Open(const char *filename, const char *envvar, ESL\_FILEPARSER **ret\_efp)]}

Opens \ccode{filename} for reading. 

As a special case, if \ccode{filename} is "-", set up the
fileparser to read and parse \ccode{stdin}.

\ccode{envvar} is optional name of an environment variable,
such as \ccode{BLASTDB}. This environment variable contains a
colon-delimited list of directories in which the
\ccode{filename} may lie relative to.  We looks first relative
to the current working directory, then in any
directories specified by \ccode{envvar}. If \ccode{envvar} is \ccode{NULL},
we only look in the current working directory.

Returns \ccode{eslOK} on success, and \ccode{ret\_fp} points
to a new \ccode{ESL\_FILEPARSER} object.

Returns \ccode{eslENOTFOUND} if \ccode{filename} can't
be opened for reading, and \ccode{ret\_fp} is set
to \ccode{NULL}.

Throws \ccode{eslEMEM} on allocation failure.


\hypertarget{func:esl_fileparser_Create()}
{\item[ESL\_FILEPARSER * esl\_fileparser\_Create(FILE *fp)]}

Take an open file \ccode{fp}, and transform it to
a fileparser object -- preparing to parse it
one whitespace-delimited field at a time.

Returns a new \ccode{ESL\_FILEPARSER} object, which must be 
free'd by the caller with \ccode{esl\_fileparser\_Destroy()}.

Throws \ccode{eslEMEM} if an allocation failed.



\hypertarget{func:esl_fileparser_CreateMapped()}
{\item[ESL\_FILEPARSER * esl\_fileparser\_CreateMapped(const void *buffer, int size)]}

Sets up a memory buffer to be parsed with the
file parser routines.Take an open file \ccode{fp}, and transform it to
a fileparser object -- preparing to parse it
one whitespace-delimited field at a time.

Returns a new \ccode{ESL\_FILEPARSER} object, which must be 
free'd by the caller with \ccode{esl\_fileparser\_Destroy()}.

Throws \ccode{eslEMEM} if an allocation failed.



\hypertarget{func:esl_fileparser_SetCommentChar()}
{\item[int esl\_fileparser\_SetCommentChar(ESL\_FILEPARSER *efp, char c)]}

Defines a single character \ccode{c} for comments. Anything
on a line following this character is ignored
when parsing.

Returns \ccode{eslOK} on success.


\hypertarget{func:esl_fileparser_GetToken()}
{\item[int esl\_fileparser\_GetToken(ESL\_FILEPARSER *efp, char **opt\_tok, int *opt\_toklen)]}

Sets a pointer to the next field in the 
file we're parsing.

The \ccode{opt\_tok} pointer is into an internal line buffer
that may be invalidated upon the next call to a
\ccode{fileparser} function. If you want to store it, make a
copy.

Returns \ccode{eslOK} if \ccode{tok}, \ccode{toklen} contain valid data.
\ccode{eslEOF} on normal end-of-file.

Throws \ccode{eslEMEM} if an allocation fails.



\hypertarget{func:esl_fileparser_NextLine()}
{\item[int esl\_fileparser\_NextLine(ESL\_FILEPARSER *efp)]}

Advance the parser to the next non-blank, non-comment
data line that contains at least one token. 

Upon return, \ccode{efp->buf} is a data-containing line, and
\ccode{efp->s} points to the first non-whitespace character on
it. A line-based parser can work on one or both of these.

A line-oriented but token-based parser will call
\ccode{esl\_fileparser\_GetTokenOnLine()} to extract successive
tokens from it.

A pure token-based parser will generally not call
\ccode{\_NextLine()}.  The only reason would be to skip the
remainder of a line it's in the middle of parsing, and
advance to the next one -- but that's a sort of
line-oriented thing to do.

Returns \ccode{eslOK} on success.
\ccode{eslEOF} if no more data lines remain in the file.  

Throws \ccode{eslEMEM} on allocation error.


\hypertarget{func:esl_fileparser_NextLinePeeked()}
{\item[int esl\_fileparser\_NextLinePeeked(ESL\_FILEPARSER *efp, char *prefix, int plen)]}

Sometimes we need to peek at the start of an input stream
to see whether it is in a binary format, before we start
parsing it as ASCII lines. When this happens, the caller
will typically have used \ccode{fread()} to read a fixed
number of bytes from the input stream, checked to see if
they are a magic number representing a binary format,
and found that they are not. The caller then wants to
switch to reading in ASCII format with the \ccode{fileparser}
API, but with those bytes included on the first
line. Because the file might start with comments or
blank lines that need to be skipped, we want to deal
with the peeked data in the context of the
\ccode{ESL\_FILEPARSER}. The caller cannot simply close and
reopen the stream, because the stream may be a pipe
(\ccode{stdin} or \ccode{gzip -dc} for example).

The caller passes the bytes it peeked at with \ccode{fread()}
in \ccode{prefix}, and the number of bytes it peeked at in
\ccode{plen}.

The parser is advanced to the next non-blank,
non-comment data line that contains at least one token,
taking the prepended \ccode{prefix} into account.

There is a significant flaw in this mechanism, and as a
result the caller must be able to guarantee the
following limitation. The first data-containing line
must be longer than \ccode{prefix}. It is sufficient for the
first data token to be longer than \ccode{prefix}.
(Equivalently, if \ccode{prefix} contains any data token, it
must not contain any newline \verb+\n+ after that data.)  The
reason is that we need to avoid a situation where the
concatenated prefix+nextline contains more than one data
line, because other routines in the module assume that
\ccode{efp->buf} is a single \verb+\n+-terminated line of input.  For
example: HMMER save files either start with a 4-byte
binary magic number, or with "HMMER", and "HMMER" is
longer than 4 bytes.

Returns \ccode{eslOK} on success.
\ccode{eslEOF} if no more tokens remain in the file.  

Throws \ccode{eslEMEM} on allocation error.



\hypertarget{func:esl_fileparser_GetTokenOnLine()}
{\item[int esl\_fileparser\_GetTokenOnLine(ESL\_FILEPARSER *efp, char **opt\_tok, int *opt\_toklen)]}

Same as \ccode{esl\_fileparser\_GetToken()}, except that it only
retrieves tokens from the line that the parser is
on. When it runs out of tokens on the line, it returns
\ccode{eslEOL}. This allows a caller to count the tokens on a
line (whereas \ccode{GetToken()} reads through newlines
silently).

The \ccode{opt\_tok} pointer is into an internal line buffer
that may be invalidated upon the next call to a
\ccode{fileparser} function. If you want to store it, make a
copy.

Normally, a call to \ccode{esl\_fileparser\_GetTokenOnLine()}
would be preceded by \ccode{esl\_fileparser\_NextLine()} to
position the parser on the next data line with at least
one token on it. However, you could also conceivably
call \ccode{esl\_fileparser\_GetTokenOnLine()} after one or more
calls to \ccode{esl\_fileparser\_GetToken()}, to get remaining
tokens from a given line. What you can't do is to call
\ccode{esl\_fileparser\_GetTokenOnLine()} immediately after 
opening a file; the parser won't have a line loaded yet.
(In this case, it would return \ccode{eslEOL}.)

Returns \ccode{eslOK} on success, and the token and its length are
in \ccode{opt\_tok} and \ccode{opt\_toklen}.

Returns \ccode{eslEOL} if no more tokens exist on the line;
in this case \ccode{opt\_tok} is set to \ccode{NULL} and \ccode{opt\_toklen}
to 0.


\hypertarget{func:esl_fileparser_GetRemainingLine()}
{\item[int esl\_fileparser\_GetRemainingLine(ESL\_FILEPARSER *efp, char **ret\_s)]}

Set a pointer \ccode{*ret\_s} to the rest of the current line
held by the fileparser \ccode{efp}. Trailing newline char,
if any, is removed.

Because \ccode{ret\_s} points to internal storage in the
fileparser, the caller should be finished with it before
making its next call to any fileparser function.

Any comment characters on the rest of the line are
ignored: this is designed for a case where the rest of
the line is to be read as free text.

Returns \ccode{eslOK} on success.
\ccode{eslEOL} if nothing remains on the line, and \ccode{*ret\_s}
is \ccode{NULL}.

Throws (no abnormal error conditions)


\hypertarget{func:esl_fileparser_Destroy()}
{\item[void esl\_fileparser\_Destroy(ESL\_FILEPARSER *efp)]}

Frees an open \ccode{ESL\_FILEPARSER}. The original fp is
still open - whoever opened it is still
responsible for closing it.



\hypertarget{func:esl_fileparser_Close()}
{\item[void esl\_fileparser\_Close(ESL\_FILEPARSER *efp)]}

Closes an open \ccode{ESL\_FILEPARSER}, including the 
file it opened. 


\end{sreapi}

