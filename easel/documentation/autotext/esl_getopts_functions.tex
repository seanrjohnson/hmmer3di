\begin{sreapi}
\hypertarget{func:esl_getopts_Create()}
{\item[ESL\_GETOPTS * esl\_getopts\_Create(const ESL\_OPTIONS *opt)]}

Creates an \ccode{ESL\_GETOPTS} object, given the
array of valid options \ccode{opt} (NULL-element-terminated).
Sets default values for all config 
options (as defined in \ccode{opt}).

Returns ptr to the new \ccode{ESL\_GETOPTS} object.

Throws NULL on failure, including allocation failures or
an invalid \ccode{ESL\_OPTIONS} structure.


\hypertarget{func:esl_getopts_CreateDefaultApp()}
{\item[ESL\_GETOPTS * esl\_getopts\_CreateDefaultApp(const ESL\_OPTIONS *options, int nargs, int argc, char **argv, char *banner, char *usage)]}

Carry out the usual sequence of events in initializing a
small Easel-based application: parses the command line,
process the \ccode{-h} option to produce a (single-sectioned)
help page, and check that the number of command line
options is right.

\ccode{options} is an array of \ccode{ESL\_OPTIONS} structures describing
the options, terminated by an all-\ccode{NULL} structure.

\ccode{nargs} is the number of commandline arguments
expected. If the number of commandline arguments isn't
equal to this, an error message is printed, with the
\ccode{usage} string, and \ccode{exit()} is called. If \ccode{nargs} is
-1, this check isn't done; if your program deliberately
has a variable number of commandline arguments (i.e.
if the number is unknown at compile time), pass -1
for \ccode{nargs}.

\ccode{argc} and \ccode{argv} are the command line
arguments (number and pointer array) from \ccode{main()}.

\ccode{banner} is an optional one-line description of the
program's function, such as \ccode{"compare RNA structures"}.
When the \ccode{-h} help option is selected, this description
will be combined with the program's name (the tail of
\ccode{argv[0]}) and Easel's copyright and license information
to give a header like:

\begin{cchunk}
# esl-compstruct :: compare RNA structures
# Easel 0.1 (February 2005)
# Copyright (C) 2004-2007 HHMI Janelia Farm Research Campus
# Freely licensed under the Janelia Software License.
# - - - - - - - - - - - - - - - - - - - - - - - - - - - - - - - - - - - -
\end{cchunk}

\ccode{usage} is an optional one-line description of command
line usage (without the command name), such as
\begin{cchunk}
[options] \ccode{trusted file} \ccode{test file}
\end{cchunk}
On errors, or
on the help page, this usage string is combined with 
the program's name to give a usage line like:

\begin{cchunk}
Usage: esl-compstruct [options] \ccode{trusted file} \ccode{test file}
\end{cchunk}  

\ccode{banner} and \ccode{usage} are optional, meaning that either
can be provided as \ccode{NULL} and they won't be shown.

Returns a pointer to a new \ccode{ESL\_GETOPTS} object, which contains
all the option settings and command line arguments.

On command line errors, this routine exits with abnormal
(1) status.

If the \ccode{-h} help option is seen, this routine exits with
normal (0) status after printing a help page.



\hypertarget{func:esl_getopts_Reuse()}
{\item[int esl\_getopts\_Reuse(ESL\_GETOPTS *g)]}

Reset application configuration \ccode{g} to initial defaults,
as if it were newly created (before any 
processing of environment, config files, or
command line). 

Returns \ccode{eslOK} on success.

Throws (no abnormal error conditions)



\hypertarget{func:esl_getopts_Destroy()}
{\item[void esl\_getopts\_Destroy(ESL\_GETOPTS *g)]}

Free's a created \ccode{ESL\_GETOPTS} object.

Returns void.


\hypertarget{func:esl_getopts_Dump()}
{\item[void esl\_getopts\_Dump(FILE *ofp, ESL\_GETOPTS *g)]}

Dump the state of \ccode{g} to an output stream
\ccode{ofp}, often stdout or stderr.


\hypertarget{func:esl_opt_ProcessConfigfile()}
{\item[int esl\_opt\_ProcessConfigfile(ESL\_GETOPTS *g, char *filename, FILE *fp)]}

Given an open configuration file \ccode{fp} (and
its name \ccode{filename}, for error reporting),
parse it and set options in \ccode{g} accordingly.
Anything following a \ccode{\#} in the file is a
comment. Blank (or all-comment) lines are
ignored. Data lines contain one option and
its optional argument: for example \ccode{--foo arg}
or \ccode{-a}. All option arguments are type and
range checked, as specified in \ccode{g}.

Returns \ccode{eslOK} on success.  

Returns \ccode{eslESYNTAX} on parse or format error in the
file, or f option argument fails a type or range check,
or if an option is set twice by the same config file.
In any of these "normal" (user) error cases, \ccode{g->errbuf}
is set to a useful error message to indicate the error.

Throws \ccode{eslEMEM} on allocation problem.            


\hypertarget{func:esl_opt_ProcessEnvironment()}
{\item[int esl\_opt\_ProcessEnvironment(ESL\_GETOPTS *g)]}

For any option defined in \ccode{g} that can be modified
by an environment variable, check the environment
and set that option accordingly. The value provided
by the environment is type and range checked.
When an option is turned on that has other options 
toggle-tied to it, those options are turned off.
An option's state may only be changed once by the
environment (even indirectly thru toggle-tying);
else an error is generated.

Returns \ccode{eslOK} on success, and \ccode{g} is loaded with all
options specified in the environment.

Returns \ccode{eslEINVAL} on user input problems, 
including type/range check failures, and 
sets \ccode{g->errbuf} to a useful error message.

Throws \ccode{eslEMEM} on allocation problem.            


\hypertarget{func:esl_opt_ProcessCmdline()}
{\item[int esl\_opt\_ProcessCmdline(ESL\_GETOPTS *g, int argc, char **argv)]}

Process a command line (\ccode{argc} and \ccode{argv}), parsing out
and setting application options in \ccode{g}. Option arguments
are type and range checked before they are set, if type
and range information was set when \ccode{g} was created.
When an option is set, if it has any other options
"toggle-tied" to it, those options are also turned off.

Any given option can only change state (on/off) once
per command line; trying to set the same option more than
once generates an error.

On successful return, \ccode{g} contains settings of all
command line options and their option arguments, for
subsequent retrieval by \ccode{esl\_opt\_Get*()}
functions.  \ccode{g} also contains an \ccode{optind} state variable
pointing to the next \ccode{argv[]} element that is not an
option. \ccode{esl\_opt\_GetArg()} needs this to know
where the options end and command line arguments begin
in \ccode{argv[0]}.

The parser starts with \ccode{argv[1]} and reads \ccode{argv[]} elements
in order until it reaches an element that is not an option; 
at this point, all subsequent \ccode{argv[]} elements are 
interpreted as arguments to the application.

Any \ccode{argv[]} element encountered in the command line that
starts with \ccode{- } is an option, except \ccode{- } or \ccode{-- } by
themselves. \ccode{- } by itself is interpreted as a command
line argument (usually meaning ``read from stdin instead
of a filename''). \ccode{-- } by itself is interpreted as
``end of options''; all subsequent \ccode{argv[]} elements are
interpreted as command-line arguments even if they
begin with \ccode{- }. 

Returns \ccode{eslOK} on success. \ccode{g} is loaded with
all option settings specified on the cmdline.
Returns \ccode{eslEINVAL} on any cmdline parsing problem,
including option argument type/range check failures,
and sets \ccode{g->errbuf} to a useful error message for
the user.

Throws \ccode{eslEMEM} on allocation problem.           


\hypertarget{func:esl_opt_ProcessSpoof()}
{\item[int esl\_opt\_ProcessSpoof(ESL\_GETOPTS *g, const char *cmdline)]}

Process the string \ccode{cmdline} as if it were a 
complete command line. 

Essentially the same as \ccode{esl\_opt\_ProcessCmdline()}
except that whitespace-delimited tokens first need to be
identified in the \ccode{cmdline} first, then passed as
\ccode{argc},\ccode{argv} to \ccode{esl\_opt\_ProcessCmdline()}.

Returns \ccode{eslOK} on success, and \ccode{g} is loaded with the
application configuration.

Returns \ccode{eslEINVAL} on any parsing problem, and sets
\ccode{g->errbuf} to an informative error message.

Throws \ccode{eslEMEM} on allocation failure.



\hypertarget{func:esl_opt_VerifyConfig()}
{\item[int esl\_opt\_VerifyConfig(ESL\_GETOPTS *g)]}

Given a \ccode{g} that we think is fully configured now --
from config file(s), environment, and command line --
verify that the configuration is self-consistent:
for every option that is set, make sure that any
required options are also set, and that no
incompatible options are set. ``Set'' means
the configured value is non-default and non-NULL (including booleans),
and ``not set'' means the value is default or NULL. (That is,
we don't go solely by \ccode{setby}, which refers to who
determined the state of an option, even if
it is turned off.)

Returns \ccode{eslOK} on success.
\ccode{eslESYNTAX} if a required option is not set, or
if an incompatible option is set; in this case, sets 
\ccode{g->errbuf} to contain a useful error message for
the user.

Throws \ccode{eslEINVAL} if something's wrong with the \ccode{ESL\_OPTIONS}
structure itself -- a coding error in the application.           


\hypertarget{func:esl_opt_ArgNumber()}
{\item[int esl\_opt\_ArgNumber(const ESL\_GETOPTS *g)]}

Returns the number of command line arguments.

Caller must have already called
\ccode{esl\_opt\_ProcessCmdline()}, in order for all the options
to be parsed first.  Everything left on the command line
is taken to be an argument.


\hypertarget{func:esl_opt_SpoofCmdline()}
{\item[int esl\_opt\_SpoofCmdline(const ESL\_GETOPTS *g, char **ret\_cmdline)]}

Given the current configuration state of the application
\ccode{g}, create a command line that would recreate the same
state by itself (without any environment or config file
settings), and return it in \ccode{*ret\_cmdline}.

Returns \ccode{eslOK} on success. The \ccode{*ret\_cmdline} is allocated here,
and caller is responsible for freeing it.

Throws \ccode{eslEMEM} on allocation error, and \ccode{*ret\_cmdline} is \ccode{NULL}.



\hypertarget{func:esl_opt_IsDefault()}
{\item[int esl\_opt\_IsDefault(const ESL\_GETOPTS *g, char *optname)]}

Returns \ccode{TRUE} if option \ccode{optname} is in its
default state; returns \ccode{FALSE} if \ccode{optname} was 
changed on the command line, in the environment, or in 
a configuration file.


\hypertarget{func:esl_opt_IsOn()}
{\item[int esl\_opt\_IsOn(const ESL\_GETOPTS *g, char *optname)]}

Returns \ccode{TRUE} if option is on (set to a non-\ccode{NULL}
value). 

This is most useful when using integer-, real-, char-,
or string-valued options also as boolean switches, where
they can either be OFF, or they can be turned ON by
having a value. Caller can test \ccode{esl\_opt\_IsOn()} to see
if the option's active at all, then use
\ccode{esl\_opt\_GetString()} or whatever to extract the option
value.

For a boolean option, the result is identical to
\ccode{esl\_opt\_GetBoolean()}.



\hypertarget{func:esl_opt_IsUsed()}
{\item[int esl\_opt\_IsUsed(const ESL\_GETOPTS *g, char *optname)]}

Returns \ccode{TRUE} if option \ccode{optname} is in use: it has been
set to a non-default value, and that value correspond to
the option being "on" (a non-\ccode{NULL} value).

This is used in printing application headers, where
we want to report all the options that are in effect that
weren't already on by default.



\hypertarget{func:esl_opt_GetSetter()}
{\item[int esl\_opt\_GetSetter(const ESL\_GETOPTS *g, char *optname)]}

For a processed options object \ccode{g}, return the code
for who set option \ccode{optname}. This code is \ccode{eslARG\_SETBY\_DEFAULT},
\ccode{eslARG\_SETBY\_CMDLINE}, \ccode{eslARG\_SETBY\_ENV}, or it
is $\geq$ \ccode{eslARG\_SETBY\_CFGFILE}. If the option 
was configured by a config file, the file number (the order
of \ccode{esl\_opt\_ProcessConfigFile()} calls) is encoded in codes
$\geq <eslARG_SETBY_CFGFILE>$ as
file number $=$ \ccode{code} - \ccode{eslARG\_SETBY\_CFGFILE} + 1.


\hypertarget{func:esl_opt_GetBoolean()}
{\item[int esl\_opt\_GetBoolean(const ESL\_GETOPTS *g, char *optname)]}

Retrieves the configured TRUE/FALSE value for option \ccode{optname}
from \ccode{g}.


\hypertarget{func:esl_opt_GetInteger()}
{\item[int esl\_opt\_GetInteger(const ESL\_GETOPTS *g, char *optname)]}

Retrieves the configured value for option \ccode{optname}
from \ccode{g}.


\hypertarget{func:esl_opt_GetReal()}
{\item[double esl\_opt\_GetReal(const ESL\_GETOPTS *g, char *optname)]}

Retrieves the configured value for option \ccode{optname}
from \ccode{g}.


\hypertarget{func:esl_opt_GetChar()}
{\item[char esl\_opt\_GetChar(const ESL\_GETOPTS *g, char *optname)]}

Retrieves the configured value for option \ccode{optname}
from \ccode{g}.


\hypertarget{func:esl_opt_GetString()}
{\item[char * esl\_opt\_GetString(const ESL\_GETOPTS *g, char *optname)]}

Retrieves the configured value for option \ccode{optname}
from \ccode{g}.

This retrieves options of type \ccode{eslARG\_STRING},
obviously, but also options of type \ccode{eslARG\_INFILE}
and \ccode{eslARG\_OUTFILE}.


\hypertarget{func:esl_opt_GetArg()}
{\item[char * esl\_opt\_GetArg(const ESL\_GETOPTS *g, int which)]}

Returns a pointer to command line argument number
\ccode{which}, where \ccode{which} ranges from \ccode{1..n} for \ccode{n}
total arguments.

If the caller has already verified that \ccode{n} arguments
exist by testing \ccode{esl\_opt\_ArgNumber(g) == n},
\ccode{esl\_opt\_GetArg()} is guaranteed to return non-\ccode{NULL}
arguments for \ccode{which = 1..n}.

Caller is responsible for verifying that the argument
makes sense for what it's supposed to be.

Returns A pointer to command line argument \ccode{which} on success, or 
\ccode{NULL} if there is no such argument.


\hypertarget{func:esl_opt_DisplayHelp()}
{\item[int esl\_opt\_DisplayHelp(FILE *ofp, const ESL\_GETOPTS *go, int docgroup, int indent,
		    int textwidth)]}

For each option in \ccode{go}, print one line of brief
documentation for it, consisting of the option name
(and argument, if any) and the help string. If space
allows, default values for the options (if any) are
shown in brackets. If space still allows, range restrictions 
for the options (if any) are shown in parentheses.

If \ccode{docgroup} is non-zero, help lines are only printed
for options with the matching \ccode{go->opt[i].docgrouptag}.
This allows the caller to group option documentation
into multiple sections with different headers. To
print all options in one call, pass 0 for \ccode{docgroup}.

\ccode{indent} specifies how many spaces to prefix each line with.

\ccode{textwidth} specifies the maximum text width for the
line.  This would typically be 80 characters. Lines are
not allowed to exceed this length. If a line does exceed
this length, range restriction display is silently
dropped (for all options). If any line still exceeds
\ccode{textwidth}, the default value display is silently dropped,
for all options. If any line still exceeds \ccode{textwidth}, even 
though it now consists almost solely of the option name and 
its help string, an \ccode{eslEINVAL} error is thrown. The
caller should either shorten the help string(s) or 
increase the \ccode{textwidth}.

Returns \ccode{eslOK} on success.

Throws \ccode{eslEINVAL} if one or more help lines are too long for
the specified \ccode{textwidth}.
\ccode{eslEWRITE} if a write fails.


\end{sreapi}

