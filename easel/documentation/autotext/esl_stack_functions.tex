\begin{sreapi}
\hypertarget{func:esl_stack_ICreate()}
{\item[ESL\_STACK * esl\_stack\_ICreate(void)]}

Creates an integer stack.

Returns a pointer to the new stack.

Throws \ccode{NULL} on an allocation failure.


\hypertarget{func:esl_stack_CCreate()}
{\item[ESL\_STACK * esl\_stack\_CCreate(void)]}

Creates a character stack.

Returns a pointer to the new stack.

Throws \ccode{NULL} on an allocation failure.


\hypertarget{func:esl_stack_PCreate()}
{\item[ESL\_STACK * esl\_stack\_PCreate(void)]}

Creates a pointer stack.

Returns a pointer to the new stack.

Throws \ccode{NULL} on an allocation failure.


\hypertarget{func:esl_stack_Reuse()}
{\item[int esl\_stack\_Reuse(ESL\_STACK *s)]}

Empties stack \ccode{s} so it can be reused without
creating a new one. The stack \ccode{s}
can be of any data type; it retains its original
type.

Returns \ccode{eslOK}


\hypertarget{func:esl_stack_Destroy()}
{\item[void esl\_stack\_Destroy(ESL\_STACK *s)]}

Destroys a created stack \ccode{s}, of any data type.


\hypertarget{func:esl_stack_IPush()}
{\item[int esl\_stack\_IPush(ESL\_STACK *ns, int x)]}

Push an integer \ccode{x} onto an integer stack \ccode{ns}.

Returns \ccode{eslOK} on success.

Throws \ccode{eslEMEM} on reallocation failure.
\ccode{eslESYS} if a pthread call fails. In this case, the
state of a pthread mutex and/or cond may be wedged.             


\hypertarget{func:esl_stack_CPush()}
{\item[int esl\_stack\_CPush(ESL\_STACK *cs, char c)]}

Push a character \ccode{c} onto a character stack \ccode{cs}.

Returns \ccode{eslOK} on success.

Throws \ccode{eslEMEM} on reallocation failure.
\ccode{eslESYS} if a pthread call fails. In this case, the
state of a pthread mutex and/or cond may be wedged.             


\hypertarget{func:esl_stack_PPush()}
{\item[int esl\_stack\_PPush(ESL\_STACK *ps, void *p)]}

Push a pointer \ccode{p} onto a pointer stack \ccode{ps}.

Returns \ccode{eslOK} on success.

Throws \ccode{eslEMEM} on reallocation failure.
\ccode{eslESYS} if a pthread call fails. In this case, the
state of a pthread mutex and/or cond may be wedged.             


\hypertarget{func:esl_stack_IPop()}
{\item[int esl\_stack\_IPop(ESL\_STACK *ns, int *ret\_x)]}

Pops an integer off the integer stack \ccode{ns}, and returns
it through \ccode{ret\_x}.

Returns \ccode{eslOK} on success.
\ccode{eslEOD} if stack is empty.

Throws \ccode{eslESYS} if a pthread mutex lock/unlock or conditional wait fails.


\hypertarget{func:esl_stack_CPop()}
{\item[int esl\_stack\_CPop(ESL\_STACK *cs, char *ret\_c)]}

Pops a character off the character stack \ccode{cs}, and returns
it through \ccode{ret\_c}.

Returns \ccode{eslOK} on success. 
\ccode{eslEOD} if stack is empty.

Throws \ccode{eslESYS} if a pthread mutex lock/unlock or conditional wait fails.


\hypertarget{func:esl_stack_PPop()}
{\item[int esl\_stack\_PPop(ESL\_STACK *ps, void **ret\_p)]}

Pops a pointer off the pointer stack \ccode{ps}, and returns
it through \ccode{ret\_p}.

Returns \ccode{eslOK} on success. 
\ccode{eslEOD} if stack is empty.

Throws \ccode{eslESYS} if a pthread mutex lock/unlock or conditional wait fails.


\hypertarget{func:esl_stack_ObjectCount()}
{\item[int esl\_stack\_ObjectCount(ESL\_STACK *s)]}

Returns the number of data objects stored in the
stack \ccode{s}. The stack may be of any datatype.


\hypertarget{func:esl_stack_Convert2String()}
{\item[char * esl\_stack\_Convert2String(ESL\_STACK *cs)]}

Converts a character stack \ccode{cs} to a NUL-terminated
string, and returns a pointer to the string. The
characters in the string are in the same order they
were pushed onto the stack.  The stack is destroyed by
this operation, as if \ccode{esl\_stack\_Destroy()} had been
called on it. The caller becomes responsible for
free'ing the returned string.

Because the stack is destroyed by this call, use care in
a multithreaded context. You don't want to have another
thread waiting to do something to this stack as another
thread is destroying it. Treat this call like
you'd treat \ccode{esl\_stack\_Destroy()}. Its internals are
not mutex-protected (unlike push/pop functions).

Returns Pointer to the string; caller must \ccode{free()} this.

Throws NULL if a reallocation fails.


\hypertarget{func:esl_stack_DiscardTopN()}
{\item[int esl\_stack\_DiscardTopN(ESL\_STACK *s, int n)]}

Throw away the top \ccode{n} elements on stack \ccode{s}.
Equivalent to \ccode{n} calls to a \ccode{Pop()} function.
If \ccode{n} equals or exceeds the number of elements 
currently in the stack, the stack is emptied
as if \ccode{esl\_stack\_Reuse()} had been called.

Returns \ccode{eslOK} on success.

Throws \ccode{eslESYS} if mutex lock/unlock fails, if pthreaded.


\hypertarget{func:esl_stack_DiscardSelected()}
{\item[int esl\_stack\_DiscardSelected(ESL\_STACK *s, int (*discard\_func)(void *, void *), void *param)]}

For each element in the stack, call \verb+(*discard_func)(&element, param)+.
If \ccode{TRUE}, discard the element. 

Passing a pointer to an arbitrary \ccode{(*discard\_func)}
allows arbitrary rules. The \ccode{(*discard\_func)} gets two
arguments: a pointer (which is either a pointer to an
element for int and char stacks, or an actual pointer
element from a pointer stack), and \ccode{param}, a \ccode{void *}
to whatever arguments the caller needs the selection
function to have.

When discarding elements from a pointer stack, the
\ccode{*discard\_func()} will generally assume responsibility
for the memory allocated to those elements. It may want
to free() or Destroy() them, for example, if they're
truly being discarded.

Returns \ccode{eslOK} on success.

Throws \ccode{eslESYS} if a pthread mutex lock/unlock fails.


\hypertarget{func:esl_stack_Shuffle()}
{\item[int esl\_stack\_Shuffle(ESL\_RANDOMNESS *r, ESL\_STACK *s)]}

Randomly shuffle the elements in stack \ccode{s}, using
random numbers from generator \ccode{r}.

Returns \ccode{eslOK} on success, and the stack is randomly 
shuffled.


\hypertarget{func:esl_stack_UseMutex()}
{\item[int esl\_stack\_UseMutex(ESL\_STACK *s)]}

Declare that this stack is going to be operated on by more
than one thread in a multithreaded program, and that all
operations that change its internal state (such as
pushing and popping) need to be protected by a mutex.

Returns \ccode{eslOK} on success.

Throws \ccode{eslEMEM} on allocation failure.
\ccode{eslESYS} if \ccode{pthread\_mutex\_init()} fails.


\hypertarget{func:esl_stack_UseCond()}
{\item[int esl\_stack\_UseCond(ESL\_STACK *s)]}

Declare that this stack is to be used for communication
between threads. If a thread tries to pop from the stack
and the stack is empty, the Pop will do a \ccode{pthread\_cond\_wait()}
to wait until another thread has done a \ccode{Push()}. If a thread
pushes onto the stack, it will do a \ccode{pthread\_cond\_signal()}
to wake up a waiting \ccode{Pop()}'er.

The stack must also have an active mutex. The caller
must call \ccode{esl\_stack\_UseMutex()} before calling
\ccode{esl\_stack\_UseCond().}

Returns \ccode{eslOK} on success.

Throws \ccode{eslEMEM} on allocation failure.
\ccode{eslEINVAL} if this stack lacks an active mutex.
\ccode{eslESYS} if \ccode{pthread\_cond\_init()} fails.


\hypertarget{func:esl_stack_ReleaseCond()}
{\item[int esl\_stack\_ReleaseCond(ESL\_STACK *s)]}

Release the conditional wait state on stack \ccode{s}. In our
idiom for using a stack to coordinate between one or
more client thread adding jobs to a stack, and one or
more worker threads popping them off, we call
\ccode{esl\_stack\_ReleaseCond()} when we know the client(s) are
done. Then the worker(s) seeing an empty job stack may
complete (Pop functions will return eslEOD), rather than
doing a conditional wait waiting for more work to appear
on the stack.

Returns \ccode{eslOK} on success.

Throws \ccode{eslESYS} on pthread call failures.


\end{sreapi}

