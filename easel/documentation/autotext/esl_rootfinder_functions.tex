\begin{sreapi}
\hypertarget{func:esl_rootfinder_Create()}
{\item[ESL\_ROOTFINDER * esl\_rootfinder\_Create(int (*func)(double, void*, double*), void *params)]}

Create a rootfinder to find a root of a function $f(x) = 0$.
\ccode{(*func)()} is a pointer to an implementation of the
function $f(x)$. \ccode{params} is a generic pointer to any
parameters or storage needed in \ccode{(*func)()} other than
the value of $x$. 

Caller implements a \ccode{func()} that takes three arguments.
The first two are the value \ccode{x}, and a void pointer to
any additional parameters that $f(x)$ depends on. The
result, $f(x)$, is returned via the third argument. This
function must return \ccode{eslOK} to indicate success. Upon
error, it may throw any error code it wishes.


Returns pointer to a new \ccode{ESL\_ROOTFINDER} structure.

Throws \ccode{NULL} on allocation failure.


\hypertarget{func:esl_rootfinder_CreateFDF()}
{\item[ESL\_ROOTFINDER * esl\_rootfinder\_CreateFDF(int (*fdf)(double, void*, double*, double*), void *params)]}

Create a rootfinder that will find 
a root of a function $f(x) = 0$ using first derivative
information $f'(x)$. 

Caller provides a pointer \ccode{*fdf()} to a function that
takes four arguments. The first two are the current \ccode{x}
value, and a void pointer to any additional parameters
that $f(x)$ depends on. \ccode{*fdf()} calculates the function
$f(x)$ and the derivative $f'(x)$ and returns them
through the remaining two arguments.

Returns pointer to a new \ccode{ESL\_ROOTFINDER} structure.

Throws \ccode{NULL} on allocation failure.


\hypertarget{func:esl_rootfinder_SetBrackets()}
{\item[int esl\_rootfinder\_SetBrackets(ESL\_ROOTFINDER *R, double xl, double xr)]}

Declare that a root is in the open interval 
\ccode{(xl..xr)}. 

The function will be evaluated at both points.

Returns \ccode{eslOK} on success.

Throws \ccode{eslEINVAL} if \ccode{xl,xr} cannot bracket a root,
because $f(x_l)$ and $f(x_r)$ do not have opposite
signs.

Additionally, if either evaluation fails in the
caller-provided function, the error code from that
failure will be thrown.


\hypertarget{func:esl_root_Bisection()}
{\item[int esl\_root\_Bisection(ESL\_ROOTFINDER *R, double xl, double xr, double *ret\_x)]}

Find a root in the open interval \ccode{xl..xr} by the bisection method,
and return it in \ccode{ret\_x}. 

The bisection method is guaranteed to succeed, provided
that \ccode{xl},\ccode{xr} do indeed bracket a root, though it may
be slow.

The rootfinder \ccode{R} can be created either by
\ccode{esl\_rootfinder\_Create()} or
\ccode{esl\_rootfinder\_CreateFDF()}; if the latter (if the
function in the rootfinder \ccode{R} includes derivative
information), the bisection method will just ignore
the derivative. 

Returns \ccode{eslOK} on success, and \ccode{ret\_x} points to a root.

Throws \ccode{eslEINVAL} if \ccode{xl,xr} do not bracket a root. 
\ccode{eslENOHALT} if the method exceeds the maximum number of
iterations set in \ccode{R}. 

Additionally, any failure code that the caller-provided
function $f(x)$ throws.


\hypertarget{func:esl_root_NewtonRaphson()}
{\item[int esl\_root\_NewtonRaphson(ESL\_ROOTFINDER *R, double guess, double *ret\_x)]}

Find a root by the Newton/Raphson method, starting from
an initial guess \ccode{guess}. Return the root in \ccode{ret\_x}.

The Newton/Raphson method is not guaranteed to succeed,
but when it does, it is much faster than bisection.

Newton/Raphson uses first derivative information, so the
rootfinder \ccode{R} must be created with
\ccode{esl\_rootfinder\_CreateFDF()} for a function that evaluates
both $f(x)$ and $f'(x)$.

Returns \ccode{eslOK} on success, and \ccode{ret\_x} points to a root.

Throws \ccode{eslENOHALT} if the method exceeds the maximum number of
iterations set in \ccode{R}. 

Additionally, any failure code that the caller-provided
function $f(x)$ throws.


\end{sreapi}

