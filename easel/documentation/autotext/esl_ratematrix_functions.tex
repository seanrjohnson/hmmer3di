\begin{sreapi}
\hypertarget{func:esl_rmx_SetWAG()}
{\item[int esl\_rmx\_SetWAG(ESL\_DMATRIX *Q, double *pi)]}

Sets a $20 \times 20$ rate matrix \ccode{Q} to WAG parameters.
The caller allocated \ccode{Q}.

If \ccode{pi} is non-\ccode{NULL}, it provides a vector of 20 amino
acid stationary probabilities in Easel alphabetic order,
A..Y, and the WAG stationary probabilities are set to
these desired $\pi_i$. If \ccode{pi} is \ccode{NULL}, the default
WAG stationary probabilities are used.

The WAG parameters are a maximum likelihood
parameterization obtained by Whelan and Goldman
\citep{WhelanGoldman01}.

Returns \ccode{eslOK} on success.

Throws \ccode{eslEINVAL} if \ccode{Q} isn't a 20x20 general matrix; and
the state of \ccode{Q} is undefined.


\hypertarget{func:esl_rmx_SetJukesCantor()}
{\item[int esl\_rmx\_SetJukesCantor(ESL\_DMATRIX *Q)]}

Sets a 4x4 rate matrix to a Jukes-Cantor model,
scaled to units of 1t = 1.0 substitutions/site.



\hypertarget{func:esl_rmx_SetKimura()}
{\item[int esl\_rmx\_SetKimura(ESL\_DMATRIX *Q, double alpha, double beta)]}

Sets a 4x4 rate matrix to a Kimura 2-parameter
model, given transition and transversion 
relative rates \ccode{alpha} and \ccode{beta}, respectively,
scaled to units of 1t = 1.0 substitutions/site.



\hypertarget{func:esl_rmx_SetF81()}
{\item[int esl\_rmx\_SetF81(ESL\_DMATRIX *Q, double *pi)]}

Sets a 4x4 rate matrix to the F81 model (aka
equal-input model) given stationary base 
compositions \ccode{pi}, 
scaled to units of 1t = 1.0 substitutions/site.


\hypertarget{func:esl_rmx_SetHKY()}
{\item[int esl\_rmx\_SetHKY( ESL\_DMATRIX *Q, double *pi, double alpha, double beta)]}

Given stationary base composition \ccode{pi} for ACGT, and
transition and transversion relative rates \ccode{alpha} and
\ccode{beta} respectively, sets the matrix \ccode{Q} to be the
corresponding HKY (Hasegawa/Kishino/Yano) DNA rate
matrix, scaled in units of 1t= 1.0 substitutions/site
\citep{Hasegawa85}.        

Returns \ccode{eslOK} 



\hypertarget{func:esl_rmx_ValidateP()}
{\item[int esl\_rmx\_ValidateP(ESL\_DMATRIX *P, double tol, char *errbuf)]}

Validates a conditional probability matrix \ccode{P}, whose
elements $P_{ij}$ represent conditional probabilities
$P(j \mid i)$; for example in a first-order Markov
chain, or a continuous-time Markov transition process
where \ccode{P} is for a particular $t$.

Rows must sum to one, and each element $P_{ij}$ is a
probability $0 \leq P_{ij} \leq 1$.

\ccode{tol} specifies the floating-point tolerance to which
the row sums must equal one: \ccode{fabs(sum-1.0) <= tol}.

\ccode{errbuf} is an optional error message buffer. The caller
may pass \ccode{NULL} or a pointer to a buffer of at least
\ccode{eslERRBUFSIZE} characters.

Returns \ccode{eslOK} on successful validation. 
\ccode{eslFAIL} on failure, and if a non-\ccode{NULL} \ccode{errbuf} was
provided by the caller, a message describing
the reason for the failure is put there.

Throws (no abnormal error conditions)


\hypertarget{func:esl_rmx_ValidateQ()}
{\item[int esl\_rmx\_ValidateQ(ESL\_DMATRIX *Q, double tol, char *errbuf)]}

Validates an instantaneous rate matrix \ccode{Q} for a
continuous-time Markov process, whose elements $q_{ij}$
represent instantaneous transition rates $i \rightarrow
j$. 

Rows satisfy the condition that
$q_{ii} = -\sum_{i \neq j} q_{ij}$, and also
that $q_{ij} \geq 0$ for all $j \neq i$. 

\ccode{tol} specifies the floating-point tolerance to which
that condition must hold: \ccode{fabs(sum-q\_ii) <= tol}.

\ccode{errbuf} is an optional error message buffer. The caller
may pass \ccode{NULL} or a pointer to a buffer of at least
\ccode{eslERRBUFSIZE} characters.

Returns \ccode{eslOK} on successful validation. 
\ccode{eslFAIL} on failure, and if a non-\ccode{NULL} \ccode{errbuf} was
provided by the caller, a message describing
the reason for the failure is put there.

Throws (no abnormal error conditions)


\hypertarget{func:esl_rmx_ScaleTo()}
{\item[int esl\_rmx\_ScaleTo(ESL\_DMATRIX *Q, double *pi, double unit)]}

Rescales rate matrix \ccode{Q} so that expected substitution
rate per dt is \ccode{unit}.

Expected substitution rate is:
$\sum_i \sum_j pi_i Q_ij  \forall i \neq j$

\ccode{unit} typically taken to be 1.0, so time units are substitutions/site.
An exception is PAM, where \ccode{unit} = 0.01 for 1 PAM unit.

Returns \ccode{eslOK} on success, and matrix Q is rescaled.



\hypertarget{func:esl_rmx_E2Q()}
{\item[int esl\_rmx\_E2Q(ESL\_DMATRIX *E, double *pi, ESL\_DMATRIX *Q)]}

Given a lower triangular matrix ($j<i$) of 
residue exchangeabilities \ccode{E}, and a stationary residue
frequency vector \ccode{pi}; assuming $E_{ij} = E_{ji}$;
calculates a rate matrix \ccode{Q} as

$Q_{ij} = E_{ij} * \pi_j$

The resulting \ccode{Q} is not normalized to any particular
number of substitutions/site/time unit. See
\ccode{esl\_rmx\_ScaleTo()} for that.

Returns \ccode{eslOK} on success; Q is calculated and filled in.



\hypertarget{func:esl_rmx_RelativeEntropy()}
{\item[double esl\_rmx\_RelativeEntropy(ESL\_DMATRIX *P, double *pi)]}

Given a conditional substitution probability matrix \ccode{P},
with stationary probabilities \ccode{pi}, calculate its
relative entropy $H$:

$H_t = \sum_{ij} P(j \mid i,t) \pi_i \log_2 \frac{P(j \mid i,t)} {\pi_j}$

This assumes that the stationary probabilities are the
same as the background (null model) probabilities.   

Returns the relative entropy, $H$, in bits


\hypertarget{func:esl_rmx_ExpectedScore()}
{\item[double esl\_rmx\_ExpectedScore(ESL\_DMATRIX *P, double *pi)]}

Given a conditional substitution probability matrix \ccode{P}
with stationary probabilities \ccode{pi}, calculate its
expected score:

$ = \sum_{ij} \pi_j \pi_i \log_2 \frac{P(j \mid i,t)} {\pi_j}$

This assumes that the stationary probabilities are the
same as the background (null model) probabilities.   

Returns the expected score, in bits


\end{sreapi}

