\begin{sreapi}
\hypertarget{func:esl_wuss2ct()}
{\item[int esl\_wuss2ct(char *ss, int len, int *ct)]}

Given a secondary structure string \ccode{ss}, \ccode{0..len-1},
in WUSS notation, convert it to a CT array, \ccode{1..len},
in \ccode{ct}. Caller provides a \ccode{ct} allocated for at least 
\ccode{len+1} ints. \ccode{ct[i]} is the position that residue i
base pairs to, or 0 if i is unpaired. \ccode{ct[0]} is undefined
(but if you care: it is set to 0).

WUSS notation is interpreted loosely here, as input
WUSS.  Any matching bracket pair or upper/lower case
alphabetic pair is interpreted as a base pair; any other
WUSS annotation is interpreted as unpaired.

Returns \ccode{eslOK} on success. Returns \ccode{eslESYNTAX} if the WUSS
string isn't valid.

Throws \ccode{eslEMEM} on allocation failure.           


\hypertarget{func:esl_ct2wuss()}
{\item[int esl\_ct2wuss(int *ct, int n, char *ss)]}

Convert a CT array \ccode{ct} for \ccode{n} residues (1..n) to a WUSS
format string \ccode{ss}. \ccode{ss} must be allocated for at least
n+1 chars (+1 for the terminal NUL). 

ER, Sat Aug 18 13:22:03 EDT 2012 
esl\_ct2wuss() extended to deal with pseudoknots structures.
Pseudoknots are annotated as AA...aa, BB...bb,..., ZZ..zz.
Attemting to convert a \ccode{ct} that requires more letters
than [A-Z] will return an \ccode{eslEINVAL} error.

Attempting to convert a \ccode{ct} that involves triplet interactions
will return an \ccode{eslEINVAL} error.

Returns \ccode{eslOK} on success.

Throws \ccode{eslEMEM} on allocation failure.
\ccode{eslEINCONCEIVABLE} on internal failure.


\hypertarget{func:esl_ct2simplewuss()}
{\item[int esl\_ct2simplewuss(int *ct, int n, char *ss)]}

Convert a CT array \ccode{ct} for \ccode{n} residues (1..n) to a simple WUSS
format string \ccode{ss}. \ccode{ss} must be allocated for at least
n+1 chars (+1 for the terminal NUL). 

This function can be used with the \ccode{ct} of a secondary
structure including arbitrary pseudoknots, or for the 
\ccode{ct} or a tertiary structure (say cWH, tWH, cSS,... H bonds). 

The string \ccode{ss} has basepairs annotated as \ccode{}, Aa, Bb, ..., Zz;
unpaired bases are annotated as '.'.

Attemting to convert a \ccode{ct} that requires more letters
than [A-Z] will return an \ccode{eslEINVAL} error.

Attempting to convert a \ccode{ct} that involves triplet interactions
will return an \ccode{eslEINVAL} error.

Returns \ccode{eslOK} on success.

Throws \ccode{eslEMEM} on allocation failure.
\ccode{eslEINCONCEIVABLE} on internal failure.


\hypertarget{func:esl_wuss2kh()}
{\item[int esl\_wuss2kh(char *ss, char *kh)]}

Converts a secondary structure string \ccode{ss} in 
WUSS notation back to old KHS format in \ccode{kh}.
\ccode{kh} must be allocated for at least as much
space as \ccode{ss}. \ccode{kh} may be the same as \ccode{ss},
in which case the conversion is done in-place.

Returns \ccode{eslOK} on success.


\hypertarget{func:esl_kh2wuss()}
{\item[int esl\_kh2wuss(char *kh, char *ss)]}

Converts an old format secondary structure string \ccode{kh}
to shorthand WUSS format \ccode{ss}. \ccode{ss} must be allocated at least
as large as \ccode{kh}. \ccode{ss} can be identical to \ccode{kh}, in which
case the conversion is done in-place.

Returns \ccode{eslOK} on success.


\hypertarget{func:esl_wuss_full()}
{\item[int esl\_wuss\_full(char *oldss, char *newss)]}

Given a simple ("input") WUSS format annotation string \ccode{oldss},
convert it to full ("output") WUSS format in \ccode{newss}.
\ccode{newss} must be allocated by the caller to be at least as 
long as \ccode{oldss}. \ccode{oldss} and \ccode{newss} can be the same,
to convert a secondary structure string in place.

Pseudoknot annotation is preserved, if \ccode{oldss} had it.

Returns \ccode{eslSYNTAX} if \ccode{oldss} isn't in valid WUSS format.

Throws \ccode{eslEMEM} on allocation failure.
\ccode{eslEINCONCEIVABLE} on internal error that can't happen.


\hypertarget{func:esl_wuss_nopseudo()}
{\item[int esl\_wuss\_nopseudo(char *ss1, char *ss2)]}

Given a WUSS format annotation string \ccode{ss1},
removes all pseudoknot annotation to create a new 
WUSS string \ccode{ss2} that contains only a "canonical"
(nonpseudoknotted) structure. \ccode{ss2} must be allocated to
be at least as large as \ccode{ss1}. \ccode{ss1} and \ccode{ss2}
may be the same, in which case the conversion is
done in place. Pseudoknot annotation in \ccode{ss1} is
simply replaced by \ccode{.} in \ccode{ss2}; the resulting
\ccode{ss2} WUSS string is therefore in valid simplified format,
but may not be valid full format WUSS.

Returns \ccode{eslOK}.


\hypertarget{func:esl_wuss_reverse()}
{\item[int esl\_wuss\_reverse(char *ss, char *new)]}

If we need to reverse complement a structure-annotated RNA
sequence, we need to "reverse complement" the WUSS
annotation string. Reverse complement the annotation string
\ccode{ss} into caller-provided space \ccode{new}. To revcomp an annotation 
in place, use \ccode{esl\_wuss\_reverse(ss, ss)}.

Old SELEX files use a different structure annotation
format, with angle brackets pointing the opposite
direction: \ccode{>\ccode{} for a base pair. As a convenient
side effect, <esl\_wuss\_reverse()} will also reverse
complement SELEX annotation lines.

Returns \ccode{eslOK} on success.


\end{sreapi}

