\begin{sreapi}
\hypertarget{func:esl_paml_ReadE()}
{\item[int esl\_paml\_ReadE(FILE *fp, ESL\_DMATRIX *E, double *pi)]}

Read an amino acid rate matrix in PAML format from stream
\ccode{fp}. Return it in two pieces: the symmetric E
exchangeability matrix in \ccode{E}, and the stationary
probability vector $\pi$ in \ccode{pi}.
Caller provides the memory for both \ccode{E} and \ccode{pi}.  \ccode{E}
is a $20 \times 20$ matrix allocated as
\ccode{esl\_dmatrix\_Create(20, 20)}. \ccode{pi} is an array with
space for at least 20 doubles.

The \ccode{E} matrix is symmetric for off-diagonal elements:
$E_{ij} = E_{ij}$ for $i \neq j$.  The on-diagonal
elements $E_{ii}$ are not valid and should not be
accessed.  (They are set to zero.)

The rate matrix will later be obtained from \ccode{E}
and \ccode{pi} as 
$Q_{ij} = E_{ij} \pi_j$ for $i \neq j$ 
and
$Q_{ii} = -\sum_{j \neq i} Q_{ij}$ 
then scaled to units of one
substitution/site; see \ccode{esl\_ratemx\_E2Q()} and
\ccode{esl\_ratemx\_ScaleTo()}.

Data file format: First 190 numbers are a
lower-triangular matrix E of amino acid
exchangeabilities $E_{ij}$. Next 20 numbers are the
amino acid frequencies $\pi_i$. Remainder of the
datafile is ignored.

The alphabet order in the matrix and the frequency
vector is assumed to be "ARNDCQEGHILKMFPSTWYV"
(alphabetical by three-letter code), which appears to be
PAML's default order. This is transformed to Easel's
"ACDEFGHIKLMNPQRSTVWY" (alphabetical by one-letter code)
in the $E_{ij}$ and $\pi_i$ that are returned.

Returns \ccode{eslOK} on success.
Returns \ccode{eslEOF} on premature end of file (parse failed), in which
case the contents of \ccode{E} and \ccode{pi} are undefined.

Throws \ccode{eslEMEM} on internal allocation failure,
and the contents of \ccode{E} and \ccode{pi} are undefined.



\end{sreapi}

