\begin{sreapi}
\hypertarget{func:esl_dst_CPairId()}
{\item[int esl\_dst\_CPairId(const char *asq1, const char *asq2, 
		double *opt\_pid, int *opt\_nid, int *opt\_n)]}

Calculates pairwise fractional identity between two
aligned character strings \ccode{asq1} and \ccode{asq2}. 
Return this distance in \ccode{opt\_pid}; return the
number of identities counted in \ccode{opt\_nid}; and
return the denominator \ccode{MIN(len1,len2)} in
\ccode{opt\_n}.

Alphabetic symbols \ccode{[a-zA-Z]} are compared
case-insensitively for identity. Any nonalphabetic
character is assumed to be a gap symbol.

This simple comparison rule is unaware of synonyms and
degeneracies in biological alphabets.  For a more
sophisticated and biosequence-aware comparison, use
digitized sequences and the \ccode{esl\_dst\_XPairId()} function
instead. Note that currently \ccode{esl\_dst\_XPairId()} does
not correctly handle degeneracies, but is set up to.

Returns \ccode{eslOK} on success. \ccode{opt\_pid}, \ccode{opt\_nid}, \ccode{opt\_n}
contain the answers (for whichever were passed non-NULL). 

Throws \ccode{eslEINVAL} if the strings are different lengths
(not aligned).


\hypertarget{func:esl_dst_CPairMatch()}
{\item[int esl\_dst\_CPairMatch(const char *asq1, const char *asq2, 
		   double *opt\_pmatch, int *opt\_nmatch, int *opt\_n)]}

Calculates pairwise fractional matches between two
aligned character strings \ccode{asq1} and \ccode{asq2}. 
Return this distance in \ccode{opt\_pmatch}; return the
number of matches counted in \ccode{opt\_nmatch}; and
return the denominator \ccode{alen - double\_gaps} in
\ccode{opt\_n}.

Alphabetic symbols \ccode{[a-zA-Z]} are compared
case-insensitively for identity. Any nonalphabetic
character is assumed to be a gap symbol.

This simple comparison rule is unaware of synonyms and
degeneracies in biological alphabets.  For a more
sophisticated and biosequence-aware comparison, use
digitized sequences and the \ccode{esl\_dst\_XPairmatch()} function
instead. Note that currently \ccode{esl\_dst\_XPairMatch()} does
not correctly handle degeneracies, but is set up to.

Returns \ccode{eslOK} on success. \ccode{opt\_pmatch}, \ccode{opt\_nmatch}, \ccode{opt\_n}
contain the answers (for whichever were passed non-NULL). 

Throws \ccode{eslEINVAL} if the strings are different lengths
(not aligned).


\hypertarget{func:esl_dst_CJukesCantor()}
{\item[int esl\_dst\_CJukesCantor(int K, const char *as1, const char *as2, 
		     double *opt\_distance, double *opt\_variance)]}

Calculate the generalized Jukes-Cantor distance between
two aligned character strings \ccode{as1} and \ccode{as2}, in
substitutions/site, for an alphabet of \ccode{K} residues
(\ccode{K=4} for nucleic acid, \ccode{K=20} for proteins). The
maximum likelihood estimate for the distance is
optionally returned in \ccode{opt\_distance}. The large-sample
variance for the distance estimate is
optionally returned in \ccode{opt\_variance}.

Alphabetic symbols \ccode{[a-zA-Z]} are compared
case-insensitively to count the number of identities
(\ccode{n1}) and mismatches (\ccode{n2}>). Any nonalphabetic
character is assumed to be a gap symbol, and aligned
columns containing gap symbols are ignored.  The
fractional difference \ccode{D} used to calculate the
Jukes/Cantor distance is \ccode{n2/n1+n2}.

Returns \ccode{eslOK} on success.

Infinite distances are possible, in which case distance
and variance are both \ccode{HUGE\_VAL}. Caller has to deal
with this case as it sees fit, perhaps by enforcing
an arbitrary maximum distance.

Throws \ccode{eslEINVAL} if the two strings aren't the same length (and
thus can't have been properly aligned).
\ccode{eslEDIVZERO} if no aligned residues were counted.
On either failure, distance and variance are both returned
as \ccode{HUGE\_VAL}.


\hypertarget{func:esl_dst_XPairId()}
{\item[int esl\_dst\_XPairId(const ESL\_ALPHABET *abc, const ESL\_DSQ *ax1, const ESL\_DSQ *ax2, 
		double *opt\_distance, int *opt\_nid, int *opt\_n)]}

Digital version of \ccode{esl\_dst\_CPairId()}: \ccode{adsq1} and
\ccode{adsq2} are digitized aligned sequences, in alphabet
\ccode{abc}. Otherwise, same as \ccode{esl\_dst\_CPairId()} except
that only canonical residues are counted and checked for
identity, while \ccode{esl\_dst\_CPairId()} (which has no
alphabet) counts and checks identity of all alphanumeric
characters.

This function does not use \ccode{esl\_abc\_Match()} to handle
degeneracies but it is set up to do so. Doing that would
require that \ccode{opt\_nid} be changed to a float or double,
or its meaning be changed to be the number of canonical
identities.

Returns \ccode{eslOK} on success. \ccode{opt\_distance}, \ccode{opt\_nid}, \ccode{opt\_n}
contain the answers, for any of these that were passed
non-\ccode{NULL} pointers.

Throws \ccode{eslEINVAL} if the strings are different lengths (not aligned).


\hypertarget{func:esl_dst_XPairMatch()}
{\item[int esl\_dst\_XPairMatch(const ESL\_ALPHABET *abc, const ESL\_DSQ *ax1, const ESL\_DSQ *ax2, 
		   double *opt\_distance, int *opt\_nmatch, int *opt\_n)]}

Digital version of \ccode{esl\_dst\_CPairMatch()}: \ccode{adsq1} and
\ccode{adsq2} are digitized aligned sequences, in alphabet
\ccode{abc}. Otherwise, same as \ccode{esl\_dst\_CPairId()} except
that only canonical residues are counted and checked for
identity, while \ccode{esl\_dst\_CPairId()} (which has no
alphabet) counts and checks identity of all alphanumeric
characters.

Returns \ccode{eslOK} on success. \ccode{opt\_distance}, \ccode{opt\_nmatch}, \ccode{opt\_n}
contain the answers, for any of these that were passed
non-\ccode{NULL} pointers.

Throws \ccode{eslEINVAL} if the strings are different lengths (not aligned).


\hypertarget{func:esl_dst_XJukesCantor()}
{\item[int esl\_dst\_XJukesCantor(const ESL\_ALPHABET *abc, const ESL\_DSQ *ax, const ESL\_DSQ *ay, 
		     double *opt\_distance, double *opt\_variance)]}

Calculate the generalized Jukes-Cantor distance between two
aligned digital strings \ccode{ax} and \ccode{ay}, in substitutions/site, 
using alphabet \ccode{abc} to evaluate identities and differences.
The maximum likelihood estimate for the distance is optionally returned in
\ccode{opt\_distance}. The large-sample variance for the distance
estimate is optionally returned in \ccode{opt\_variance}.

Identical to \ccode{esl\_dst\_CJukesCantor()}, except that it takes
digital sequences instead of character strings.

Returns \ccode{eslOK} on success. As in \ccode{esl\_dst\_CJukesCantor()}, the
distance and variance may be infinite, in which case they
are returned as \ccode{HUGE\_VAL}.

Throws \ccode{eslEINVAL} if the two strings aren't the same length (and
thus can't have been properly aligned).
\ccode{eslEDIVZERO} if no aligned residues were counted.
On either failure, the distance and variance are set
to \ccode{HUGE\_VAL}.


\hypertarget{func:esl_dst_CPairIdMx()}
{\item[int esl\_dst\_CPairIdMx(char **as, int N, ESL\_DMATRIX **ret\_S)]}

Given a multiple sequence alignment \ccode{as}, consisting
of \ccode{N} aligned character strings; calculate
a symmetric fractional pairwise identity matrix by $N(N-1)/2$
calls to \ccode{esl\_dst\_CPairId()}, and return it in 
\ccode{ret\_D}.

Returns \ccode{eslOK} on success, and \ccode{ret\_S} contains the fractional
identity matrix. Caller free's \ccode{S} with
\ccode{esl\_dmatrix\_Destroy()}.

Throws \ccode{eslEINVAL} if a seq has a different
length than others. On failure, \ccode{ret\_D} is returned \ccode{NULL}
and state of inputs is unchanged.

\ccode{eslEMEM} on allocation failure.


\hypertarget{func:esl_dst_CDiffMx()}
{\item[int esl\_dst\_CDiffMx(char **as, int N, ESL\_DMATRIX **ret\_D)]}

Same as \ccode{esl\_dst\_CPairIdMx()}, but calculates
the fractional difference \ccode{d=1-s} instead of the
fractional identity \ccode{s} for each pair.

Returns \ccode{eslOK} on success, and \ccode{ret\_D} contains the
fractional difference matrix. Caller free's \ccode{D} with 
\ccode{esl\_dmatrix\_Destroy()}.

Throws \ccode{eslEINVAL} if any seq has a different
length than others. On failure, \ccode{ret\_D} is returned \ccode{NULL}
and state of inputs is unchanged.


\hypertarget{func:esl_dst_CJukesCantorMx()}
{\item[int esl\_dst\_CJukesCantorMx(int K, char **aseq, int nseq, 
		       ESL\_DMATRIX **opt\_D, ESL\_DMATRIX **opt\_V)]}

Given a multiple sequence alignment \ccode{aseq}, consisting of
\ccode{nseq} aligned character sequences in an alphabet of
\ccode{K} letters (usually 4 for DNA, 20 for protein);
calculate a symmetric Jukes/Cantor pairwise distance
matrix for all sequence pairs, and optionally return the distance
matrix in \ccode{ret\_D}, and optionally return a symmetric matrix of the
large-sample variances for those ML distance estimates
in \ccode{ret\_V}.

Infinite distances (and variances) are possible; they
are represented as \ccode{HUGE\_VAL} in \ccode{D} and \ccode{V}. Caller must
be prepared to deal with them as appropriate.

Returns \ccode{eslOK} on success. \ccode{D} and \ccode{V} contain the
distance matrix (and variances); caller frees these with
\ccode{esl\_dmatrix\_Destroy()}. 

Throws \ccode{eslEINVAL} if any pair of sequences have differing lengths
(and thus cannot have been properly aligned). 
\ccode{eslEDIVZERO} if some pair of sequences had no aligned
residues. On failure, \ccode{D} and \ccode{V} are both returned \ccode{NULL}
and state of inputs is unchanged.

\ccode{eslEMEM} on allocation failure.


\hypertarget{func:esl_dst_XPairIdMx()}
{\item[int esl\_dst\_XPairIdMx(const ESL\_ALPHABET *abc,  ESL\_DSQ **ax, int N, ESL\_DMATRIX **ret\_S)]}

Given a digitized multiple sequence alignment \ccode{ax}, consisting
of \ccode{N} aligned digital sequences in alphabet \ccode{abc}; calculate
a symmetric pairwise fractional identity matrix by $N(N-1)/2$
calls to \ccode{esl\_dst\_XPairId()}, and return it in \ccode{ret\_S}.

Returns \ccode{eslOK} on success, and \ccode{ret\_S} contains the distance
matrix. Caller is obligated to free \ccode{S} with 
\ccode{esl\_dmatrix\_Destroy()}. 

Throws \ccode{eslEINVAL} if a seq has a different
length than others. On failure, \ccode{ret\_S} is returned \ccode{NULL}
and state of inputs is unchanged.

\ccode{eslEMEM} on allocation failure.


\hypertarget{func:esl_dst_XDiffMx()}
{\item[int esl\_dst\_XDiffMx(const ESL\_ALPHABET *abc, ESL\_DSQ **ax, int N, ESL\_DMATRIX **ret\_D)]}

Same as \ccode{esl\_dst\_XPairIdMx()}, but calculates fractional
difference \ccode{1-s} instead of fractional identity \ccode{s} for
each pair.

Returns \ccode{eslOK} on success, and \ccode{ret\_D} contains the difference
matrix; caller is obligated to free \ccode{D} with 
\ccode{esl\_dmatrix\_Destroy()}. 

Throws \ccode{eslEINVAL} if a seq has a different
length than others. On failure, \ccode{ret\_D} is returned \ccode{NULL}
and state of inputs is unchanged.


\hypertarget{func:esl_dst_XJukesCantorMx()}
{\item[int esl\_dst\_XJukesCantorMx(const ESL\_ALPHABET *abc, ESL\_DSQ **ax, int nseq, 
		       ESL\_DMATRIX **opt\_D, ESL\_DMATRIX **opt\_V)]}

Given a digitized multiple sequence alignment \ccode{ax},
consisting of \ccode{nseq} aligned digital sequences in
bioalphabet \ccode{abc}, calculate a symmetric Jukes/Cantor
pairwise distance matrix for all sequence pairs;
optionally return the distance matrix in \ccode{ret\_D} and 
a matrix of the large-sample variances for those ML distance
estimates in \ccode{ret\_V}.

Infinite distances (and variances) are possible. They
are represented as \ccode{HUGE\_VAL} in \ccode{D} and \ccode{V}. Caller must
be prepared to deal with them as appropriate.

Returns \ccode{eslOK} on success. \ccode{D} (and optionally \ccode{V}) contain the
distance matrix (and variances). Caller frees these with
\ccode{esl\_dmatrix\_Destroy()}. 

Throws \ccode{eslEINVAL} if any pair of sequences have differing lengths
(and thus cannot have been properly aligned). 
\ccode{eslEDIVZERO} if some pair of sequences had no aligned
residues. On failure, \ccode{D} and \ccode{V} are both returned \ccode{NULL}
and state of inputs is unchanged.

\ccode{eslEMEM} on allocation failure.


\hypertarget{func:esl_dst_CAverageId()}
{\item[int esl\_dst\_CAverageId(char **as, int N, int max\_comparisons, double *ret\_id)]}

Calculates the average pairwise fractional identity in
a multiple sequence alignment \ccode{as}, consisting of \ccode{N}
aligned character sequences of identical length.

If an exhaustive calculation would require more than
\ccode{max\_comparisons} pairwise comparisons, then instead of
looking at all pairs, calculate the average over a
stochastic sample of \ccode{max\_comparisons} random pairs.
This allows the routine to work efficiently even on very
deep MSAs.

Each fractional pairwise identity (range $[0..$ pid $..1]$
is calculated using \ccode{esl\_dst\_CPairId()}.

Returns \ccode{eslOK} on success, and \ccode{*ret\_id} contains the average
fractional identity.

Throws \ccode{eslEMEM} on allocation failure.
\ccode{eslEINVAL} if any of the aligned sequence pairs aren't 
of the same length.
In either case, \ccode{*ret\_id} is set to 0.


\hypertarget{func:esl_dst_CAverageMatch()}
{\item[int esl\_dst\_CAverageMatch(char **as, int N, int max\_comparisons, double *ret\_match)]}

Calculates the average pairwise fractional matches in
a multiple sequence alignment \ccode{as}, consisting of \ccode{N}
aligned character sequences of identical length.

If an exhaustive calculation would require more than
\ccode{max\_comparisons} pairwise comparisons, then instead of
looking at all pairs, calculate the average over a
stochastic sample of \ccode{max\_comparisons} random pairs.
This allows the routine to work efficiently even on very
deep MSAs.

Each fractional pairwise matches (range $[0..$ pid $..1]$
is calculated using \ccode{esl\_dst\_CPairMatch()}.

Returns \ccode{eslOK} on success, and \ccode{*ret\_match} contains the average
fractional matches.

Throws \ccode{eslEMEM} on allocation failure.
\ccode{eslEINVAL} if any of the aligned sequence pairs aren't 
of the same length.
In either case, \ccode{*ret\_match} is set to 0.


\hypertarget{func:esl_dst_XAverageId()}
{\item[int esl\_dst\_XAverageId(const ESL\_ALPHABET *abc, ESL\_DSQ **ax, int N, int max\_comparisons, double *ret\_id)]}

Calculates the average pairwise fractional identity in
a digital multiple sequence alignment \ccode{ax}, consisting of \ccode{N}
aligned digital sequences of identical length.

If an exhaustive calculation would require more than
\ccode{max\_comparisons} pairwise comparisons, then instead of
looking at all pairs, calculate the average over a
stochastic sample of \ccode{max\_comparisons} random pairs.
This allows the routine to work efficiently even on very
deep MSAs.

Each fractional pairwise identity (range $[0..$ pid $..1]$
is calculated using \ccode{esl\_dst\_XPairId()}.

Returns \ccode{eslOK} on success, and \ccode{*ret\_id} contains the average
fractional identity.

Throws \ccode{eslEMEM} on allocation failure.
\ccode{eslEINVAL} if any of the aligned sequence pairs aren't 
of the same length.
In either case, \ccode{*ret\_id} is set to 0.


\hypertarget{func:esl_dst_XAverageMatch()}
{\item[int esl\_dst\_XAverageMatch(const ESL\_ALPHABET *abc, ESL\_DSQ **ax, int N, int max\_comparisons, double *ret\_match)]}

Calculates the average pairwise fractional matches in
a digital multiple sequence alignment \ccode{ax}, consisting of \ccode{N}
aligned digital sequences of identical length.

If an exhaustive calculation would require more than
\ccode{max\_comparisons} pairwise comparisons, then instead of
looking at all pairs, calculate the average over a
stochastic sample of \ccode{max\_comparisons} random pairs.
This allows the routine to work efficiently even on very
deep MSAs.

Each fractional pairwise matches (range $[0..$ pid $..1]$
is calculated using \ccode{esl\_dst\_XPairMatch()}.

Returns \ccode{eslOK} on success, and \ccode{*ret\_match} contains the average
fractional identity.

Throws \ccode{eslEMEM} on allocation failure.
\ccode{eslEINVAL} if any of the aligned sequence pairs aren't 
of the same length.
In either case, \ccode{*ret\_match} is set to 0.


\end{sreapi}

