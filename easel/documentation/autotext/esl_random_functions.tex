\begin{sreapi}
\hypertarget{func:esl_randomness_Create()}
{\item[ESL\_RANDOMNESS * esl\_randomness\_Create(uint32\_t seed)]}

Create a random number generator using
a given random seed. 

The default random number generator uses the Mersenne
Twister MT19937 algorithm \citep{Matsumoto98}.  It has a
period of $2^{19937}-1$, and equidistribution over
$2^{32}$ values.

If \ccode{seed} is $>0$, the random number generator is
reproducibly initialized with that seed.  Two RNGs
created with the same nonzero seed will give exactly the
same stream of pseudorandom numbers. This allows you to
make reproducible stochastic simulations, for example.

If \ccode{seed} is 0, an arbitrary seed is chosen.
Internally, this comes from hashing together the current
\ccode{time()}, \ccode{clock()}, and process id.  Two RNGs created
with \ccode{seed}=0 probably (but not assuredly) give
different streams of pseudorandom numbers. The true seed
can be retrieved from the \ccode{ESL\_RANDOMNESS} object using
\ccode{esl\_randomness\_GetSeed()}.  The strategy used for
choosing the arbitrary seed is predictable; it should
not be considered cryptographically secure.

Returns an initialized \ccode{ESL\_RANDOMNESS *} on success.
Caller free's with \ccode{esl\_randomness\_Destroy()}.

Throws \ccode{NULL} on failure.



\hypertarget{func:esl_randomness_CreateFast()}
{\item[ESL\_RANDOMNESS * esl\_randomness\_CreateFast(uint32\_t seed)]}

Same as \ccode{esl\_randomness\_Create()}, except that a simple
linear congruential generator (LCG) will be used.
This is a $(a=69069, c=1)$ LCG, with a period of
$2^{32}$. 

This is a low quality RNG. It fails several standard
NIST statistical tests. Successive samples from an LCG
are correlated, and it has a relatively short period. IT
SHOULD NOT BE USED (period). It is present in Easel for
legacy reasons. It used to be substantially faster than
our high quality RNG, but now our default Mersenne
Twister is plenty fast. We have regression and unit
tests that use the LCG and depend on a fixed seed and a
particular pseudorandom number sequence; they would have
to be re-studied carefully to upgrade them to a
different RNG.

Here's an example of how serial correlation arises in an
LCG, and how it can lead to serious (and difficult to
diagnose) failure in a Monte Carlo simulation.  An LCG
calculates $x_{i+1} = ax_i + c$. Suppose $x_i$ is small:
in the range 0..6000, say, as a specific example. Now
$x_{i+1}$ cannot be larger than 4.1e8, for an LCG with
$a=69069$,$c=1$. So if you take a sample and test
whether it is $< 1e-6$ (say), the next sample will be in
a range of about 0..0.1, rather than being uniform on
0..1.

Returns an initialized \ccode{ESL\_RANDOMNESS *} on success.
Caller free's with \ccode{esl\_randomness\_Destroy()}.

Throws \ccode{NULL} on failure.



\hypertarget{func:esl_randomness_CreateTimeseeded()}
{\item[ESL\_RANDOMNESS * esl\_randomness\_CreateTimeseeded(void)]}

Like \ccode{esl\_randomness\_Create()}, but it initializes the
the random number generator using a POSIX \ccode{time()} call 
(number of seconds since the POSIX epoch).

This function is deprecated. Use 
\ccode{esl\_randomness\_Create(0)} instead.

Returns an initialized \ccode{ESL\_RANDOMNESS *} on success.
Caller free's with \ccode{esl\_randomness\_Destroy()}.

Throws \ccode{NULL} on failure.



\hypertarget{func:esl_randomness_Init()}
{\item[int esl\_randomness\_Init(ESL\_RANDOMNESS *r, uint32\_t seed)]}

Reset and reinitialize an existing \ccode{ESL\_RANDOMNESS}
object with a new seed. 

Sometimes it's useful to reseed an RNG to guarantee a
particular reproducible series of pseudorandom numbers
at an arbitrary point in a program; HMMER does this, for
example, to guarantee the same results from the same
HMM/sequence comparison regardless of where in a search
the HMM or sequence occurs.

Returns \ccode{eslOK} on success.



\hypertarget{func:esl_randomness_GetSeed()}
{\item[uint32\_t esl\_randomness\_GetSeed(const ESL\_RANDOMNESS *r)]}

Return the value of the seed. 


\hypertarget{func:esl_randomness_Destroy()}
{\item[void esl\_randomness\_Destroy(ESL\_RANDOMNESS *r)]}

Frees an \ccode{ESL\_RANDOMNESS} object.


\hypertarget{func:esl_random()}
{\item[double esl\_random(ESL\_RANDOMNESS *r)]}

Returns a uniform deviate x, $0.0 \leq x < 1.0$, given
RNG \ccode{r}.

If you cast the return value to float, the [0,1) interval
guarantee is lost because values close to 1 will round to
1.0.

Returns a uniformly distribute random deviate on interval
$0.0 \leq x < 1.0$.


\hypertarget{func:esl_random_uint32()}
{\item[uint32\_t esl\_random\_uint32(ESL\_RANDOMNESS *r)]}

Returns a uniform deviate x, a 32-bit unsigned
integer $0 \leq x < 2^{32}$, given RNG \ccode{r}.


\hypertarget{func:esl_rnd_mix3()}
{\item[uint32\_t esl\_rnd\_mix3(uint32\_t a, uint32\_t b, uint32\_t c)]}

This is Bob Jenkin's \ccode{mix()}. Given \ccode{a,b,c},
generate a number that's generated reasonably
uniformly on $[0,2^{32}-1]$ even for closely
spaced choices of $a,b,c$. 


\hypertarget{func:esl_rnd_UniformPositive()}
{\item[double esl\_rnd\_UniformPositive(ESL\_RANDOMNESS *r)]}

Same as \ccode{esl\_random()}, but assure $0 < x < 1$;
(positive uniform deviate).


\hypertarget{func:esl_rnd_Gaussian()}
{\item[double esl\_rnd\_Gaussian(ESL\_RANDOMNESS *r, double mean, double stddev)]}

Pick a Gaussian-distributed random variable
with a given \ccode{mean} and standard deviation \ccode{stddev}, and
return it. 

Implementation is derived from the public domain
RANLIB.c \ccode{gennor()} function, written by Barry W. Brown
and James Lovato (M.D. Anderson Cancer Center, Texas
USA) using the method described in
\citep{AhrensDieter73}.

Original algorithm said to use uniform deviates on [0,1)
interval (i.e. \ccode{esl\_random()}), but this appears to be
wrong.  Use uniform deviates on (0,1) interval instead
(i.e., \ccode{esl\_rnd\_UniformPositive()}). RANLIB, GNU Octave
have made this alteration, possibly inadvertently.
[xref cryptogenomicon post, 13 Oct 2014].



\hypertarget{func:esl_rnd_Gamma()}
{\item[double esl\_rnd\_Gamma(ESL\_RANDOMNESS *r, double a)]}

Return a random deviate distributed as Gamma(a, 1.)
\citep[pp. 133--134]{Knu-81a}.

The implementation follows not only Knuth \citep{Knu-81a},
but also relied on examination of the implementation in
the GNU Scientific Library (libgsl) \citep{Galassi06}.



\hypertarget{func:esl_rnd_Dirichlet()}
{\item[int esl\_rnd\_Dirichlet(ESL\_RANDOMNESS *rng, const double *alpha, int K, double *p)]}

Using generator \ccode{rng}, sample a Dirichlet-distributed
probabilty vector \ccode{p} of \ccode{K} elements, using Dirichlet
parameters \ccode{alpha} (also of \ccode{K} elements). 

Caller provides the allocated space for \ccode{p}.

\ccode{alpha} is optional. If it isn't provided (i.e. is
\ccode{NULL}), sample \ccode{p} uniformly. (That is, use \ccode{alpha[i] =
1.} for all i=0..K-1.)

This routine is redundant with \ccode{esl\_dirichlet\_DSample()}
and \ccode{esl\_dirichlet\_DSampleUniform()} in the dirichlet
module. Provided here because there's cases where we
just want to sample a probability vector without
introducing a dependency on all the stats/dirichlet code
in Easel.

Returns \ccode{eslOK} on success, and \ccode{p} is a sampled probability vector.


\hypertarget{func:esl_rnd_Deal()}
{\item[int esl\_rnd\_Deal(ESL\_RANDOMNESS *rng, int m, int n, int *deal)]}

Obtain a random sequential sample of \ccode{m} integers without
replacement from \ccode{n} possible ones, like dealing a hand
of m=5 cards from a deck of n=52 possible cards with
cards numbered \ccode{0..n-1}. Caller provides allocated space
\ccode{deal}, allocated for at least \ccode{m} elements. Return the
sample in \ccode{deal}, in sorted order (smallest to largest).

Uses the selection sampling algorithm [Knuth, 3.4.2],
which is O(n) time. For more impressive O(m)
performance, see \ccode{esl\_rand64\_Deal()}.

As a special case, if m = 0 (for whatever reason,
probably an edge-case-exercising unit test), do
nothing. \ccode{deal} is untouched, and can even be \ccode{NULL}.

Returns \ccode{eslOK} on success, and the sample of \ccode{m} integers
is in \ccode{deal}.

Throws (no abnormal error conditions)


\hypertarget{func:esl_rnd_DChoose()}
{\item[int esl\_rnd\_DChoose(ESL\_RANDOMNESS *r, const double *p, int N)]}

Make a random choice from a normalized discrete
distribution \ccode{p} of \ccode{N} elements, where \ccode{p}
is double-precision. Returns the index of the
selected element, $0..N-1$.

\ccode{p} must be a normalized probability distribution
(i.e. must sum to one). Sampling distribution is
undefined otherwise: that is, a choice will always
be returned, but it might be an arbitrary one.

All $p_i$ must be $>$ \ccode{DBL\_EPSILON} in order to 
have a non-zero probability of being sampled.

\ccode{esl\_rnd\_FChoose()} is the same, but for floats in \ccode{p},
and all $p_i$ must be $>$ \ccode{FLT\_EPSILON}.


\hypertarget{func:esl_rnd_DChooseCDF()}
{\item[int esl\_rnd\_DChooseCDF(ESL\_RANDOMNESS *r, const double *cdf, int N)]}

Given a random number generator \ccode{r} and a cumulative
multinomial distribution \ccode{cdf[0..N-1]}, sample an element
\ccode{0..N-1} from that distribution. Return the index \ccode{0..N-1}.

Caller should be sure that \ccode{cdf[0..N-1]} is indeed a
cumulative multinomial distribution -- in particular, that
\ccode{cdf[N-1]} is tolerably close to 1.0 (within roundoff error).

When sampling many times from the same multinomial
distribution \ccode{p}, it will generally be faster to
calculate the CDF once using \ccode{esl\_vec\_DCDF(p, N, cdf)},
then sampling many times from the CDF with
\ccode{esl\_rnd\_DChooseCDF(r, cdf, N)}, as opposed to calling
\ccode{esl\_rnd\_DChoose(r, p, N)} many times, because
\ccode{esl\_rnd\_DChoose()} has to calculated the CDF before
sampling. This also gives you a bit more control over
error detection: you can make sure that the CDF is ok (p
does sum to ~1.0) before doing a lot of sampling from
it.

\ccode{esl\_rnd\_FChooseCDF()} is the same, but for
a single-precision float \ccode{cdf}.

Returns index 0..N-1 of the randomly sampled choice from \ccode{cdf}.



\hypertarget{func:esl_rnd_mem()}
{\item[int esl\_rnd\_mem(ESL\_RANDOMNESS *rng, void *buf, int n)]}

Write \ccode{n} bytes of random garbage into buffer
\ccode{buf}, by uniform sampling of values 0..255,
using generator \ccode{rng}.

Used in unit tests that are reusing memory, and that
want to make sure that there's no favorable side effects
from that reuse.


\hypertarget{func:esl_rnd_floatstring()}
{\item[int esl\_rnd\_floatstring(ESL\_RANDOMNESS *rng, char *s)]}

Generate a string decimal representation of a random 
floating point number in \ccode{s}, which is space provided
by the caller, allocated for at least 20 characters.

The string representation of the number consists of:
* an optional - sign  (50%)
* a 1-6 digit integer part; (1/7 '0'; else 1/7 for each len, [1-9][0-9]* uniformly)
* an optional fractional part (50%), consisting of:
- '.'
- a 1-7 digit fractional part; 1/7 for each len, [0-9]* uniformly
* an optional exponent (50%), consisting of:
- 'e'
-  -20..e20, 1/41 uniformly.
* '\0' terminator.

Maximum length is 1+6+1+7+4+1 = 20.

The string representation is an intersection of what's
valid for \ccode{strtod()}, JSON number values, and C. JSON
number values do not permit numbers like "1."  (a
decimal part, decimal point, no fractional part), nor
hexadecimal representations, nor leading or trailing
whitespace.

The generated number does not attempt to exercise
extreme values. The absolute magnitude of non-zero
numbers ranges from 0.0000001e-20..999999e20 (1e-27 to
~1e27), well within IEEE754 FLT_MIN..FLT_MAX
range. Special values "infinity" and "nan" are not
generated.  Zero can be represented by many different
patterns \ccode{-?0.0+(e-?..)?}.

Returns \ccode{eslOK} on success.


\end{sreapi}

