\begin{sreapi}
\hypertarget{func:esl_histogram_Create()}
{\item[ESL\_HISTOGRAM * esl\_histogram\_Create(double bmin, double bmax, double w)]}

Creates and returns a new histogram object, initially
allocated to count values $>$ \ccode{bmin} and $<=$ \ccode{bmax} into
bins of width \ccode{w}. Thus, a total of \ccode{bmax}-\ccode{bmin}/\ccode{w} bins
are initially created. 

The lower bound \ccode{bmin} and the width \ccode{w} permanently
determine the offset and width of the binning, but not
the range.  For example, \ccode{esl\_histogram\_Create(-100,
100, 0.5)} would initialize the object to collect scores into
400 bins $[-100< x \leq -99.5],[-99.5 < x \leq
-99.0]...[99.5 <x \leq 100.0]$.  Aside from this, the
range specified by the bounds \ccode{bmin} and \ccode{bmax} only
needs to be an initial guess. The histogram object will
reallocate itself dynamically as needed to accommodate
scores that exceed current bounds.

You can be sloppy about \ccode{bmax}; it does not have to
exactly match a bin upper bound. The initial allocation
is for all full-width bins with upper bounds $\leq
bmax$.

\ccode{esl\_histogram\_Create()} creates a simplified histogram
object that collates only the "display" histogram. For
a more complex object that also keeps the raw data samples,
better suited for fitting distributions and goodness-of-fit
testing, use \ccode{esl\_histogram\_CreateFull()}.

There is currently no way to alter where the equals sign
is, in setting the bin bounds: that is, you can't make bins
that have \ccode{bmin} $\leq x$ and $x <$ \ccode{bmax}, alas.


Returns ptr to new \ccode{ESL\_HISTOGRAM} object, which caller is responsible
for free'ing with \ccode{esl\_histogram\_Destroy()}.

Throws \ccode{NULL} on allocation failure.


\hypertarget{func:esl_histogram_CreateFull()}
{\item[ESL\_HISTOGRAM * esl\_histogram\_CreateFull(double bmin, double bmax, double w)]}

Alternative form of \ccode{esl\_histogram\_Create()} that 
creates a more complex histogram that will contain not just the 
display histogram, but also keeps track of all
the raw sample values. Having a complete vector of raw
samples improves distribution-fitting and goodness-of-fit 
tests, but will consume more memory. 


\hypertarget{func:esl_histogram_Destroy()}
{\item[void esl\_histogram\_Destroy(ESL\_HISTOGRAM *h)]}

Frees an \ccode{ESL\_HISTOGRAM} object \ccode{h}.


\hypertarget{func:esl_histogram_Score2Bin()}
{\item[int esl\_histogram\_Score2Bin(ESL\_HISTOGRAM *h, double x, int *ret\_b)]}

For a real-valued \ccode{x}, figure out what bin it would
go into in the histogram \ccode{h}; return this value in
\ccode{*ret\_b}.

Returns \ccode{eslOK} on success.

Throws \ccode{eslERANGE} if bin \ccode{b} would exceed the range of
an integer; for instance, if \ccode{x} isn't finite.



\hypertarget{func:esl_histogram_Add()}
{\item[int esl\_histogram\_Add(ESL\_HISTOGRAM *h, double x)]}

Adds score \ccode{x} to a histogram \ccode{h}.

The histogram will be automatically reallocated as
needed if the score is smaller or larger than the
current allocated bounds.  

Returns \ccode{eslOK} on success.

Throws \ccode{eslEMEM} on reallocation failure.

\ccode{eslERANGE} if \ccode{x} is beyond the reasonable range for
the histogram to store -- either because it isn't finite,
or because the histogram would need to allocate a number
of bins that exceeds \ccode{INT\_MAX}.

Throws \ccode{eslEINVAL} for cases where something has been done
to the histogram that requires it to be 'finished', and
adding more data is prohibited; for example, 
if tail or censoring information has already been set.
On either failure, initial state of \ccode{h} is preserved.


\hypertarget{func:esl_histogram_DeclareCensoring()}
{\item[int esl\_histogram\_DeclareCensoring(ESL\_HISTOGRAM *h, int z, double phi)]}

Declare that the dataset collected in \ccode{h} is known to be a
censored distribution, where \ccode{z} samples were unobserved because
they had values $\leq$ some threshold \ccode{phi} ($\phi$).

No more data can be added to the histogram with \ccode{\_Add()}
after censoring information has been set.

This function is for "true" censored datasets, where
the histogram truly contains no observed points
$x \leq \phi$, and the number that were censored is known
to be \ccode{z}. 

Returns \ccode{eslOK} on success.

Throws \ccode{eslEINVAL} if you try to set \ccode{phi} to a value that is
greater than the minimum \ccode{x} stored in the histogram.


\hypertarget{func:esl_histogram_DeclareRounding()}
{\item[int esl\_histogram\_DeclareRounding(ESL\_HISTOGRAM *h)]}

Declare that the data sample values in the histogram \ccode{h}
are rounded off. Ideally, your bins in \ccode{h} should exactly 
match the rounding procedure. This raises a flag that
binned parameter fitting routines will use when they set
an origin, using the lower bound of the bin instead of
the lowest raw value in the bin.


\hypertarget{func:esl_histogram_SetTail()}
{\item[int esl\_histogram\_SetTail(ESL\_HISTOGRAM *h, double phi, double *ret\_newmass)]}

Suggest a threshold \ccode{phi} to split a histogram \ccode{h}
into "unobserved" data (values $\leq \phi$) and "observed" 
data (values $> \phi$). 

The suggested \ccode{phi} is revised downwards to a $\phi$ at the next 
bin lower bound, because operations on binned data in \ccode{h}
need to know unambiguously whether all the data in a given bin
will be counted as observed or unobserved. 

The probability mass that is in the resulting right tail
is optionally returned in \ccode{ret\_newmass}. You need to know
this number if you're fitting a distribution solely to the
tail (an exponential tail, for example).

Any data point $x_i \leq \phi$ is then considered to be
in a censored (unobserved) region for purposes of parameter
fitting, calculating expected binned counts,
and binned goodness-of-fit tests. 

No more data can be added to the histogram after
censoring information has been set.

This function defines a "virtual" left-censoring: the
histogram actually contains complete data, but appropriate
flags are set to demarcate the "observed" data in the right
tail.

Returns \ccode{eslOK} on success.

Throws \ccode{eslERANGE} if \ccode{phi} is an unreasonable value that
can't be converted to an integer bin value.


\hypertarget{func:esl_histogram_SetTailByMass()}
{\item[int esl\_histogram\_SetTailByMass(ESL\_HISTOGRAM *h, double pmass, double *ret\_newmass)]}

Given a histogram \ccode{h} (with or without raw data samples),
find a cutoff score that at least fraction \ccode{pmass} of the samples
exceed. This threshold is stored internally in the histogram
as \ccode{h->phi}. The number of "virtually censored" samples (to the 
left, with scores $\leq \phi$) is stored internally in \ccode{h->z}.

The identified cutoff score must be a lower bound for some bin
(bins can't be partially censored). The censored mass
will thus usually be a bit greater than \ccode{pmass}, as the
routine will find the highest satisfactory \ccode{h->phi}. The
narrower the bin widths, the more accurately the routine
will be able to satisfy the requested \ccode{frac}. The actual
probability mass in the right tail is optionally returned
in \ccode{ret\_newmass}. You need to know this number if you're 
fitting a distribution solely to the tail (an exponential tail,
for example). It is safe for \ccode{ret\_newmass} to point at 
\ccode{pmass}, in which case the suggested \ccode{pmass} will be overwritten
with the actual mass upon return.

This function defines that the binned data will be
fitted either as a tail, or as a (virtually) left-censored dataset.

Returns \ccode{eslOK} on success.


\hypertarget{func:esl_histogram_GetRank()}
{\item[int esl\_histogram\_GetRank(ESL\_HISTOGRAM *h, int rank, double *ret\_x)]}

Retrieve the \ccode{rank}'th highest score from a 
full histogram \ccode{h}. \ccode{rank} is \ccode{1..n}, for
\ccode{n} total samples in the histogram; return it through
\ccode{ret\_x}.

If the raw scores aren't sorted, they are sorted
first (an $N \log N$ operation).

This can be called at any time, even during data
collection, to see the current \ccode{rank}'th highest score.

Returns \ccode{eslOK} on success.

Throws \ccode{eslEINVAL} if the histogram is display-only,
or if \ccode{rank} isn't in the range 1..n.


\hypertarget{func:esl_histogram_GetData()}
{\item[int esl\_histogram\_GetData(ESL\_HISTOGRAM *h, double **ret\_x, int *ret\_n)]}

Retrieve the raw data values from the histogram \ccode{h}.
Return them in the vector \ccode{ret\_x}, and the number
of values in \ccode{ret\_n}. The values are indexed \ccode{[0..n-1]},
from smallest to largest (\ccode{x[n-1]} is the high score).

\ccode{ret\_x} is a pointer to internal memory in the histogram \ccode{h}.
The histogram \ccode{h} is still responsible for that storage;
its memory will be free'd when you call
\ccode{esl\_histogram\_Destroy()}.

You can only call this after you have finished collecting
all the data. Subsequent calls to \ccode{esl\_histogram\_Add()}
will fail.

Returns \ccode{eslOK} on success.

Throws \ccode{eslEINVAL} if the histogram \ccode{h} is not a full histogram.


\hypertarget{func:esl_histogram_GetTail()}
{\item[int esl\_histogram\_GetTail(ESL\_HISTOGRAM *h, double phi, 
		      double **ret\_x, int *ret\_n, int *ret\_z)]}

Given a full histogram \ccode{h}, retrieve all data values 
above the threshold \ccode{phi} in the right (high scoring) 
tail, as a ptr \ccode{ret\_x} to an array of \ccode{ret\_n} values 
indexed \ccode{[0..n-1]} from lowest to highest score. 
Optionally, it also returns the number of values in 
rest of the histogram in \ccode{ret\_z};
this number is useful if you are going to fit
the tail as a left-censored distribution.

The test is strictly greater than \ccode{phi}, not greater
than or equal to.

\ccode{ret\_x} is a pointer to internal memory in the histogram \ccode{h}.
The histogram \ccode{h} is still responsible for that storage;
its memory will be free'd when you call 
\ccode{esl\_histogram\_Destroy()}.

You can only call this after you have finished collecting
all the data. Subsequent calls to \ccode{esl\_histogram\_Add()}
will fail.             

Returns \ccode{eslOK} on success.

Throws \ccode{eslEINVAL} if the histogram is not a full histogram.


\hypertarget{func:esl_histogram_GetTailByMass()}
{\item[int esl\_histogram\_GetTailByMass(ESL\_HISTOGRAM *h, double pmass,
			    double **ret\_x, int *ret\_n, int *ret\_z)]}

Given a full histogram \ccode{h}, retrieve the data values in
the right (high scoring) tail, as a pointer \ccode{ret\_x}
to an array of \ccode{ret\_n} values indexed \ccode{[0..n-1]} from
lowest to highest score. The tail is defined by a
given mass fraction threshold \ccode{pmass}; the mass in the returned
tail is $\leq$ this threshold. \ccode{pmass} is a probability,
so it must be $\geq 0$ and $\leq 1$.

Optionally, the number of values in the rest of the
histogram can be returned in \ccode{ret\_z}. This is useful
if you are going to fit the tail as a left-censored
distribution.

\ccode{ret\_x} is a pointer to internal memory in \ccode{h}. 
The histogram \ccode{h} remains responsible for its storage,
which will be free'd when you call \ccode{esl\_histogram\_Destroy()}.
As a consequence, you can only call 
\ccode{esl\_histogram\_GetTailByMass()} after you have finished
collecting data. Subsequent calls to \ccode{esl\_histogram\_Add()}
will fail.

Returns \ccode{eslOK} on success.

Throws \ccode{eslEINVAL} if the histogram is not a full histogram, 
or \ccode{pmass} is not a probability.


\hypertarget{func:esl_histogram_SetExpect()}
{\item[int esl\_histogram\_SetExpect(ESL\_HISTOGRAM *h, 
			double (*cdf)(double x, void *params), void *params)]}

Given a histogram \ccode{h} containing some number of empirically
observed binned counts, and a pointer to a function \ccode{(*cdf)()}
that describes the expected cumulative distribution function 
(CDF) for the complete data, conditional on some parameters 
\ccode{params}; calculate the expected counts in each bin of the 
histogram, and hold that information internally in the structure.

The caller provides a function \ccode{(*cdf)()} that calculates
the CDF via a generic interface, taking only two
arguments: a quantile \ccode{x} and a void pointer to whatever
parameters it needs, which it will cast and interpret.
The \ccode{params} void pointer to the given parameters is
just passed along to the generic \ccode{(*cdf)()} function. The
caller will probably implement this \ccode{(*cdf)()} function as
a wrapper around its real CDF function that takes
explicit (non-void-pointer) arguments.

Returns \ccode{eslOK} on success.

Throws \ccode{eslEMEM} on allocation failure; state of \ccode{h} is preserved.


\hypertarget{func:esl_histogram_SetExpectedTail()}
{\item[int esl\_histogram\_SetExpectedTail(ESL\_HISTOGRAM *h, double base\_val, double pmass,
			      double (*cdf)(double x, void *params), 
			      void *params)]}

Given a histogram \ccode{h}, and a pointer to a generic function
\ccode{(*cdf)()} that describes the expected cumulative
distribution function for the right (high-scoring) tail
starting at \ccode{base\_val} (all expected \ccode{x} $>$ \ccode{base\_val}) and
containing a fraction \ccode{pmass} of the complete data
distribution (\ccode{pmass} $\geq 0$ and $\leq 1$);
set the expected binned counts for all complete bins
$\geq$ \ccode{base\_val}. 

If \ccode{base\_val} falls within a bin, that bin is considered
to be incomplete, and the next higher bin is the starting
point. 

Returns \ccode{eslOK} on success.

Throws \ccode{eslEMEM} on memory allocation failure.
\ccode{eslERANGE} if \ccode{base\_val} isn't a reasonable value within
the histogram (it converts to a bin value outside
integer range).


\hypertarget{func:esl_histogram_Write()}
{\item[int esl\_histogram\_Write(FILE *fp, ESL\_HISTOGRAM *h)]}

Print a "prettified" display histogram \ccode{h} to a file
pointer \ccode{fp}.  Deliberately a look-and-feel clone of
Bill Pearson's excellent FASTA output.

Also displays expected binned counts, if they've been
set.

Display will only work well if the bin width (w) is 0.1
or more, because the score labels are only shown to one
decimal point.

Returns \ccode{eslOK} on success.

Throws \ccode{eslEWRITE} on any system write error, such as a
filled disk.


\hypertarget{func:esl_histogram_Plot()}
{\item[int esl\_histogram\_Plot(FILE *fp, ESL\_HISTOGRAM *h)]}

Print observed (and expected, if set) binned counts
in a histogram \ccode{h} to open file pointer \ccode{fp}
in xmgrace XY input file format.

The number that's plotted on the X axis is the minimum
(starting) value of the bin's interval. The Y value is
the total number of counts in the interval (x,x+w] for bin
width w. In xmgrace, you want to set "right stairs" as
the line type in an XY plot.

Returns \ccode{eslOK} on success.

Throws \ccode{eslEWRITE} on any system write error.


\hypertarget{func:esl_histogram_PlotSurvival()}
{\item[int esl\_histogram\_PlotSurvival(FILE *fp, ESL\_HISTOGRAM *h)]}

Given a histogram \ccode{h}, output the observed (and
expected, if available) survival function $P(X>x)$
to file pointer \ccode{fp} in xmgrace XY input file format.

One point is plotted per bin, so the narrower the
bin width, the more smooth and accurate the resulting
plots will be.

As a special case, always plot the highest score with
survival probability 1/N, if it occurred in a bin with
other samples. This is to prevent a survival plot from
looking like it was artificially truncated.

Returns \ccode{eslOK} on success.

Throws \ccode{eslEWRITE} on any system write error.


\hypertarget{func:esl_histogram_PlotQQ()}
{\item[int esl\_histogram\_PlotQQ(FILE *fp, ESL\_HISTOGRAM *h, 
		     double (*invcdf)(double x, void *params), void *params)]}

Given a histogram \ccode{h} containing an empirically observed
distribution, and a pointer to a function \ccode{(*invcdf)()}
for an expected inverse cumulative distribution
function conditional on some parameters \ccode{params};
output a Q-Q plot in xmgrace XY format to file \ccode{fp}.

Same domain limits as goodness-of-fit testing: output
is restricted to overlap between observed data (excluding
any censored data) and expected data (which may be limited
if only a tail was fit).

Returns \ccode{eslOK} on success.

Throws \ccode{eslEWRITE} on any system write error.


\hypertarget{func:esl_histogram_Goodness()}
{\item[int esl\_histogram\_Goodness(ESL\_HISTOGRAM *h, 
		       int nfitted, int *ret\_nbins,
		       double *ret\_G,  double *ret\_Gp,
		       double *ret\_X2, double *ret\_X2p)]}

Given a histogram \ccode{h} with observed and expected counts,
where, for the expected counts, \ccode{nfitted} ($\geq 0$)
parameters were fitted (and thus should be subtracted
from the degrees of freedom);
Perform a G-test and/or a $\chi^2$ test for goodness of 
fit between observed and expected, and optionally return
the number of bins the data were sorted into
(\ccode{ret\_bins}), the G statistic and its probability (\ccode{ret\_G} and
\ccode{ret\_Gp}), and the $\chi^2$ statistic and its probability
(\ccode{ret\_X2} and \ccode{ret\_X2p}). 

If a goodness-of-fit probability is less than some threshold
(usually taken to be 0.01 or 0.05), that is considered to
be evidence that the observed data are unlikely to be consistent
with the tested distribution.

The two tests should give similar
probabilities. However, both tests are sensitive to
arbitrary choices in how the data are binned, and
neither seems to be on an entirely sound theoretical
footing.

On some datasets, pathological and/or very small, it may
be impossible to calculate goodness of fit
statistics. In this case, \ccode{eslENORESULT} is returned.

Returns \ccode{eslOK} on success.

\ccode{eslENORESULT} if the data are such that goodness-of-fit
statistics can't be calculated, probably because there
just aren't many data points. On this error, \ccode{*ret\_G}
and \ccode{*ret\_X2} are 0.0, and \ccode{*ret\_Gp} and \ccode{*ret\_X2p} are
1.0. (Because suppose n=1: then any fit to a single data
point is "perfect".)

Throws \ccode{eslEINVAL} if expected counts have not been set in
the histogram; \ccode{eslERANGE} or \ccode{eslENOHALT} on different internal
errors that can arise in calculating the probabilities;
\ccode{eslEMEM} on internal allocation failure.


\end{sreapi}

