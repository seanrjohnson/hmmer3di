\begin{sreapi}
\hypertarget{func:esl_dsqdata_Open()}
{\item[int esl\_dsqdata\_Open(ESL\_ALPHABET **byp\_abc, char *basename, int nconsumers, ESL\_DSQDATA **ret\_dd)]}

Open dsqdata database \ccode{basename} for reading.  The file
\ccode{basename} is a stub describing the database. The bulk
of the data are in three accompanying binary files: the
index file \ccode{basename}.dsqi, the metadata file
\ccode{basename}.dsqm, and the sequence file \ccode{basename}.dsqs.

\ccode{nconsumers} is an upper bound on the number of threads
in which the caller plans to be calling
\ccode{esl\_dsqdata\_Read()} -- or, more precisely, the maximum
number of data chunks that the caller could be working
on at any given instant.  This is a hint, not a
commitment. The dsqdata loader uses it to determine the
maximum number of data chunks that can be in play at
once (including chunks it is juggling internally, plus
if all the caller's reader threads are busy on
theirs). If \ccode{nconsumers} is set too small, the loader
may stall waiting for chunks to come back for recycling.

Reading digital sequence data requires a digital
alphabet.  You can either provide one (in which case we
validate that it matches the alphabet used by the
dsqdata) or, as a convenience, \ccode{esl\_dsqdata\_Open()} can
create one for you. Either way, you pass a pointer to an
\ccode{ESL\_ALPHABET} structure \ccode{abc}, in \ccode{byp\_abc}.  \ccode{byp\_abc}
uses a partial Easel "bypass" idiom: if \ccode{*byp\_abc} is
NULL, we allocate and return a new alphabet; if
\ccode{*byp\_abc} is a ptr to an existing alphabet, we use it
for validation. That is, you have two choices:

```
ESL_ALPHABET *abc = NULL;
esl_dsqdata_Open(&abc, basename...)
// \ccode{abc} is now the alphabet of \ccode{basename}; 
// now you're responsible for Destroy'ing it
```

or:

```
ESL_ALPHABET *abc = esl_alphabet_Create(eslAMINO);
status = esl_dsqdata_Open(&abc, basename);
// if status == eslEINCOMPAT, alphabet in basename 
// doesn't match caller's expectation
```

Returns \ccode{eslOK} on success.

\ccode{eslENOTFOUND} if one or more of the expected datafiles
aren't there or can't be opened.

\ccode{eslEFORMAT} if something looks wrong in parsing file
formats.  Includes problems in headers, and also the
case where caller provides a digital alphabet in
\ccode{*byp\_abc} and it doesn't match the database's alphabet.

On any normal error, \ccode{*ret\_dd} is still returned, but in
an error state, and \ccode{dd->errbuf} is a user-directed
error message that the caller can relay to the user. Other
than the \ccode{errbuf}, the rest of the contents are undefined.

Caller is responsible for destroying \ccode{*byp\_abc}.

Throws \ccode{eslEMEM} on allocation error.
\ccode{eslESYS} on system call failure.
\ccode{eslEUNIMPLEMENTED} if data are byteswapped
TODO: handle byteswapping

On any thrown exception, \ccode{*ret\_dd} is returned NULL.

On \ccode{eslESYS} exceptions, some thread resources may
not be fully freed, leading to some memory leakage.


\hypertarget{func:esl_dsqdata_Read()}
{\item[int esl\_dsqdata\_Read(ESL\_DSQDATA *dd, ESL\_DSQDATA\_CHUNK **ret\_chu)]}

Read the next chunk from \ccode{dd}, return a pointer to it in
\ccode{*ret\_chu}, and return \ccode{eslOK}. When data are exhausted,
return \ccode{eslEOF}, and \ccode{*ret\_chu} is \ccode{NULL}. 

Threadsafe. All thread operations in the dsqdata reader
are handled internally. Caller does not have to worry
about wrapping this in a mutex. Multiple caller threads
can call \ccode{esl\_dsqdata\_Read()}.

All chunk allocation and deallocation is handled
internally. After using a chunk, caller gives it back to
the reader using \ccode{esl\_dsqdata\_Recycle()}.

Returns \ccode{eslOK} on success. \ccode{*ret\_chu} is a chunk of seq data.
Caller must call \ccode{esl\_dsqdata\_Recycle()} on each chunk
that it Read()'s.

\ccode{eslEOF} if we've reached the end of the input file;
\ccode{*ret\_chu} is NULL.

Throws \ccode{eslESYS} if a pthread call fails. 
Caller should treat this as disastrous. Without correctly
working pthread calls, we cannot read, and we may not be able
to correctly clean up and close the reader. Caller should
treat \ccode{dd} as toxic, clean up whatever else it may need to,
and exit.


\hypertarget{func:esl_dsqdata_Recycle()}
{\item[int esl\_dsqdata\_Recycle(ESL\_DSQDATA *dd, ESL\_DSQDATA\_CHUNK *chu)]}

Recycle chunk \ccode{chu} back to the reader \ccode{dd}.  The reader
is responsible for all allocation and deallocation of
chunks. The reader will either reuse the chunk's memory
if more chunks remain to be read, or it will free it.

Returns \ccode{eslOK} on success. 

Throws \ccode{eslESYS} on a pthread call failure. Caller should regard
such an error as disastrous; if pthread calls are
failing, you cannot depend on the reader to be working
at all, and you should treat \ccode{dd} as toxic. Do whatever
desperate things you need to do and exit.


\hypertarget{func:esl_dsqdata_Close()}
{\item[int esl\_dsqdata\_Close(ESL\_DSQDATA *dd)]}

Close a dsqdata reader.

Returns \ccode{eslOK} on success.

Throws \ccode{eslESYS} on a system call failure, including pthread
calls and fclose(). Caller should regard such a failure
as disastrous: treat \ccode{dd} as toxic and exit as soon as 
possible without making any other system calls, if possible.



\hypertarget{func:esl_dsqdata_Write()}
{\item[int esl\_dsqdata\_Write(ESL\_SQFILE *sqfp, char *basename, char *errbuf)]}

Caller has just opened \ccode{sqfp}, in digital mode.
Create a dsqdata database \ccode{basename} from the sequence
data in \ccode{sqfp}.

\ccode{sqfp} must be protein, DNA, or RNA sequence data.  It
must be rewindable (i.e. a file), because we have to
read it twice. It must be newly opened (i.e. positioned
at the start).

Returns \ccode{eslOK} on success.

\ccode{eslEWRITE} if an output file can't be opened. \ccode{errbuf}
contains user-directed error message.

\ccode{eslEFORMAT} if a parse error is encountered while
reading \ccode{sqfp}.


Throws \ccode{eslESYS}   A system call failed, such as fwrite().
\ccode{eslEINVAL} Sequence handle \ccode{sqfp} isn't digital and rewindable.
\ccode{eslEMEM}   Allocation failure
\ccode{eslEUNIMPLEMENTED} Sequence is too long to be encoded.
(TODO: chromosome-scale DNA sequences)


\end{sreapi}

