\begin{sreapi}
\hypertarget{func:esl_cluster_SingleLinkage()}
{\item[int esl\_cluster\_SingleLinkage(void *base, size\_t n, size\_t size, 
			  int (*linkfunc)(const void *, const void *, const void *, int *), void *param,
			  int *workspace, int *assignments, int *ret\_C)]}

Given a set of vertices, cluster them by single-linkage
clustering.

The data describing each vertex is provided in an array
starting at \ccode{base}, consisting of \ccode{n} vertices. Each
vertex can be of any type (structure, scalar, pointer)
so long as each vertex element is of fixed size \ccode{n}
bytes.

A pointer to the clustering function is provided in
\ccode{(*linkfunc)()}, and a pointer to any necessary
parameters for that function (for example, any
thresholds) is provided in \ccode{param}. 

The \ccode{int (*linkfunc)()} must be written by the
caller. It takes arguments \ccode{(void *v1, void *v2, void
*param, int *ret\_link)}: pointers to two vertices to
test for linkage and a pointer to any necessary
parameters, and it passes the answer \ccode{TRUE} (1) or
\ccode{FALSE} (0) back in \ccode{*ret\_link}. The \ccode{(*linkfunc)()}
returns \ccode{eslOK} (0) on success, and a nonzero error code
on failure (see \ccode{easel.h} for a list of Easel's error
codes).

The caller provides an allocated \ccode{workspace} with space
for at least \ccode{2n} integers. (Allocation in the caller
allows the caller to reuse memory and save
allocation/free cycles, if it has many rounds of
clustering to do.)

The caller also provides allocated space in
\ccode{assignments} for \ccode{n} integers which, upon successful
return, contains assignments of the \ccode{0..n-1} vertices to
\ccode{0..C-1} clusters. That is, if \ccode{assignments[42] = 1},
that means vertex 42 is assigned to cluster 1.  The
total number of clusters is returned in \ccode{ret\_C}.

The algorithm runs in $O(N)$ memory; importantly, it
does not require a $O(N^2)$ adjacency matrix. Worst case
time complexity is $O(N^2)$ (multiplied by any
additional complexity in the \ccode{(*linkfunc()} itself), but
the worst case (no links at all; \ccode{C=n} clusters) should
be unusual. More typically, time scales as about $N \log
N$. Best case is $N$, for a completely connected graph
in which all vertices group into one cluster. (More
precisely, best case complexity arises when vertex 0 is
connected to all other \ccode{n-1} vertices.)

Returns \ccode{eslOK} on success; \ccode{assignments[0..n-1]} contains cluster assigments 
\ccode{0..C-1} for each vertex, and \ccode{*ret\_C} contains the number of clusters
\ccode{C}

Throws status codes from the caller's \ccode{(*linkfunc)} on failure; in this case, 
the contents of \ccode{*assignments} is undefined, and \ccode{*ret\_C} is 0.


\end{sreapi}

