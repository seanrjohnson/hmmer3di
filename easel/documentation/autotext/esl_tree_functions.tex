\begin{sreapi}
\hypertarget{func:esl_tree_Create()}
{\item[ESL\_TREE * esl\_tree\_Create(int ntaxa)]}

Allocate an empty tree structure for \ccode{ntaxa} taxa
and return a pointer to it. \ccode{ntaxa} must be $\geq 2$.

Returns pointer to the new \ccode{ESL\_TREE} object; caller frees 
this with \ccode{esl\_tree\_Destroy()}.

Throws \ccode{NULL} if allocation fails.


\hypertarget{func:esl_tree_CreateGrowable()}
{\item[ESL\_TREE * esl\_tree\_CreateGrowable(int nalloc)]}

Allocate a growable tree structure for an initial
allocation of \ccode{nalloc} taxa, and return a pointer to it.
\ccode{nalloc} must be $\geq 2$.

Returns pointer to a new growable \ccode{ESL\_TREE} object; caller frees 
this with \ccode{esl\_tree\_Destroy()}.

Throws \ccode{NULL} if allocation fails.


\hypertarget{func:esl_tree_CreateFromString()}
{\item[ESL\_TREE * esl\_tree\_CreateFromString(char *s)]}

A convenience for making small test cases in the test
suites: given the contents of a Newick file as a 
single string \ccode{s}, convert it to an \ccode{ESL\_TREE}.

Returns a pointer to the new \ccode{ESL\_TREE} on success.

Throws \ccode{NULL} if it fails to obtain, open, or read the
temporary file that it puts the string \ccode{s} in.


\hypertarget{func:esl_tree_Grow()}
{\item[int esl\_tree\_Grow(ESL\_TREE *T)]}

Given a tree \ccode{T}, make sure it can hold one more taxon;
reallocate internally if necessary by doubling the
number of taxa it is currently allocated to hold.

Returns \ccode{eslOK} on success.

Throws \ccode{eslEMEM} on allocation failure. In this case, 
the data in the tree are unaffected.


\hypertarget{func:esl_tree_SetTaxaParents()}
{\item[int esl\_tree\_SetTaxaParents(ESL\_TREE *T)]}

Constructs the \ccode{T->taxaparent[]} array in the tree
structure \ccode{T}, by an O(N) traversal of the tree.
Upon return, \ccode{T->taxaparent[i]} is the index
of the internal node that taxon \ccode{i} is a child of.

Returns \ccode{eslOK} on success.

Throws \ccode{eslEMEM} on internal allocation error. In this case, the tree is 
returned unchanged.



\hypertarget{func:esl_tree_SetCladesizes()}
{\item[int esl\_tree\_SetCladesizes(ESL\_TREE *T)]}

Constructs the \ccode{T->cladesize[]} array in tree structure
\ccode{T}. Upon successful return, \ccode{T->cladesize[i]} is the
number of taxa contained by the clade rooted at node \ccode{i}
in the tree. For example, \ccode{T->cladesize[0]} is $N$ by
definition, because 0 is the root of the tree.

Returns \ccode{eslOK} on success.

Throws \ccode{eslEMEM} on allocation error; in this case, the
original \ccode{T} is unmodified.


\hypertarget{func:esl_tree_SetTaxonlabels()}
{\item[int esl\_tree\_SetTaxonlabels(ESL\_TREE *T, char **names)]}

Given an array of taxon names \ccode{names[0..N-1]} with the
same order and number as the taxa in tree \ccode{T}, make a
copy of those names in \ccode{T}. For example, \ccode{names} might
be the sequence names in a multiple alignment,
\ccode{msa->sqname}.

If the tree already had taxon names assigned to it, they
are replaced.

As a special case, if the \ccode{names} argument is passed as
\ccode{NULL}, then the taxon labels are set to a string
corresponding to their internal index; that is, taxon 0
is labeled "0". 

Returns \ccode{eslOK} on success, and internal state of \ccode{T}
(specifically, \ccode{T->taxonlabel[]}) now contains a copy
of the taxa names.

Throws \ccode{eslEMEM} on allocation failure. \ccode{T->taxonlabel[]} will be
\ccode{NULL} (even if it was already set).


\hypertarget{func:esl_tree_RenumberNodes()}
{\item[int esl\_tree\_RenumberNodes(ESL\_TREE *T)]}

Given a tree \ccode{T} whose internal nodes might be numbered in
any order, with the sole requirement that node 0 is the
root; renumber the internal nodes (if necessary) to be in Easel's
convention of preorder traversal. No other aspect of \ccode{T} is
altered (including its allocation size).

Returns \ccode{eslOK} on success.

Throws \ccode{eslEMEM} on allocation failure.



\hypertarget{func:esl_tree_VerifyUltrametric()}
{\item[int esl\_tree\_VerifyUltrametric(ESL\_TREE *T)]}

Verify that tree \ccode{T} is ultrametric. 

Returns \ccode{eslOK} if so; \ccode{eslFAIL} if not.

Throws \ccode{eslEMEM} on an allocation failure.


\hypertarget{func:esl_tree_Validate()}
{\item[int esl\_tree\_Validate(ESL\_TREE *T, char *errbuf)]}

Validates the integrity of the data structure in \ccode{T}.
Returns \ccode{eslOK} if the internal data in \ccode{T} are
consistent and valid. Returns \ccode{eslFAIL} if not,
and if a non-\ccode{NULL} message buffer \ccode{errbuf} has been
provided by the caller, an informative message is
left in \ccode{errbuf} describing the reason for the 
failure.



\hypertarget{func:esl_tree_Destroy()}
{\item[void esl\_tree\_Destroy(ESL\_TREE *T)]}

Frees an \ccode{ESL\_TREE} object.


\hypertarget{func:esl_tree_WriteNewick()}
{\item[int esl\_tree\_WriteNewick(FILE *fp, ESL\_TREE *T)]}

Writes tree \ccode{T} to stream \ccode{fp} in Newick format.

Certain options are set in \ccode{T} to control output style, 
as follows.

If \ccode{T->show\_unrooted} is \ccode{TRUE}, \ccode{T} is printed as an
unrooted tree starting with a trifurcation, a la PHYLIP
format (default=\ccode{FALSE}).

If \ccode{T->show\_node\_labels} is \ccode{TRUE}, then labels are
shown for internal nodes, if any are available
(default=\ccode{TRUE}).

If \ccode{T->show\_branchlengths} is \ccode{TRUE}, then branch
lengths are shown, as opposed to just printing a labeled
topology (default=\ccode{TRUE}).

If \ccode{T->show\_root\_branchlength} is also \ccode{TRUE}, then a
0.0 branchlength is shown to the root node, a la Hein's
TreeAlign Newick format (default=\ccode{FALSE}).

If \ccode{T->show\_quoted\_labels} is \ccode{TRUE}, then all labels
are shown in Newick's quoted format, as opposed to only
using quoted labels where necessary (default=\ccode{FALSE}).

Returns \ccode{eslOK} on success.

Throws \ccode{eslEMEM} on allocation error.
\ccode{eslEWRITE} on any system write error.
\ccode{eslEINCONCEIVABLE} on internal error.



\hypertarget{func:esl_tree_ReadNewick()}
{\item[int esl\_tree\_ReadNewick(FILE *fp, char *errbuf, ESL\_TREE **ret\_T)]}

Read a Newick format tree from an open input stream \ccode{fp}.
Return the new tree in \ccode{ret\_T}. 

The new tree \ccode{T} will have the optional \ccode{T->taxonlabel} and
\ccode{T->nodelabel} arrays allocated, containing names of all the
taxa and nodes. Whenever no label appeared in the Newick file
for a node or taxon, the label is set to the empty string.

Caller may optionally provide an \ccode{errbuf} of at least
\ccode{eslERRBUFSIZE} chars, to retrieve diagnostic information
in case of a parsing problem; or \ccode{errbuf} may be passed as
\ccode{NULL}.

Returns Returns \ccode{eslOK} on success, and \ccode{ret\_T} points
to the new tree.

Returns \ccode{eslEFORMAT} on parse errors, such as premature EOF
or bad syntax in the Newick file. In this case, \ccode{ret\_T} is
returned NULL, and the \ccode{errbuf} (if provided> contains an
informative error message.

Throws \ccode{eslEMEM} on memory allocation errors.
\ccode{eslEINCONCEIVABLE} may also arise in case of internal bugs.



\hypertarget{func:esl_tree_Compare()}
{\item[int esl\_tree\_Compare(ESL\_TREE *T1, ESL\_TREE *T2)]}

Given two trees \ccode{T1} and \ccode{T2} for the same
set of \ccode{N} taxa, compare the topologies of the
two trees.

The routine must be able to determine which taxa are
equivalent in \ccode{T1} and \ccode{T2}. If \ccode{T1} and \ccode{T2} both have
taxon labels set, then the routine compares labels.
This is the usual case. (Therefore, the \ccode{N} labels must
all be different, or the routine will be unable to do
this mapping uniquely.) As a special case, if neither
\ccode{T1} nor \ccode{T2} has taxon labels, then the indexing of
taxa \ccode{0..N-1} is assumed to be exactly the same in the
two trees. (And if one tree has labels and the other
does not, an \ccode{eslEINVAL} exception is thrown.)

For comparing unrooted topologies, be sure that \ccode{T1} and
\ccode{T2} both obey the unrooted tree convention that the
"root" is placed on the branch to taxon 0. (That is,
\ccode{T->left[0] = 0}.)

Returns \ccode{eslOK} if tree topologies are identical. \ccode{eslFAIL}
if they aren't.           

Throws \ccode{eslEMEM} on allocation error. \ccode{eslEINVAL} if the taxa in
the trees can't be mapped uniquely and completely to
each other (because one tree doesn't have labels and
one does, or because the labels aren't unique, or the
two trees have different taxa).


\hypertarget{func:esl_tree_UPGMA()}
{\item[int esl\_tree\_UPGMA(ESL\_DMATRIX *D, ESL\_TREE **ret\_T)]}

Given distance matrix \ccode{D}, use the UPGMA algorithm
to construct a tree \ccode{T}.

Returns \ccode{eslOK} on success; the tree is returned in \ccode{ret\_T},
and must be freed by the caller with \ccode{esl\_tree\_Destroy()}.

Throws \ccode{eslEMEM} on allocation problem, and \ccode{ret\_T} is set \ccode{NULL}.


\hypertarget{func:esl_tree_WPGMA()}
{\item[int esl\_tree\_WPGMA(ESL\_DMATRIX *D, ESL\_TREE **ret\_T)]}

Given distance matrix \ccode{D}, use the WPGMA algorithm
to construct a tree \ccode{T}.

Returns \ccode{eslOK} on success; the tree is returned in \ccode{ret\_T},
and must be freed by the caller with \ccode{esl\_tree\_Destroy()}.

Throws \ccode{eslEMEM} on allocation problem, and \ccode{ret\_T} is set \ccode{NULL}.


\hypertarget{func:esl_tree_SingleLinkage()}
{\item[int esl\_tree\_SingleLinkage(ESL\_DMATRIX *D, ESL\_TREE **ret\_T)]}

Given distance matrix \ccode{D}, construct a single-linkage
(minimum distances) clustering tree \ccode{T}.

Returns \ccode{eslOK} on success; the tree is returned in \ccode{ret\_T},
and must be freed by the caller with \ccode{esl\_tree\_Destroy()}.

Throws \ccode{eslEMEM} on allocation problem, and \ccode{ret\_T} is set \ccode{NULL}.


\hypertarget{func:esl_tree_CompleteLinkage()}
{\item[int esl\_tree\_CompleteLinkage(ESL\_DMATRIX *D, ESL\_TREE **ret\_T)]}

Given distance matrix \ccode{D}, construct a complete-linkage
(maximum distances) clustering tree \ccode{T}.

Returns \ccode{eslOK} on success; the tree is returned in \ccode{ret\_T},
and must be freed by the caller with \ccode{esl\_tree\_Destroy()}.

Throws \ccode{eslEMEM} on allocation problem, and \ccode{ret\_T} is set \ccode{NULL}.


\hypertarget{func:esl_tree_Simulate()}
{\item[int esl\_tree\_Simulate(ESL\_RANDOMNESS *r, int N, ESL\_TREE **ret\_T)]}

Generate a random rooted ultrametric tree of \ccode{N} taxa,
using the algorithm of Kuhner and Felsenstein (1994).

The branch lengths are generated by choosing \ccode{N-1}
exponentially distributed split times, with decreasing
expectations of $\frac{1}{2},\frac{1}{3}..\frac{1}{N}$
as the simulation proceeds from the root. Thus the
total expected branch length on the tree is
$\sum_{k=2}^{N} \frac{1}{k}$.

Returns \ccode{eslOK} on success, and the new tree is allocated
here and returned via \ccode{ret\_tree}; caller is 
responsible for free'ing it.

Throws \ccode{eslEMEM} on allocation failure, in which case
the \ccode{ret\_T} is returned \ccode{NULL}.



\hypertarget{func:esl_tree_ToDistanceMatrix()}
{\item[int esl\_tree\_ToDistanceMatrix(ESL\_TREE *T, ESL\_DMATRIX **ret\_D)]}

Given tree \ccode{T}, calculate a pairwise distance matrix
and return it in \ccode{ret\_D}.

Returns \ccode{eslOK} on success, and \ccode{ret\_D} points to the distance 
matrix, which caller is responsible for free'ing with
\ccode{esl\_dmatrix\_Destroy()}.

Throws \ccode{eslEMEM} on allocation failure, in which case
\ccode{ret\_D} is returned \ccode{NULL}.



\end{sreapi}

