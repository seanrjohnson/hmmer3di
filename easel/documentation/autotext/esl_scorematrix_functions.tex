\begin{sreapi}
\hypertarget{func:esl_scorematrix_Create()}
{\item[ESL\_SCOREMATRIX * esl\_scorematrix\_Create(const ESL\_ALPHABET *abc)]}

Allocates a score matrix for alphabet \ccode{abc}, initializes
all scores to zero.

Returns a pointer to the new object.

Throws \ccode{NULL} on allocation failure.


\hypertarget{func:esl_scorematrix_Copy()}
{\item[int esl\_scorematrix\_Copy(const ESL\_SCOREMATRIX *src, ESL\_SCOREMATRIX *dest)]}

Copy \ccode{src} score matrix into \ccode{dest}. Caller
has allocated \ccode{dest} for the same alphabet as
\ccode{src}.

Returns \ccode{eslOK} on success.

Throws \ccode{eslEINCOMPAT} if \ccode{dest} isn't allocated for
the same alphabet as \ccode{src}.
\ccode{eslEMEM} on allocation error.


\hypertarget{func:esl_scorematrix_Clone()}
{\item[ESL\_SCOREMATRIX * esl\_scorematrix\_Clone(const ESL\_SCOREMATRIX *S)]}

Allocates a new matrix and makes it a duplicate
of \ccode{S}. Return a pointer to the new matrix.

Throws \ccode{NULL} on allocation failure.


\hypertarget{func:esl_scorematrix_Compare()}
{\item[int esl\_scorematrix\_Compare(const ESL\_SCOREMATRIX *S1, const ESL\_SCOREMATRIX *S2)]}

Compares two score matrices. Returns \ccode{eslOK} if they 
are identical, \ccode{eslFAIL} if they differ. Every aspect
of the two matrices is compared.

The annotation (name, filename path) are not
compared; we may want to compare an internally
generated scorematrix to one read from a file.


\hypertarget{func:esl_scorematrix_CompareCanon()}
{\item[int esl\_scorematrix\_CompareCanon(const ESL\_SCOREMATRIX *S1, const ESL\_SCOREMATRIX *S2)]}

Compares the scores of canonical residues in 
two score matrices \ccode{S1} and \ccode{S2} for equality.
Returns \ccode{eslOK} if they are identical, \ccode{eslFAIL} 
if they differ. Peripheral aspects of the scoring matrices
having to do with noncanonical residues, output
order, and suchlike are ignored.


\hypertarget{func:esl_scorematrix_Max()}
{\item[int esl\_scorematrix\_Max(const ESL\_SCOREMATRIX *S)]}

Returns the maximum value in score matrix \ccode{S}.


\hypertarget{func:esl_scorematrix_Min()}
{\item[int esl\_scorematrix\_Min(const ESL\_SCOREMATRIX *S)]}

Returns the minimum value in score matrix \ccode{S}.


\hypertarget{func:esl_scorematrix_IsSymmetric()}
{\item[int esl\_scorematrix\_IsSymmetric(const ESL\_SCOREMATRIX *S)]}

Returns \ccode{TRUE} if matrix \ccode{S} is symmetric,
or \ccode{FALSE} if it's not.


\hypertarget{func:esl_scorematrix_ExpectedScore()}
{\item[int esl\_scorematrix\_ExpectedScore(ESL\_SCOREMATRIX *S, double *fi, double *fj, double *ret\_E)]}

Calculates the expected score of a matrix \ccode{S},
given background frequencies \ccode{fi} and \ccode{fj};
return it in \ccode{*ret\_E}.

The expected score is defined as
$\sum_{ab} f_a f_b \sigma_{ab}$.

The expected score is in whatever units the score matrix
\ccode{S} is in. If you know $\lambda$, you can convert it to
units of bits ($\log 2$) by multiplying it by $\lambda /
\log 2$.

Returns \ccode{eslOK} on success.


\hypertarget{func:esl_scorematrix_RelEntropy()}
{\item[int esl\_scorematrix\_RelEntropy(const ESL\_SCOREMATRIX *S, const double *fi, const double *fj, double lambda, double *ret\_D)]}

Calculates the relative entropy of score matrix \ccode{S} in
bits, given its background distributions \ccode{fi} and \ccode{fj} and
its scale \ccode{lambda}.

The relative entropy is defined as $\sum_{ab} p_{ab}
\log_2 \frac{p_{ab}} {f_a f_b}$, the average score (in
bits) of homologous aligned sequences. In general it is
$\geq 0$ (and certainly so in the case when background
frequencies $f_a$ and $f_b$ are the marginals of the
$p_{ab}$ joint ptobabilities).

Returns \ccode{eslOK} on success, and \ccode{ret\_D} contains the relative
entropy.

Throws \ccode{eslEMEM} on allocation error. 
\ccode{eslEINVAL} if the implied $p_{ij}$'s don't sum to one,
probably indicating that \ccode{lambda} was not the correct
\ccode{lambda} for \ccode{S}, \ccode{fi}, and \ccode{fj}.
In either exception, \ccode{ret\_D} is returned as 0.0.


\hypertarget{func:esl_scorematrix_JointToConditionalOnQuery()}
{\item[int esl\_scorematrix\_JointToConditionalOnQuery(const ESL\_ALPHABET *abc, ESL\_DMATRIX *P)]}

Given a joint probability matrix \ccode{P} that has been calculated
by \ccode{esl\_scorematrix\_ProbifyGivenBG()} or \ccode{esl\_scorematrix\_Probify()}
(or one that obeys the same conditions; see below), 
convert the joint probabilities \ccode{P(a,b)} to conditional 
probabilities \ccode{P(b | a)}, where \ccode{b} is a residue in the target,
and \ccode{a} is a residue in the query.

$P(b \mid a) = P(ab) / P(a)$, where $P(a) = \sum_b P(ab)$.

The value stored in \ccode{P->mx[a][b]} is $P(b \mid a)$.

All values in \ccode{P} involving the codes for gap,
nonresidue, and missing data (codes \ccode{K},\ccode{Kp-2}, and
\ccode{Kp-1}) are 0.0, not probabilities. Only rows/columns
\ccode{i=0..K,K+1..Kp-3} are valid probability vectors.

Returns \ccode{eslOK} on success.

Throws (no abnormal error conditions)



\hypertarget{func:esl_scorematrix_Destroy()}
{\item[void esl\_scorematrix\_Destroy(ESL\_SCOREMATRIX *S)]}

Frees a score matrix.


\hypertarget{func:esl_scorematrix_Set()}
{\item[int esl\_scorematrix\_Set(const char *name, ESL\_SCOREMATRIX *S)]}

Set the allocated score matrix \ccode{S} to standard score
matrix \ccode{name}, where \ccode{name} is the name of one of
several matrices built-in to Easel. For example,
\ccode{esl\_scorematrix\_Set("BLOSUM62", S)}.

The alphabet for \ccode{S} (\ccode{S->abc\_r}) must be set already.

Built-in amino acid score matrices in Easel include
BLOSUM45, BLOSUM50, BLOSUM62, BLOSUM80, BLOSUM90, PAM30,
PAM70, PAM120, and PAM240.

Returns \ccode{eslOK} on success, and the scores in \ccode{S} are set.

\ccode{eslENOTFOUND} if \ccode{name} is not available as a built-in matrix
for the alphabet that's set in \ccode{S}.

Throws \ccode{eslEMEM} on allocation error.


\hypertarget{func:esl_scorematrix_SetIdentity()}
{\item[int esl\_scorematrix\_SetIdentity(ESL\_SCOREMATRIX *S)]}

Sets score matrix \ccode{S} to be +1 for a match, 
0 for a mismatch. \ccode{S} may be for any alphabet.

Rarely useful in real use, but may be useful to create
simple examples (including debugging).

Returns \ccode{eslOK} on success, and the scores in \ccode{S} are set.


\hypertarget{func:esl_scorematrix_SetFromProbs()}
{\item[int esl\_scorematrix\_SetFromProbs(ESL\_SCOREMATRIX *S, double lambda, const ESL\_DMATRIX *P, const double *fi, const double *fj)]}

Sets the scores in a new score matrix \ccode{S} from target joint
probabilities in \ccode{P}, query background probabilities \ccode{fi}, and 
target background probabilities \ccode{fj}, with scale factor \ccode{lambda}:
$s_{ij} = \frac{1}{\lambda} \frac{p_{ij}}{f_i f_j}$.

Size of everything must match the canonical alphabet
size in \ccode{S}. That is, \ccode{S->abc->K} is the canonical
alphabet size of \ccode{S}; \ccode{P} must contain $K times K$
probabilities $P_{ij}$, and \ccode{fi},\ccode{fj} must be vectors of
K probabilities. All probabilities must be nonzero.

Returns \ccode{eslOK} on success, and \ccode{S} contains the calculated score matrix.


\hypertarget{func:esl_scorematrix_SetWAG()}
{\item[int esl\_scorematrix\_SetWAG(ESL\_SCOREMATRIX *S, double lambda, double t)]}

Parameterize an amino acid score matrix \ccode{S} using the WAG
rate matrix \citep{WhelanGoldman01} as the underlying
evolutionary model, at a distance of \ccode{t}
substitutions/site, with scale factor \ccode{lambda}.

Returns \ccode{eslOK} on success, and the 20x20 residue scores in \ccode{S} are set.

Throws \ccode{eslEINVAL} if \ccode{S} isn't an allocated amino acid score matrix.
\ccode{eslEMEM} on allocation failure.


\hypertarget{func:esl_scorematrix_Read()}
{\item[int esl\_scorematrix\_Read(ESL\_FILEPARSER *efp, const ESL\_ALPHABET *abc, ESL\_SCOREMATRIX **ret\_S)]}

Given a pointer \ccode{efp} to an open file parser for a file
containing a score matrix (such as a PAM or BLOSUM
matrix), parse the file and create a new score matrix
object. The scores are expected to be for the alphabet
\ccode{abc}. 

The score matrix file is in the format that BLAST or
FASTA use. The first line is a header contains N
single-letter codes for the residues. Each of N
subsequent rows optionally contains a residue row label
(in the same order as the columns), followed by N
residue scores.  (Older matrix files do not contain the
leading row label; newer ones do.) The residues may
appear in any order. They must minimally include the
canonical K residues (K=4 for DNA, K=20 for protein),
and may also contain none, some, or all degeneracy
codes. Any other residue code that is not in the Easel
digital alphabet (including, in particular, the '*' code
for a stop codon) is ignored by the parser.

Returns \ccode{eslOK} on success, and \ccode{ret\_S} points to a newly allocated 
score matrix. 

Returns \ccode{eslEFORMAT} on parsing error; in which case, \ccode{ret\_S} is
returned \ccode{NULL}, and \ccode{efp->errbuf} contains an informative
error message.

Throws \ccode{eslEMEM} on allocation error.


\hypertarget{func:esl_scorematrix_Write()}
{\item[int esl\_scorematrix\_Write(FILE *fp, const ESL\_SCOREMATRIX *S)]}

Writes a score matrix \ccode{S} to an open stream \ccode{fp}, in 
format compatible with BLAST, FASTA, and other common
sequence alignment software.

Returns \ccode{eslOK} on success.

Throws \ccode{eslEWRITE} on any system write error, such as filled disk.


\hypertarget{func:esl_scorematrix_ProbifyGivenBG()}
{\item[int esl\_scorematrix\_ProbifyGivenBG(const ESL\_SCOREMATRIX *S, const double *fi, const double *fj, 
			       double *opt\_lambda, ESL\_DMATRIX **opt\_P)]}

Given a score matrix \ccode{S} and known query and target
background frequencies \ccode{fi} and \ccode{fj} respectively, calculate scale
\ccode{lambda} and implicit target probabilities \citep{Altschul01}. 
Optionally returns either (or both) in \ccode{opt\_lambda} and \ccode{opt\_P}.

The implicit target probabilities are returned in a
newly allocated $Kp \times Kp$ \ccode{ESL\_DMATRIX}, over both
the canonical (typically K=4 or K=20) residues in the
residue alphabet, and the degenerate residue codes.
Values involving degenerate residue codes are marginal
probabilities (i.e. summed over the degeneracy).
Only actual residue degeneracy can have nonzero values
for \ccode{p\_ij}; by convention, all values involving the
special codes for gap, nonresidue, and missing data
(\ccode{K}, \ccode{Kp-2}, \ccode{Kp-1}) are 0.

If the caller wishes to convert this joint probability
matrix to conditionals, it can take advantage of the
fact that the degenerate probability \ccode{P(X,j)} is our
marginalized \ccode{pj}, and \ccode{P(i,X)} is \ccode{pi}. 
i.e., \ccode{P(j|i) = P(i,j) / P(i) = P(i,j) / P(X,j)}.
Those X values are \ccode{P->mx[i][esl\_abc\_GetUnknown(abc)]},
\ccode{P->mx[esl\_abc\_GetUnknown(abc)][j]}; equivalently, just use
code \ccode{Kp-3} for X.

By convention, i is always the query sequence, and j is
always the target. We do not assume symmetry in the
scoring system, though that is usually the case.

Returns \ccode{eslOK} on success, \ccode{*ret\_lambda} contains the
calculated $\lambda$ parameter, and \ccode{*ret\_P} points to
the target probability matrix (which is allocated here,
and must be free'd by caller with \ccode{esl\_dmatrix\_Destroy(*ret\_P)}.

Throws \ccode{eslEMEM} on allocation error; 
\ccode{eslEINVAL} if matrix is invalid and has no solution for $\lambda$;
\ccode{eslENOHALT} if the solver fails to find $\lambda$.
In these cases, \ccode{*ret\_lambda} is 0.0, and \ccode{*ret\_P} is \ccode{NULL}. 


\hypertarget{func:esl_scorematrix_Probify()}
{\item[int esl\_scorematrix\_Probify(const ESL\_SCOREMATRIX *S, ESL\_DMATRIX **opt\_P, double **opt\_fi, double **opt\_fj, double *opt\_lambda)]}

Reverse engineering of a score matrix: given a "valid"
substitution matrix \ccode{S}, obtain implied joint
probabilities $p_{ij}$, query composition $f_i$, target
composition $f_j$, and scale $\lambda$, by assuming that
$f_i$ and $f_j$ are the appropriate marginals of $p_{ij}$.
Optionally return any or all of these solutions in
\ccode{*opt\_P}, \ccode{*opt\_fi}, \ccode{*opt\_fj}, and \ccode{*opt\_lambda}.

The calculation is run only on canonical residue scores
$0..K-1$ in S, to calculate joint probabilities for all
canonical residues. Joint and background probabilities 
involving degenerate residues are then calculated by
appropriate marginalizations. See notes on
\ccode{esl\_scorematrix\_ProbifyGivenBG()} about how probabilities
involving degeneracy codes are calculated.

This implements an algorithm described in
\citep{YuAltschul03} and \citep{YuAltschul05}.

Although this procedure may succeed in many cases,
it is unreliable and should be used with great caution.
Yu and Altschul note that it can find invalid solutions
(negative probabilities), and although they say that one
can keep searching until a valid solution is found, 
one can produce examples where this does not seem to be
the case. The caller MUST check return status, and
MUST expect \ccode{eslENORESULT}.

Returns \ccode{eslOK} on success, and \ccode{opt\_P}, \ccode{opt\_fi}, \ccode{opt\_fj}, and \ccode{opt\_lambda}
point to the results (for any of these that were passed non-\ccode{NULL}).

\ccode{opt\_P}, \ccode{opt\_fi}, and \ccode{opt\_fj}, if requested, are new
allocations, and must be freed by the caller.

Returns \ccode{eslENORESULT} if the algorithm fails to determine a valid solution,
but the solution is still returned (and caller needs to free).

Returns \ccode{eslEINVAL} if input score matrix isn't valid (sensu YuAltschul05);
now \ccode{opt\_P}, \ccode{opt\_fi}, \ccode{opt\_fj} are returned NULL and \ccode{opt\_lambda} is returned
as 0.

Throws \ccode{eslEMEM} on allocation failure.



\end{sreapi}

