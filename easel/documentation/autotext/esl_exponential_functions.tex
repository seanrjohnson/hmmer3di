\begin{sreapi}
\hypertarget{func:esl_exp_pdf()}
{\item[double esl\_exp\_pdf(double x, double mu, double lambda)]}

Calculates the probability density function for the
exponential, $P(X=x)$, given value \ccode{x}, offset \ccode{mu},
and decay parameter \ccode{lambda}.


\hypertarget{func:esl_exp_logpdf()}
{\item[double esl\_exp\_logpdf(double x, double mu, double lambda)]}

Calculates the log probability density function for the
exponential, $P(X=x)$, given value \ccode{x}, offset \ccode{mu},
and decay parameter \ccode{lambda}.


\hypertarget{func:esl_exp_cdf()}
{\item[double esl\_exp\_cdf(double x, double mu, double lambda)]}

Calculates the cumulative distribution function for the
exponential, $P(X \leq x)$, given value \ccode{x}, offset \ccode{mu},
and decay parameter \ccode{lambda}.


\hypertarget{func:esl_exp_logcdf()}
{\item[double esl\_exp\_logcdf(double x, double mu, double lambda)]}

Calculates the log of the cumulative distribution function
for the exponential, $log P(X \leq x)$, given value \ccode{x},
offset \ccode{mu}, and decay parameter \ccode{lambda}.


\hypertarget{func:esl_exp_surv()}
{\item[double esl\_exp\_surv(double x, double mu, double lambda)]}

Calculates the survivor function, $P(X>x)$ (that is, 1-CDF,
the right tail probability mass) for an exponential distribution,
given value \ccode{x}, offset \ccode{mu}, and decay parameter \ccode{lambda}.


\hypertarget{func:esl_exp_logsurv()}
{\item[double esl\_exp\_logsurv(double x, double mu, double lambda)]}

Calculates the log survivor function, $\log P(X>x)$ (that is,
log(1-CDF), the log of the right tail probability mass) for an 
exponential distribution, given value \ccode{x}, offset \ccode{mu}, and 
decay parameter \ccode{lambda}.


\hypertarget{func:esl_exp_invcdf()}
{\item[double esl\_exp\_invcdf(double p, double mu, double lambda)]}

Calculates the inverse of the CDF; given a \ccode{cdf} value
$0 <= p < 1$, returns the value $x$ at which the CDF
has that value.


\hypertarget{func:esl_exp_invsurv()}
{\item[double esl\_exp\_invsurv(double p, double mu, double lambda)]}

Calculates the inverse of the survivor function, the score
at which the right tail's mass is $0 <= p < 1$, for an
exponential function with parameters \ccode{mu} and \ccode{lambda}.


\hypertarget{func:esl_exp_generic_pdf()}
{\item[double esl\_exp\_generic\_pdf(double x, void *params)]}

Generic-API version of PDF.


\hypertarget{func:esl_exp_generic_cdf()}
{\item[double esl\_exp\_generic\_cdf(double x, void *params)]}

Generic-API version of CDF.


\hypertarget{func:esl_exp_generic_surv()}
{\item[double esl\_exp\_generic\_surv(double x, void *params)]}

Generic-API version of survival function.


\hypertarget{func:esl_exp_generic_invcdf()}
{\item[double esl\_exp\_generic\_invcdf(double p, void *params)]}

Generic-API version of inverse CDF.


\hypertarget{func:esl_exp_Plot()}
{\item[int esl\_exp\_Plot(FILE *fp, double mu, double lambda, 
	     double (*func)(double x, double mu, double lambda), 
	     double xmin, double xmax, double xstep)]}

Plot some exponential function \ccode{func} (for instance,
\ccode{esl\_exp\_pdf()}) for parameters \ccode{mu} and \ccode{lambda}, for
a range of values x from \ccode{xmin} to \ccode{xmax} in steps of \ccode{xstep};
output to an open stream \ccode{fp} in xmgrace XY input format.

Returns \ccode{eslOK} on success.

Throws \ccode{eslEWRITE} on any system write error, such as a filled disk.


\hypertarget{func:esl_exp_Sample()}
{\item[double esl\_exp\_Sample(ESL\_RANDOMNESS *r, double mu, double lambda)]}

Sample an exponential random variate
by the transformation method, given offset \ccode{mu}
and decay parameter \ccode{lambda}.


\hypertarget{func:esl_exp_FitComplete()}
{\item[int esl\_exp\_FitComplete(double *x, int n, double *ret\_mu, double *ret\_lambda)]}

Given an array of \ccode{n} samples \ccode{x[0]..x[n-1]}, fit
them to an exponential distribution.
Return maximum likelihood parameters \ccode{ret\_mu} and \ccode{ret\_lambda}.

Returns \ccode{eslOK} on success.

Throws \ccode{eslEINVAL} if n=0 (no data).



\hypertarget{func:esl_exp_FitCompleteScale()}
{\item[int esl\_exp\_FitCompleteScale(double *x, int n, double mu, double *ret\_lambda)]}

Given an array of \ccode{n} samples \ccode{x[0]..x[n-1]}, fit
them to an exponential distribution of known location
parameter \ccode{mu}. Return maximum likelihood scale 
parameter \ccode{ret\_lambda}. 

All $x_i \geq \mu$.

Returns \ccode{eslOK} on success.



\hypertarget{func:esl_exp_FitCompleteBinned()}
{\item[int esl\_exp\_FitCompleteBinned(ESL\_HISTOGRAM *g, double *ret\_mu, double *ret\_lambda)]}

Fit a complete exponential distribution to the observed
binned data in a histogram \ccode{g}, where each
bin i holds some number of observed samples x with values from 
lower bound l to upper bound u (that is, $l < x \leq u$);
find maximum likelihood parameters $\mu,\lambda$ and 
return them in \ccode{*ret\_mu}, \ccode{*ret\_lambda}.

If the binned data in \ccode{g} were set to focus on 
a tail by virtual censoring, the "complete" exponential is 
fitted to this tail. The caller then also needs to
remember what fraction of the probability mass was in this
tail.

The ML estimate for $mu$ is the smallest observed
sample.  For complete data, \ccode{ret\_mu} is generally set to
the smallest observed sample value, except in the
special case of a "rounded" complete dataset, where
\ccode{ret\_mu} is set to the lower bound of the smallest
occupied bin. For tails, \ccode{ret\_mu} is set to the cutoff
threshold \ccode{phi}, where we are guaranteed that \ccode{phi} is
at the lower bound of a bin (by how the histogram
object sets tails). 

The ML estimate for \ccode{ret\_lambda} has an analytical 
solution, so this routine is fast. 

If all the data are in one bin, the ML estimate of
$\lambda$ will be $\infty$. This is mathematically correct,
but is probably a situation the caller wants to avoid, perhaps
by choosing smaller bins.

This function currently cannot fit an exponential tail
to truly censored, binned data, because it assumes that
all bins have equal width, but in true censored data, the
lower cutoff \ccode{phi} may fall anywhere in the first bin.

Returns \ccode{eslOK} on success.

Throws \ccode{eslEINVAL} if dataset is true-censored.


\end{sreapi}

