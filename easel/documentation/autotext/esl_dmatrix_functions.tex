\begin{sreapi}
\hypertarget{func:esl_dmatrix_Create()}
{\item[ESL\_DMATRIX * esl\_dmatrix\_Create(int n, int m)]}

Creates a general \ccode{n} x \ccode{m} matrix (\ccode{n} rows, \ccode{m} 
columns).

Returns a pointer to a new \ccode{ESL\_DMATRIX} object. Caller frees
with \ccode{esl\_dmatrix\_Destroy()}.

Throws \ccode{NULL} if an allocation failed.


\hypertarget{func:esl_dmatrix_CreateUpper()}
{\item[ESL\_DMATRIX * esl\_dmatrix\_CreateUpper(int n)]}

Creates a packed upper triangular matrix of \ccode{n} rows and
\ccode{n} columns. Caller may only access cells $i \leq j$.
Cells $i > j$ are not stored and are implicitly 0.

Not all matrix operations in Easel can work on packed
upper triangular matrices.

Returns a pointer to a new \ccode{ESL\_DMATRIX} object of type
\ccode{eslUPPER}. Caller frees with \ccode{esl\_dmatrix\_Destroy()}.

Throws \ccode{NULL} if allocation fails.



\hypertarget{func:esl_dmatrix_Destroy()}
{\item[int esl\_dmatrix\_Destroy(ESL\_DMATRIX *A)]}

Frees an \ccode{ESL\_DMATRIX} object \ccode{A}.


\hypertarget{func:esl_dmatrix_Copy()}
{\item[int esl\_dmatrix\_Copy(const ESL\_DMATRIX *src, ESL\_DMATRIX *dest)]}

Copies \ccode{src} matrix into \ccode{dest} matrix. \ccode{dest} must
be allocated already by the caller.

You may copy to a matrix of a different type, so long as
the copy makes sense. If \ccode{dest} matrix is a packed type
and \ccode{src} is not, the values that should be zeros must
be zero in \ccode{src}, else the routine throws
\ccode{eslEINCOMPAT}. If the \ccode{src} matrix is a packed type and
\ccode{dest} is not, the values that are implicitly zeros are
set to zeros in the \ccode{dest} matrix.

Returns \ccode{eslOK} on success.

Throws \ccode{eslEINCOMPAT} if \ccode{src}, \ccode{dest} are different sizes,
or if their types differ and \ccode{dest} cannot represent
\ccode{src}.


\hypertarget{func:esl_dmatrix_Clone()}
{\item[ESL\_DMATRIX * esl\_dmatrix\_Clone(const ESL\_DMATRIX *A)]}

Duplicates matrix \ccode{A}, making a copy in newly
allocated space.

Returns a pointer to the copy. Caller frees with 
\ccode{esl\_dmatrix\_Destroy()}.

Throws \ccode{NULL} on allocation failure.


\hypertarget{func:esl_dmatrix_Compare()}
{\item[int esl\_dmatrix\_Compare(const ESL\_DMATRIX *A, const ESL\_DMATRIX *B, double tol)]}

Compares matrix \ccode{A} to matrix \ccode{B} element by element,
using \ccode{esl\_DCompare()} on each cognate element pair,
with relative equality defined by a fractional tolerance
\ccode{tol}.  If all elements are equal, return \ccode{eslOK}; if
any elements differ, return \ccode{eslFAIL}.

\ccode{A} and \ccode{B} may be of different types; for example,
a packed upper triangular matrix A is compared to
a general matrix B by assuming \ccode{A->mx[i][j] = 0.} for
all $i>j$.


\hypertarget{func:esl_dmatrix_CompareAbs()}
{\item[int esl\_dmatrix\_CompareAbs(const ESL\_DMATRIX *A, const ESL\_DMATRIX *B, double tol)]}

Compares matrix \ccode{A} to matrix \ccode{B} element by element,
using \ccode{esl\_DCompareAbs()} on each cognate element pair,
with absolute equality defined by a absolute difference tolerance
\ccode{tol}.  If all elements are equal, return \ccode{eslOK}; if
any elements differ, return \ccode{eslFAIL}.

\ccode{A} and \ccode{B} may be of different types; for example,
a packed upper triangular matrix A is compared to
a general matrix B by assuming \ccode{A->mx[i][j] = 0.} for
all $i>j$.


\hypertarget{func:esl_dmatrix_Set()}
{\item[int esl\_dmatrix\_Set(ESL\_DMATRIX *A, double x)]}

Set all elements $a_{ij}$ in matrix \ccode{A} to \ccode{x},
and returns \ccode{eslOK}.


\hypertarget{func:esl_dmatrix_SetZero()}
{\item[int esl\_dmatrix\_SetZero(ESL\_DMATRIX *A)]}

Sets all elements $a_{ij}$ in matrix \ccode{A} to 0,
and returns \ccode{eslOK}.


\hypertarget{func:esl_dmatrix_SetIdentity()}
{\item[int esl\_dmatrix\_SetIdentity(ESL\_DMATRIX *A)]}

Given a square matrix \ccode{A}, sets all diagonal elements 
$a_{ii}$ to 1, and all off-diagonal elements $a_{ij},
j \ne i$ to 0. Returns \ccode{eslOK} on success.

Throws \ccode{eslEINVAL} if the matrix isn't square.


\hypertarget{func:esl_dmatrix_Dump()}
{\item[int esl\_dmatrix\_Dump(FILE *ofp, const ESL\_DMATRIX *A, const char *rowlabel, const char *collabel)]}

Given a matrix \ccode{A}, dump it to output stream \ccode{ofp} in human-readable
format.

If \ccode{rowlabel} or \ccode{collabel} are non-NULL, they specify a
string of single-character labels to put on the rows and
columns, respectively. (For example, these might be a
sequence alphabet for a 4x4 or 20x20 rate matrix or
substitution matrix.)  Numbers \ccode{1..ncols} or \ccode{1..nrows} are
used if \ccode{collabel} or \ccode{rowlabel} are passed as \ccode{NULL}.

Returns \ccode{eslOK} on success.


\hypertarget{func:esl_dmatrix_PlotHeatMap()}
{\item[int esl\_dmatrix\_PlotHeatMap(FILE *fp, ESL\_DMATRIX *D, double min, double max)]}

Export a heat map visualization of the matrix in \ccode{D}
to open stream \ccode{fp}, in PostScript format. 

All values between \ccode{min} and \ccode{max} in \ccode{D} are rescaled
linearly and assigned to shades. Values below \ccode{min}
are assigned to the lowest shade; values above \ccode{max}, to
the highest shade.

The plot is hardcoded to be a full US 8x11.5" page,
with at least a 20pt margin.

Several color schemes are enumerated in the code
but all but one is commented out. The currently enabled
scheme is a 10-class scheme consisting of the 9-class
Reds from colorbrewer2.org plus a blue background class.

Returns \ccode{eslOK} on success.

Throws (no abnormal error conditions)


\hypertarget{func:esl_permutation_Create()}
{\item[ESL\_PERMUTATION * esl\_permutation\_Create(int n)]}

Creates a new permutation "matrix" of size \ccode{n} for
permuting \ccode{n} x \ccode{n} square matrices; returns a 
pointer to it.

A permutation matrix consists of 1's and 0's such that
any given row or column contains only one 1. We store it
more efficiently as a vector; each value $p_i$
represents the column $j$ that has the 1. Thus, on
initialization, $p_i = i$ for all $i = 0..n-1$.

Returns a pointer to a new \ccode{ESL\_PERMUTATION} object. Free with 
\ccode{esl\_permutation\_Destroy()}.

Throws \ccode{NULL} if allocation fails.


\hypertarget{func:esl_permutation_Destroy()}
{\item[int esl\_permutation\_Destroy(ESL\_PERMUTATION *P)]}

Frees an \ccode{ESL\_PERMUTATION} object \ccode{P}.


\hypertarget{func:esl_permutation_Reuse()}
{\item[int esl\_permutation\_Reuse(ESL\_PERMUTATION *P)]}

Resets a permutation matrix \ccode{P} to
$p_i = i$ for all $i = 0..n-1$.

Returns \ccode{eslOK} on success.           


\hypertarget{func:esl_permutation_Dump()}
{\item[int esl\_permutation\_Dump(FILE *ofp, const ESL\_PERMUTATION *P, const char *rowlabel, const char *collabel)]}

Given a permutation matrix \ccode{P}, dump it to output stream \ccode{ofp}
in human-readable format.

If \ccode{rowlabel} or \ccode{collabel} are non-NULL, they represent
single-character labels to put on the rows and columns,
respectively. (For example, these might be a sequence
alphabet for a 4x4 or 20x20 rate matrix or substitution
matrix.)  Numbers 1..ncols or 1..nrows are used if
\ccode{collabel} or \ccode{rowlabel} are NULL.

Returns \ccode{eslOK} on success.


\hypertarget{func:esl_dmx_Max()}
{\item[double esl\_dmx\_Max(const ESL\_DMATRIX *A)]}

Returns the maximum value of all the elements $a_{ij}$ in matrix \ccode{A}.


\hypertarget{func:esl_dmx_Min()}
{\item[double esl\_dmx\_Min(const ESL\_DMATRIX *A)]}

Returns the minimum value of all the elements $a_{ij}$ in matrix \ccode{A}.


\hypertarget{func:esl_dmx_MinMax()}
{\item[int esl\_dmx\_MinMax(const ESL\_DMATRIX *A, double *ret\_min, double *ret\_max)]}

Finds the maximum and minimum values of the
elements $a_{ij}$ in matrix \ccode{A}, and returns
them in \ccode{ret\_min} and \ccode{ret\_max}.

Returns \ccode{eslOK} on success.            



\hypertarget{func:esl_dmx_Sum()}
{\item[double esl\_dmx\_Sum(const ESL\_DMATRIX *A)]}

Returns the scalar sum of all the elements $a_{ij}$ in matrix \ccode{A},
$\sum_{ij} a_{ij}$.


\hypertarget{func:esl_dmx_FrobeniusNorm()}
{\item[int esl\_dmx\_FrobeniusNorm(const ESL\_DMATRIX *A, double *ret\_fnorm)]}

Calculates the Frobenius norm of a matrix, which
is the element-wise equivalant of a 
Euclidean vector norm: 
$ = \sqrt(\sum a_{ij}^2)$

Returns \ccode{eslOK} on success, and the Frobenius norm
is in \ccode{ret\_fnorm}.


\hypertarget{func:esl_dmx_Multiply()}
{\item[int esl\_dmx\_Multiply(const ESL\_DMATRIX *A, const ESL\_DMATRIX *B, ESL\_DMATRIX *C)]}

Matrix multiplication: calculate \ccode{AB}, store result in \ccode{C}.
\ccode{A} is $n times m$; \ccode{B} is $m \times p$; \ccode{C} is $n \times p$.
Matrix \ccode{C} must be allocated appropriately by the caller.

Not supported for anything but general (\ccode{eslGENERAL})
matrix type, at present.

Throws \ccode{eslEINVAL} if matrices don't have compatible dimensions,
or if any of them isn't a general (\ccode{eslGENERAL}) matrix.


\hypertarget{func:esl_dmx_Exp()}
{\item[int esl\_dmx\_Exp(const ESL\_DMATRIX *Q, double t, ESL\_DMATRIX *P)]}

Calculates the matrix exponential $\mathbf{P} = e^{t\mathbf{Q}}$,
using a scaling and squaring algorithm with
the Taylor series approximation \citep{MolerVanLoan03}.

\ccode{Q} must be a square matrix of type \ccode{eslGENERAL}.
Caller provides an allocated \ccode{P} matrix of the same size and type as \ccode{Q}.

A typical use of this function is to calculate a
conditional substitution probability matrix $\mathbf{P}$
(whose elements $P_{xy}$ are conditional substitution
probabilities $\mathrm{Prob}(y \mid x, t)$ from time $t$
and instantaneous rate matrix $\mathbf{Q}$.

Returns \ccode{eslOK} on success.

Throws \ccode{eslEMEM} on allocation error.



\hypertarget{func:esl_dmx_Transpose()}
{\item[int esl\_dmx\_Transpose(ESL\_DMATRIX *A)]}

Transpose a square matrix \ccode{A} in place.

\ccode{A} must be a general (\ccode{eslGENERAL}) matrix type.

Throws \ccode{eslEINVAL} if \ccode{A} isn't square, or if it isn't
of type \ccode{eslGENERAL}.


\hypertarget{func:esl_dmx_Add()}
{\item[int esl\_dmx\_Add(ESL\_DMATRIX *A, const ESL\_DMATRIX *B)]}

\ccode{A = A+B}; adds matrix \ccode{B} to matrix \ccode{A} and leaves result
in matrix \ccode{A}.

\ccode{A} and \ccode{B} may be of any type. However, if \ccode{A} is a
packed upper triangular matrix (type
\ccode{eslUPPER}), all values $i>j$ in \ccode{B} must be
zero (i.e. \ccode{B} must also be upper triangular, though
not necessarily packed upper triangular).

Throws \ccode{eslEINVAL} if matrices aren't the same dimensions, or
if \ccode{A} is \ccode{eslUPPER} and any cell $i>j$ in
\ccode{B} is nonzero.


\hypertarget{func:esl_dmx_Scale()}
{\item[int esl\_dmx\_Scale(ESL\_DMATRIX *A, double k)]}

Calculates \ccode{A = kA}: multiply matrix \ccode{A} by scalar
\ccode{k} and leave answer in \ccode{A}.


\hypertarget{func:esl_dmx_AddScale()}
{\item[int esl\_dmx\_AddScale(ESL\_DMATRIX *A, double k, const ESL\_DMATRIX *B)]}

Calculates \ccode{A + kB}, leaves answer in \ccode{A}.

Only defined for matrices of the same type (\ccode{eslGENERAL}
or \ccode{eslUPPER}).

Throws \ccode{eslEINVAL} if matrices aren't the same dimensions, or
of different types.


\hypertarget{func:esl_dmx_Permute_PA()}
{\item[int esl\_dmx\_Permute\_PA(const ESL\_PERMUTATION *P, const ESL\_DMATRIX *A, ESL\_DMATRIX *B)]}

Computes \ccode{B = PA}: do a row-wise permutation of a square
matrix \ccode{A}, using the permutation matrix \ccode{P}, and put
the result in a square matrix \ccode{B} that the caller has
allocated.

Throws \ccode{eslEINVAL} if \ccode{A}, \ccode{B}, \ccode{P} do not have compatible dimensions,
or if \ccode{A} or \ccode{B} is not of type \ccode{eslGENERAL}.


\hypertarget{func:esl_dmx_LUP_decompose()}
{\item[int esl\_dmx\_LUP\_decompose(ESL\_DMATRIX *A, ESL\_PERMUTATION *P)]}

Calculates a permuted LU decomposition of square matrix
\ccode{A}; upon return, \ccode{A} is replaced by this decomposition,
where \ccode{U} is in the lower triangle (inclusive of the 
diagonal) and \ccode{L} is the upper triangle (exclusive of
diagonal, which is 1's by definition), and \ccode{P} is the
permutation matrix. Caller provides an allocated 
permutation matrix \ccode{P} compatible with the square matrix
\ccode{A}.

Implements Gaussian elimination with pivoting 
\citep[p.~759]{Cormen99}.

Throws \ccode{eslEINVAL} if \ccode{A} isn't square, or if \ccode{P} isn't the right
size for \ccode{A}, or if \ccode{A} isn't of general type.


\hypertarget{func:esl_dmx_LU_separate()}
{\item[int esl\_dmx\_LU\_separate(const ESL\_DMATRIX *LU, ESL\_DMATRIX *L, ESL\_DMATRIX *U)]}

Separate a square \ccode{LU} decomposition matrix into its two
triangular matrices \ccode{L} and \ccode{U}. Caller provides two
allocated \ccode{L} and \ccode{U} matrices of same size as \ccode{LU} for
storing the results.

\ccode{U} may be an upper triangular matrix in either unpacked
(\ccode{eslGENERAL}) or packed (\ccode{eslUPPER}) form.
\ccode{LU} and \ccode{L} must be of \ccode{eslGENERAL} type.

Throws \ccode{eslEINVAL} if \ccode{LU}, \ccode{L}, \ccode{U} are not of compatible dimensions,
or if \ccode{LU} or \ccode{L} aren't of general type. 


\hypertarget{func:esl_dmx_Invert()}
{\item[int esl\_dmx\_Invert(const ESL\_DMATRIX *A, ESL\_DMATRIX *Ai)]}

Calculates the inverse of square matrix \ccode{A}, and stores the
result in matrix \ccode{Ai}. Caller provides an allocated
matrix \ccode{Ai} of same dimensions as \ccode{A}. Both must be
of type \ccode{eslGENERAL}.

Peforms the inversion by LUP decomposition followed by 
forward/back-substitution \citep[p.~753]{Cormen99}.

Throws \ccode{eslEINVAL} if \ccode{A}, \ccode{Ai} do not have same dimensions, 
if \ccode{A} isn't square, or if either isn't of
type \ccode{eslGENERAL}.
\ccode{eslEMEM}   if internal allocations (for LU, and some other
bookkeeping) fail.


\hypertarget{func:esl_dmx_Diagonalize()}
{\item[int esl\_dmx\_Diagonalize(const ESL\_DMATRIX *A, double **ret\_Er, double **ret\_Ei, 
		    ESL\_DMATRIX **ret\_UL, ESL\_DMATRIX **ret\_UR)]}

Given a square real matrix \ccode{A}, diagonalize it:
solve for $U^{-1} A U = diag(\lambda_1... \lambda_n)$.

Upon return, \ccode{ret\_Er} and \ccode{ret\_Ei} are vectors
containing the real and complex parts of the eigenvalues
$\lambda_i$; \ccode{ret\_UL} is the $U^{-1}$ matrix containing
the left eigenvectors; and \ccode{ret\_UR} is the $U$ matrix
containing the right eigenvectors.

\ccode{ret\_UL} and \ccode{ret\_UR} are optional; pass \ccode{NULL} for
either if you don't want that set of eigenvectors.

This is a C interface to the \ccode{dgeev()} routine in the
LAPACK linear algebra library.

Returns \ccode{eslOK} on success.
\ccode{ret\_Er} and \ccode{ret\_Ei} (and \ccode{ret\_UL},\ccode{ret\_UR} when they are
requested) are allocated here, and must be free'd by the caller.

Throws \ccode{eslEMEM} on allocation failure.
In this case, the four return pointers are returned \ccode{NULL}.



\end{sreapi}

