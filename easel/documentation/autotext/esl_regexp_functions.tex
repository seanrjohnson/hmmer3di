\begin{sreapi}
\hypertarget{func:esl_regexp_Create()}
{\item[ESL\_REGEXP * esl\_regexp\_Create(void)]}

Creates a new \ccode{ESL\_REGEXP} machine.

Throws NULL on allocation failure.



\hypertarget{func:esl_regexp_Destroy()}
{\item[void esl\_regexp\_Destroy(ESL\_REGEXP *machine)]}

Destroy a machine created by \ccode{esl\_regexp\_Create()}.

Returns void.


\hypertarget{func:esl_regexp_Compile()}
{\item[int esl\_regexp\_Compile(ESL\_REGEXP *machine, const char *pattern)]}

Precompile an NDFA for \ccode{pattern} and store it in 
a \ccode{machine}, in preparation for using the same
pattern for multiple searches.

Returns \ccode{eslOK} on success.

Throws \ccode{eslEINVAL} if compilation fails.


\hypertarget{func:esl_regexp_Match()}
{\item[int esl\_regexp\_Match(ESL\_REGEXP *machine, const char *pattern, const char *s)]}

Determine if string \ccode{s} matches the regular expression
\ccode{pattern}, using a \ccode{machine}. Upon return, \ccode{machine}
contains the compiled \ccode{pattern}.

If \ccode{pattern} is \ccode{NULL}, use the last pattern compiled
into the \ccode{machine}. 

If there's a match, return \ccode{eslOK}, and the \ccode{machine}
contains information about the match, which you can
extract with \ccode{esl\_regexp\_Submatch*()} functions.

If there's no match, return \ccode{eslEOD}.

Returns \ccode{eslOK} if \ccode{pattern} matches \ccode{s}; \ccode{eslEOD} if it doesn't.

Throws \ccode{eslEINVAL} if the \ccode{pattern} couldn't be compiled for any reason.
\ccode{eslEINCONCEIVABLE} or \ccode{eslECORRUPT} if something
went wrong in the search phase.

(At the failure point, an error was generated with an appropriate
code and message; an \ccode{ESL\_SYNTAX} code, for example, may have
been generated to indicate that the \ccode{pattern} is an invalid syntax.)


\hypertarget{func:esl_regexp_MultipleMatches()}
{\item[int esl\_regexp\_MultipleMatches(ESL\_REGEXP *machine, char **sptr)]}

Given a \ccode{machine} that contains a precompiled NDFA (see
\ccode{esl\_regexp\_Compile()}, search it against a \ccode{string}.
pointed to by \ccode{sptr}. When a match is found, returns
\ccode{eslOK}, and resets \ccode{sptr} to point at the next character
after the matched substring. (This may be 
trailing NUL byte if the matched substring is at the
very end of the string.)  If no match is found in the
string, returns \ccode{eslEOD}.

Because \ccode{sptr} is changed, the caller should
initialize and use a temporary pointer into the string
to be searched, not the caller's own pointer to the
target string.

Throws \ccode{eslEINCONCEIVABLE} or \ccode{eslECORRUPT} if something goes awry internally
during the search.


\hypertarget{func:esl_regexp_GetMatch()}
{\item[int esl\_regexp\_GetMatch(ESL\_REGEXP *machine, int which, char **ret\_s, esl\_pos\_t *ret\_n)]}

Given a \ccode{machine} that just got done matching a pattern
against a target string, retrieve a pointer to and a
length of text that matched. \ccode{which} indicates which
submatch to retrieve; 0 means the entire match, and
1..15 are up to 15 ()'d submatches in the pattern.

Returns \ccode{eslOK} on success. Now \ccode{*ret\_s} points to the start of
the match, and \ccode{*ret\_n} is its length.

Throws (no abnormal error conditions)


\hypertarget{func:esl_regexp_SubmatchDup()}
{\item[char * esl\_regexp\_SubmatchDup(ESL\_REGEXP *machine, int elem)]}

Given a \ccode{machine} that has just got done matching 
some pattern against a target string, 
retrieve a substring that matched the pattern
or one of the ()'d parts of it. \ccode{elem} indicates
which submatch to retrieve. \ccode{elem} 0 is the complete
match;  1..15 (assuming the default \ccode{ESL\_REGEXP\_NSUB}=16)
are up to 15 ()'d submatches in the pattern.

Returns ptr to an allocated, NUL-terminated string containing
the matched part of the string. Caller is responsible
for free'ing this string.

Throws NULL on any internal failure.


\hypertarget{func:esl_regexp_SubmatchCopy()}
{\item[int esl\_regexp\_SubmatchCopy(ESL\_REGEXP *machine, int elem, char *buffer, int nc)]}

Given a \ccode{machine} that has just got done matching some
pattern against a target string, copy a substring that
matched the pattern or one of the ()'d parts of it into
a provided \ccode{buffer} with \ccode{nc} chars of space allocated.
\ccode{elem} indicates which submatch to retrieve. \ccode{elem} 0 is
the complete match; 1..15 (assuming the default
\ccode{ESL\_REGEXP\_NSUB}=16) are up to 15 ()'d submatches in
the pattern.

Returns \ccode{eslOK} on success, and buffer contains the NUL-terminated
substring. 

Throws \ccode{eslEINVAL} on any of several possible internal failures,
including the \ccode{buffer} being too small to contain the 
substring.


\hypertarget{func:esl_regexp_SubmatchCoords()}
{\item[int esl\_regexp\_SubmatchCoords(ESL\_REGEXP *machine, char *origin, int elem, 
			  int *ret\_start, int *ret\_end)]}

Given a \ccode{machine} that has just got done matching some
pattern against a target string, find the start/end
coordinates of the substring that matched the
pattern or one of the ()'d parts of it, relative to
a pointer \ccode{origin} on the target string. Return the result
through the ptrs \ccode{ret\_start} and \ccode{ret\_end}.  \ccode{elem}
indicates which submatch to retrieve. \ccode{elem} 0 is the
complete match; 1..15 (assuming the default
\ccode{ESL\_REGEXP\_NSUB} = 16) are up to 15 ()'d submatches in
the pattern.

The coordinates given in zero-offset convention relative
to an \ccode{origin}. \ccode{origin} will usually be a pointer to
the complete target string, in which case the coords
would be [0..L-1]. However, one can extract coords
relative to any other \ccode{origin} in the target string,
even including an \ccode{origin} downstream of the match, so
relative coords can be negative, ranging from -(L-1) to
(L-1).

Coords will be correct even if the match was
found by a \ccode{esl\_regexp\_MultipleMatches()} call against
a temp pointer into the target string.

Returns \ccode{eslOK} on success, and \ccode{ret\_start} and \ccode{ret\_end}
are set to the start/end coordinates of the submatch.

Throws \ccode{eslEINVAL} on internal failures.
The function is incapable of detecting a case in
where \ccode{origin} is not in the same string that the
\ccode{machine} matched like it should be. If a caller does
this, the function may appear to succeed, but start and end  
coords will be garbage.


\hypertarget{func:esl_regexp_ParseCoordString()}
{\item[int esl\_regexp\_ParseCoordString(const char *cstring, int64\_t *ret\_start, int64\_t *ret\_end)]}

Given a string \ccode{cstring} of the format required for a
range (\ccode{from}..\ccode{to}, e.g. 10..23  or 39-91) parse out
the start and end, and return them within the variables
\ccode{ret\_start} and \ccode{ret\_end}.

Returns \ccode{eslOK} on success, and \ccode{ret\_start} and \ccode{ret\_end}
are set to the start/end coordinates of the parse.

Throws \ccode{eslESYNTAX} if a regexp match is not made, and
\ccode{eslFAIL} if the start or end values are not parsed.


\end{sreapi}

