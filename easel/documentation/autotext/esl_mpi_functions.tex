\begin{sreapi}
\hypertarget{func:esl_mpi_PackOpt()}
{\item[int esl\_mpi\_PackOpt(void *inbuf, int incount, MPI\_Datatype type, void *pack\_buf, int pack\_buf\_size, int *position, MPI\_Comm comm)]}

Pack data array \ccode{inbuf} of \ccode{incount} elements of type \ccode{type} into
an MPI packed buffer \ccode{pack\_buf} of total size \ccode{pack\_buf\_size} destined
for MPI communicator \ccode{comm} that is currently filled to position \ccode{*position}.

\ccode{inbuf} may be \ccode{NULL}, in which case \ccode{incount} is
assumed to be 0, and a `null array' is packed that
\ccode{esl\_mpi\_UnpackOpt()} knows how to decode as a \ccode{NULL}
pointer.

As a special case for strings, if \ccode{type} is \ccode{MPI\_CHAR},
\ccode{incount} may be passed as \ccode{-1} to indicate `unknown';
the routine will use \ccode{strlen(inbuf)+1} to determine the
size of the string including its \ccode{NUL} terminator.

Returns \ccode{eslOK} on success, the array is packed into \ccode{pack\_buf}, 
and the \ccode{*position} counter is updated to point to the next byte
in \ccode{pack\_buf} after the packed array.

Throws \ccode{eslESYS} if an MPI call fails.


\hypertarget{func:esl_mpi_PackOptSize()}
{\item[int esl\_mpi\_PackOptSize(void *inbuf, int incount, MPI\_Datatype type, MPI\_Comm comm, int *ret\_n)]}

Determine an upper bound on the size (in bytes) required
to pack an array \ccode{inbuf} of \ccode{incount} elements of type
\ccode{type} destined for MPI communicator \ccode{comm} using
\ccode{esl\_mpi\_PackOpt()}, and return it in \ccode{*ret\_n}.

If \ccode{inbuf} is non-\ccode{NULL}, the packed message consists
of 1 integer (the length, \ccode{incount}) followed by the array.
If \ccode{inbuf} is \ccode{NULL}, the packed message consists of one
integer (0). 

As a special case for strings, if \ccode{type} is \ccode{MPI\_CHAR},
\ccode{incount} may be passed as \ccode{-1} to indicate `unknown';
in this case, the routine uses \ccode{strlen(inbuf)+1} to determine the
size of the string including its \ccode{NUL} terminator.

Returns \ccode{eslOK} on success, and \ccode{*ret\_n} contains the upper limit size in
bytes.

Throws \ccode{eslESYS} if an MPI call fails, and \ccode{*ret\_n} is 0.           


\hypertarget{func:esl_mpi_UnpackOpt()}
{\item[int esl\_mpi\_UnpackOpt(void *pack\_buf, int pack\_buf\_size, int *pos, void **outbuf, int *opt\_n, MPI\_Datatype type, MPI\_Comm comm)]}

Unpack a packed MPI message in buffer \ccode{pack\_buf}, of total size
\ccode{pack\_buf\_size}, at current position \ccode{*pos} in \ccode{pack\_buf},
for MPI communicator \ccode{comm}, where the next packed element is an optional
array of type \ccode{type}, consisting of a \ccode{(n,data)} pair, with \ccode{n=0}
indicating no data. 

If array data is present (\ccode{n}0>), allocate \ccode{*outbuf},
put the array in it, and optionally return \ccode{n} in
\ccode{*opt\_n}. The caller is responsible for free'ing this
\ccode{*outbuf}.

If data are not present (\ccode{n=0}), no allocation is done,
\ccode{*outbuf} is set to \ccode{NULL}, and the optional \ccode{*opt\_n} is
0.

\ccode{*pos} is updated to point at the next element in \ccode{pack\_buf}
that needs to be unpacked.

This routine is designed for an optional-array idiom in
which \ccode{array==NULL} means the array isn't available, and
otherwise the array contains valid data. For instance,
this is used for optional annotation on multiple
alignments. 

Returns \ccode{eslOK} on success; \ccode{*pos} is updated; \ccode{*outbuf} is either a newly allocated 
array (that caller is responsible for freeing) and optional \ccode{*opt\_n}
is its length, or \ccode{*outbuf} is \ccode{NULL} and optional \ccode{*opt\_n} is 0.

Throws \ccode{eslESYS} on an MPI call failure; \ccode{eslEINVAL} if something's wrong
with the arguments; \ccode{eslEMEM} on allocation failure. 
In either case, \ccode{*outbuf} is \ccode{NULL} and optional \ccode{*opt\_n} is 0.


\hypertarget{func:esl_sq_MPISend()}
{\item[int esl\_sq\_MPISend(ESL\_SQ *sq, int dest, int tag, MPI\_Comm comm, char **buf, int *nalloc)]}

Sends an \ccode{ESL\_SQ} \ccode{esl\_sq} as a work unit to MPI process
\ccode{dest} (where \ccode{dest} ranges from 0..\ccode{nproc-1}), tagged
with MPI tag \ccode{tag}, for MPI communicator \ccode{comm}, as 
the sole workunit or result. 

Work units are prefixed by a status code. If \ccode{esl\_sq} is
\ccode{non-NULL}, the work unit is an \ccode{eslOK} code followed by
the packed \ccode{ESL\_SQ}. If \ccode{esl\_sq} is NULL, the work unit is an
\ccode{eslEOD} code, which \ccode{esl\_sq\_MPIRecv()} knows how to
interpret; this is typically used for an end-of-data
signal to cleanly shut down worker processes.

In order to minimize alloc/free cycles in this routine,
caller passes a pointer to a working buffer \ccode{*buf} of
size \ccode{*nalloc} characters. If necessary (i.e. if \ccode{esl\_sq} is
too big to fit), \ccode{*buf} will be reallocated and \ccode{*nalloc}
increased to the new size. As a special case, if \ccode{*buf}
is \ccode{NULL} and \ccode{*nalloc} is 0, the buffer will be
allocated appropriately, but the caller is still
responsible for free'ing it.

Returns \ccode{eslOK} on success; \ccode{*buf} may have been reallocated and
\ccode{*nalloc} may have been increased.

Throws \ccode{eslESYS} if an MPI call fails; \ccode{eslEMEM} if a malloc/realloc
fails. In either case, \ccode{*buf} and \ccode{*nalloc} remain valid and useful
memory (though the contents of \ccode{*buf} are undefined). 



\hypertarget{func:esl_sq_MPIPackSize()}
{\item[int esl\_sq\_MPIPackSize(ESL\_SQ *sq, MPI\_Comm comm, int *ret\_n)]}

Calculate an upper bound on the number of bytes
that \ccode{esl\_sq\_MPIPack()} will need to pack an \ccode{ESL\_SQ}
\ccode{sq} in a packed MPI message for MPI communicator
\ccode{comm}; return that number of bytes in \ccode{*ret\_n}.

Returns \ccode{eslOK} on success, and \ccode{*ret\_n} contains the answer.

Throws \ccode{eslESYS} if an MPI call fails, and \ccode{*ret\_n} is 0.


\hypertarget{func:esl_sq_MPIPack()}
{\item[int esl\_sq\_MPIPack(ESL\_SQ *sq, char *buf, int n, int *pos, MPI\_Comm comm)]}

Packs \ccode{ESL\_SQ} \ccode{esl\_sq} into an MPI packed message buffer \ccode{buf}
of length \ccode{n} bytes, starting at byte position \ccode{*position},
for MPI communicator \ccode{comm}.

The caller must know that \ccode{buf}'s allocation of \ccode{n}
bytes is large enough to append the packed \ccode{ESL\_SQ} at
position \ccode{*pos}. This typically requires a call to
\ccode{esl\_sq\_MPIPackSize()} first, and reallocation if
needed.

Returns \ccode{eslOK} on success; \ccode{buf} now contains the
packed \ccode{esl\_sq}, and \ccode{*position} is set to the byte
immediately following the last byte of the \ccode{ESL\_SQ}
in \ccode{buf}. 

Throws \ccode{eslESYS} if an MPI call fails; or \ccode{eslEMEM} if the
buffer's length \ccode{n} was overflowed in trying to pack
\ccode{sq} into \ccode{buf}. In either case, the state of
\ccode{buf} and \ccode{*position} is undefined, and both should
be considered to be corrupted.


\hypertarget{func:esl_sq_MPIUnpack()}
{\item[int esl\_sq\_MPIUnpack(const ESL\_ALPHABET *abc, char *buf, int n, int *pos, MPI\_Comm comm, ESL\_SQ **ret\_sq)]}

Unpack a newly allocated \ccode{ESL\_SQ} from MPI packed buffer
\ccode{buf}, starting from position \ccode{*pos}, where the total length
of the buffer in bytes is \ccode{n}. 

Caller may or may not already know what alphabet the \ccode{ESL\_SQ}
is expected to be in.  

Returns \ccode{eslOK} on success. \ccode{*pos} is updated to the position of
the next element in \ccode{buf} to unpack (if any). \ccode{*ret\_hmm}
contains a newly allocated \ccode{ESL\_SQ}, which the caller is
responsible for free'ing. 


Throws \ccode{eslESYS} on an MPI call failure. \ccode{eslEMEM} on allocation failure.
In either case, \ccode{*ret\_esl\_sq} is \ccode{NULL}, and the state of \ccode{buf}
and \ccode{*pos} is undefined and should be considered to be corrupted.


\hypertarget{func:esl_sq_MPIRecv()}
{\item[int esl\_sq\_MPIRecv(int source, int tag, MPI\_Comm comm, const ESL\_ALPHABET *abc, char **buf, int *nalloc, ESL\_SQ **ret\_sq)]}

Receive a work unit that consists of a single \ccode{ESL\_SQ}
sent by MPI \ccode{source} (\ccode{0..nproc-1}, or
\ccode{MPI\_ANY\_SOURCE}) tagged as \ccode{tag} for MPI communicator \ccode{comm}.

Work units are prefixed by a status code. If the unit's
code is \ccode{eslOK} and no errors are encountered, this
routine will return \ccode{eslOK} and a non-\ccode{NULL} \ccode{*ret\_esl\_sq}.
If the unit's code is \ccode{eslEOD} (a shutdown signal), 
this routine returns \ccode{eslEOD} and \ccode{*ret\_esl\_sq} is \ccode{NULL}.

Caller provides a working buffer \ccode{*buf} of size
\ccode{*nalloc} characters. These are passed by reference, so
that \ccode{*buf} can be reallocated and \ccode{*nalloc} increased
if necessary. As a special case, if \ccode{*buf} is \ccode{NULL} and
\ccode{*nalloc} is 0, the buffer will be allocated
appropriately, but the caller is still responsible for
free'ing it.

Returns \ccode{eslOK} on success. \ccode{*ret\_esl\_sq} contains the received \ccode{ESL\_SQ};
it is allocated here, and the caller is responsible for
free'ing it.  \ccode{*buf} may have been reallocated to a
larger size, and \ccode{*nalloc} may have been increased. 

Throws \ccode{eslEMEM} on allocation error, in which case \ccode{*ret\_esl\_sq} is 
\ccode{NULL}.           


\hypertarget{func:esl_msa_MPISend()}
{\item[int esl\_msa\_MPISend(const ESL\_MSA *msa, int dest, int tag, MPI\_Comm comm, char **buf, int *nalloc)]}

Sends the essential elements of a multiple alignment \ccode{msa} 
as a work unit to MPI process \ccode{dest} (\ccode{dest} ranges from \ccode{0..nproc-1}),
tagging the message with MPI tag \ccode{tag} for MPI communicator
\ccode{comm}. The receiver uses \ccode{esl\_msa\_MPIRecv()} to receive the MSA.

Work units are prefixed by a status code. If \ccode{msa} is
\ccode{non-NULL}, the work unit is an \ccode{eslOK} code followed by
the packed MSA. If \ccode{msa} is NULL, the work unit is an
\ccode{eslEOD} code, which \ccode{esl\_msa\_hmm\_MPIRecv()} knows how
to interpret; this is typically used for an end-of-data
signal to cleanly shut down worker processes.

Only an essential subset of the elements in \ccode{msa} are
transmitted, sufficient to do computationally intensive
work on the \ccode{msa}. Most msa annotation is not
transmitted, for example. Specifically, \ccode{name}, \ccode{nseq},
\ccode{alen}, \ccode{flags}, \ccode{wgt}, \ccode{ax} or \ccode{aseq}, \ccode{desc}, \ccode{acc},
\ccode{au}, \ccode{ss\_cons}, \ccode{sa\_cons}, \ccode{rf}, \ccode{cutoff}, and \ccode{cutset}
are transmitted.

In order to minimize alloc/free cycles, caller passes a
pointer to a working buffer \ccode{*buf} of size \ccode{*nalloc}
characters. If necessary (i.e. if \ccode{msa} is too big to
fit), \ccode{*buf} will be reallocated and \ccode{*nalloc} increased
to the new size. As a special case, if \ccode{*buf} is \ccode{NULL}
and \ccode{*nalloc} is 0, the buffer will be allocated
appropriately, but the caller is still responsible for
free'ing it.

Returns \ccode{eslOK} on success; \ccode{*buf} may have been reallocated and
\ccode{*nalloc} may have been increased.

Throws \ccode{eslESYS} if an MPI call fails; \ccode{eslEMEM} if a malloc/realloc
fails. In either case, \ccode{*buf} and \ccode{*nalloc} remain valid and useful
memory (though the contents of \ccode{*buf} are undefined). 



\hypertarget{func:esl_msa_MPIPackSize()}
{\item[int esl\_msa\_MPIPackSize(const ESL\_MSA *msa, MPI\_Comm comm, int *ret\_n)]}

Calculate an upper bound on the number of bytes
that \ccode{esl\_msa\_MPIPack()} will need to pack an 
essential subset of the data in MSA \ccode{msa}
in a packed MPI message in communicator \ccode{comm};
return that number of bytes in \ccode{*ret\_n}. 

Caller will generally use this result to determine how
to allocate a buffer before starting to pack into it.

If \ccode{msa} is \ccode{NULL} (which can happen, if \ccode{msa} is
optional in the caller), size \ccode{*ret\_n} is set to 0.

Returns \ccode{eslOK} on success, and \ccode{*ret\_n} contains the answer.

Throws \ccode{eslESYS} if an MPI call fails, and \ccode{*ret\_n} is set to 0. 



\hypertarget{func:esl_msa_MPIPack()}
{\item[int esl\_msa\_MPIPack(const ESL\_MSA *msa, char *buf, int n, int *position, MPI\_Comm comm)]}

Packs essential subset of data in MSA \ccode{msa} into an MPI
packed message buffer \ccode{buf} of length \ccode{n} bytes,
starting at byte position \ccode{*position}, for MPI
communicator \ccode{comm}.

If \ccode{msa} is \ccode{NULL} (which can happen, if \ccode{msa} is being
treated as optional in the caller), does nothing, and
just return \ccode{eslOK}.

Returns \ccode{eslOK} on success; \ccode{buf} now contains the
packed \ccode{msa}, and \ccode{*position} is set to the byte
immediately following the last byte of the MSA
in \ccode{buf}. 

Throws \ccode{eslESYS} if an MPI call fails; or \ccode{eslEMEM} if the
buffer's length \ccode{n} is overflowed by trying to pack
\ccode{msa} into \ccode{buf}. In either case, the state of
\ccode{buf} and \ccode{*position} is undefined, and both should
be considered to be corrupted.



\hypertarget{func:esl_msa_MPIUnpack()}
{\item[int esl\_msa\_MPIUnpack(const ESL\_ALPHABET *abc, char *buf, int n, int *pos, MPI\_Comm comm, ESL\_MSA **ret\_msa)]}

Unpack a newly allocated MSA from MPI packed buffer
\ccode{buf}, starting from position \ccode{*pos}, where the total length
of the buffer in bytes is \ccode{n}. 

MSAs are usually transmitted in digital mode. In digital
mode, caller must provide the alphabet \ccode{abc} for this
MSA. (Thus the caller already know it before the MSA
arrives, by an appropriate initialization.) If MSAs are
being transmitted in text mode, \ccode{abc} is ignored; caller
may pass \ccode{NULL} for it.

Returns \ccode{eslOK} on success. \ccode{*pos} is updated to the position of
the next element in \ccode{buf} to unpack (if any). \ccode{*ret\_msa}
contains a newly allocated MSA, which the caller is 
responsible for free'ing.

Throws \ccode{eslESYS} on an MPI call failure. \ccode{eslEMEM} on allocation failure.
In either case, \ccode{*ret\_msa} is \ccode{NULL}, and the state of \ccode{buf}
and \ccode{*pos} is undefined and should be considered to be corrupted.



\hypertarget{func:esl_msa_MPIRecv()}
{\item[int esl\_msa\_MPIRecv(int source, int tag, MPI\_Comm comm, const ESL\_ALPHABET *abc, char **buf, int *nalloc, ESL\_MSA **ret\_msa)]}

Receives a work unit that consists of a single MSA from \ccode{source} (\ccode{0..nproc-1}, or
\ccode{MPI\_ANY\_SOURCE}) tagged as \ccode{tag} from communicator \ccode{comm}.

Work units are prefixed by a status code. If the unit's
code is \ccode{eslOK} and no errors are encountered, this
routine will return \ccode{eslOK} and a non-\ccode{NULL} \ccode{*ret\_msa}.
If the unit's code is \ccode{eslEOD} (a shutdown signal), 
this routine returns \ccode{eslEOD} and \ccode{*ret\_msa} is \ccode{NULL}.

MSAs are transmitted in digital mode. Caller must know and
provide the alphabet \ccode{abc} for this MSA.

To minimize alloc/free cycles in this routine, caller
passes a pointer to a buffer \ccode{*buf} of size \ccode{*nalloc}
characters. These are passed by reference, because when
necessary, \ccode{*buf} will be reallocated and \ccode{*nalloc}
increased to the new size. As a special case, if \ccode{*buf}
is \ccode{NULL} and \ccode{*nalloc} is 0, the buffer will be
allocated appropriately, but the caller is still
responsible for free'ing it.

If the packed MSA is an end-of-data signal, return
\ccode{eslEOD}, and \ccode{*ret\_msa} is \ccode{NULL}.

Returns \ccode{eslOK} on success. \ccode{*ret\_msa} contains the new MSA; it
is allocated here, and the caller is responsible for
free'ing it.  \ccode{*buf} may have been reallocated to a
larger size, and \ccode{*nalloc} may have been increased.


Throws \ccode{eslESYS} if an MPI call fails; \ccode{eslEMEM} if an allocation fails.
In either case, \ccode{*ret\_msa} is NULL, and the \ccode{buf} and its size
\ccode{*nalloc} remain valid.


\hypertarget{func:esl_stopwatch_MPIReduce()}
{\item[int esl\_stopwatch\_MPIReduce(ESL\_STOPWATCH *w, int root, MPI\_Comm comm)]}

Collect all user/sys times from stopped stopwatch \ccode{w} from
all MPI processes, and sum them into the watch on the
master process of rank \ccode{root}, for MPI communicator
\ccode{comm}.  A subsequent \ccode{esl\_stopwatch\_Display()} will
then show total user/sys times, not just the master's
usage.

This routine needs to be called synchronously on all
processes; it does a collective communication using
\ccode{MPI\_Reduce()}.

Returns \ccode{eslOK} on success.

Throws \ccode{eslESYS} on MPI call failure.


\end{sreapi}

