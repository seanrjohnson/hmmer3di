\begin{sreapi}
\hypertarget{func:esl_msacluster_SingleLinkage()}
{\item[int esl\_msacluster\_SingleLinkage(const ESL\_MSA *msa, double maxid, 
			     int **opt\_c, int **opt\_nin, int *opt\_nc)]}

Perform single link clustering of the sequences in 
multiple alignment \ccode{msa}. Any pair of sequences with
percent identity $\geq$ \ccode{maxid} are linked (using
the definition from the \eslmod{distance} module).

The resulting clustering is optionally returned in one
or more of \ccode{opt\_c}, \ccode{opt\_nin}, and \ccode{opt\_nc}.  The
\ccode{opt\_c[0..nseq-1]} array assigns a cluster index
\ccode{(0..nc-1)} to each sequence. For example, \ccode{c[4] = 1}
means that sequence 4 is assigned to cluster 1.  The
\ccode{opt\_nin[0..nc-1]} array is the number of sequences
in each cluster. \ccode{opt\_nc} is the number of clusters.

Importantly, this algorithm runs in $O(N)$ memory, and
produces one discrete clustering. Compare to
\ccode{esl\_tree\_SingleLinkage()}, which requires an $O(N^2)$ 
adjacency matrix, and produces a hierarchical clustering
tree.

The algorithm is worst case $O(LN^2)$ time, for $N$
sequences of length $L$. However, the worst case is no
links at all, and this is unusual. More typically, time
scales as about $LN \log N$. The best case scales as
$LN$, when there is just one cluster in a completely
connected graph.

Returns \ccode{eslOK} on success; the \ccode{opt\_c[0..nseq-1]} array contains
cluster indices \ccode{0..nc-1} assigned to each sequence; the
\ccode{opt\_nin[0..nc-1]} array contains the number of seqs in
each cluster; and \ccode{opt\_nc} contains the number of
clusters. The \ccode{opt\_c} array and \ccode{opt\_nin} arrays will be
allocated here, if non-\ccode{NULL}, and must be free'd by the
caller. The input \ccode{msa} is unmodified.

The caller may pass \ccode{NULL} for either \ccode{opt\_c} or
\ccode{opt\_nc} if it is only interested in one of the two
results.

Throws \ccode{eslEMEM} on allocation failure, and \ccode{eslEINVAL} if a pairwise
comparison is invalid (which means the MSA is corrupted, so it
shouldn't happen). In either case, \ccode{opt\_c} and \ccode{opt\_nin} are set to \ccode{NULL}
and \ccode{opt\_nc} is set to 0, and the \ccode{msa} is unmodified.


\end{sreapi}

