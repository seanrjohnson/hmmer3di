\begin{sreapi}
\hypertarget{func:esl_alphabet_Create()}
{\item[ESL\_ALPHABET * esl\_alphabet\_Create(int type)]}

Creates one of the three standard bio alphabets:
\ccode{eslDNA}, \ccode{eslRNA}, or \ccode{eslAMINO}, and returns
a pointer to it.

Returns pointer to the new alphabet.

Throws \ccode{NULL} if any allocation or initialization fails.


\hypertarget{func:esl_alphabet_CreateCustom()}
{\item[ESL\_ALPHABET * esl\_alphabet\_CreateCustom(const char *alphabet, int K, int Kp)]}

Creates a customized biosequence alphabet,
and returns a ptr to it. The alphabet type is set 
to \ccode{eslNONSTANDARD}.

\ccode{alphabet} is the internal alphabet string;
\ccode{K} is the size of the base alphabet;
\ccode{Kp} is the total size of the alphabet string. 

In the alphabet string, residues \ccode{0..K-1} are the base alphabet; 
residue \ccode{K} is the canonical gap (indel) symbol; 
residues \ccode{K+1..Kp-4} are additional degeneracy symbols (possibly 0 of them);
residue \ccode{Kp-3} is an "any" symbol (such as N or X); 
residue \ccode{Kp-2} is a "nonresidue" symbol (such as *); 
and residue \ccode{Kp-1} is a "missing data" gap symbol.

The two gap symbols, the nonresidue, and the "any"
symbol are mandatory even for nonstandard alphabets, so
\ccode{Kp} $\geq$ \ccode{K+4}.

Returns pointer to new \ccode{ESL\_ALPHABET} structure.

Throws \ccode{NULL} if any allocation or initialization fails.


\hypertarget{func:esl_alphabet_SetEquiv()}
{\item[int esl\_alphabet\_SetEquiv(ESL\_ALPHABET *a, char sym, char c)]}

Maps an additional input alphabetic symbol \ccode{sym} to 
an internal alphabet symbol \ccode{c}; for example,
we might map T to U for an RNA alphabet, so that we
allow for reading input DNA sequences.

Returns \ccode{eslOK} on success.

Throws \ccode{eslEINVAL} if \ccode{c} is not in the internal alphabet, or if \ccode{sym} is.


\hypertarget{func:esl_alphabet_SetCaseInsensitive()}
{\item[int esl\_alphabet\_SetCaseInsensitive(ESL\_ALPHABET *a)]}

Given a custom alphabet \ccode{a}, with all equivalences set,
make the input map case-insensitive: for every
letter that is mapped in either lower or upper
case, map the other case to the same internal
residue.

For the standard alphabets, this is done automatically.

Returns \ccode{eslOK} on success.                

Throws \ccode{eslECORRUPT} if any lower/uppercase symbol pairs
are already both mapped to different symbols.


\hypertarget{func:esl_alphabet_SetDegeneracy()}
{\item[int esl\_alphabet\_SetDegeneracy(ESL\_ALPHABET *a, char c, char *ds)]}

Given an alphabet under construction, 
define the degenerate character \ccode{c} to mean
any of the characters in the string \ccode{ds}.

\ccode{c} must exist in the digital alphabet, as
one of the optional degenerate residues (\ccode{K+1}..\ccode{Kp-3}).
All the characters in the \ccode{ds} string must exist
in the canonical alphabet (\ccode{0}..\ccode{K-1}).

You may not redefine the mandatory all-degenerate character
(typically \ccode{N} or \ccode{X}; \ccode{Kp-3} in the digital alphabet).
It is defined automatically in all alphabets. 

Returns \ccode{eslOK} on success.

Throws \ccode{eslEINVAL} if \ccode{c} or \ccode{ds} arguments aren't valid.


\hypertarget{func:esl_alphabet_SetIgnored()}
{\item[int esl\_alphabet\_SetIgnored(ESL\_ALPHABET *a, const char *ignoredchars)]}

Given an alphabet \ccode{a} (either standard or custom), define
all the characters in string \ccode{ignoredchars} to be
unmapped: valid, but ignored when converting input text.

By default, the standard alphabets do not define any
ignored characters.

The most common ignored characters would be space, tab,
and digits, to skip silently over whitespace and
sequence coordinates when parsing loosely-defined
sequence file formats.

Returns \ccode{eslOK} on success.


\hypertarget{func:esl_alphabet_Sizeof()}
{\item[size\_t esl\_alphabet\_Sizeof(ESL\_ALPHABET *a)]}

Returns the size of alphabet \ccode{a} object, in bytes.


\hypertarget{func:esl_alphabet_Destroy()}
{\item[void esl\_alphabet\_Destroy(ESL\_ALPHABET *a)]}

Free's an \ccode{ESL\_ALPHABET} structure.

Returns (void).


\hypertarget{func:esl_abc_CreateDsq()}
{\item[int esl\_abc\_CreateDsq(const ESL\_ALPHABET *a, const char *seq, ESL\_DSQ **ret\_dsq)]}

Given an alphabet \ccode{a} and an ASCII sequence \ccode{seq},
digitize the sequence into newly allocated space, and 
return a pointer to that space in \ccode{ret\_dsq}.

Returns \ccode{eslOK} on success, and \ccode{ret\_dsq} contains the digitized
sequence; caller is responsible for free'ing this
memory. Returns \ccode{eslEINVAL} if \ccode{seq} contains
one or more characters that are not in the input map of
alphabet \ccode{a}. If this happens, \ccode{ret\_dsq} is still valid upon
return: invalid characters are replaced by full ambiguities
(typically X or N).

Throws \ccode{eslEMEM} on allocation failure.



\hypertarget{func:esl_abc_Digitize()}
{\item[int esl\_abc\_Digitize(const ESL\_ALPHABET *a, const char *seq, ESL\_DSQ *dsq)]}

Given an alphabet \ccode{a} and a nul-terminated ASCII sequence
\ccode{seq}, digitize the sequence and put it in \ccode{dsq}. Caller
provides space in \ccode{dsq} allocated for at least \ccode{L+2}
\ccode{ESL\_DSQ} residues, where \ccode{L} is the length of \ccode{seq}.

Returns \ccode{eslOK} on success.
Returns \ccode{eslEINVAL} if \ccode{seq} contains one or more characters
that are not recognized in the alphabet \ccode{a}. (This is classed
as a normal error, because the \ccode{seq} may be untrusted user input.)
If this happens, the digital sequence \ccode{dsq} is still valid upon
return; invalid ASCII characters are replaced by ambiguities
(X or N).


\hypertarget{func:esl_abc_Textize()}
{\item[int esl\_abc\_Textize(const ESL\_ALPHABET *a, const ESL\_DSQ *dsq, int64\_t L, char *seq)]}

Make an ASCII sequence \ccode{seq} by converting a digital
sequence \ccode{dsq} of length \ccode{L} back to text, according to
the digital alphabet \ccode{a}. 

Caller provides space in \ccode{seq} allocated for at least
\ccode{L+1} bytes (\ccode{(L+1) * sizeof(char)}).

Returns \ccode{eslOK} on success.


\hypertarget{func:esl_abc_TextizeN()}
{\item[int esl\_abc\_TextizeN(const ESL\_ALPHABET *a, const ESL\_DSQ *dptr, int64\_t L, char *buf)]}

Similar in semantics to \ccode{strncpy()}, this procedure takes
a window of \ccode{L} residues in a digitized sequence
starting at the residue pointed to by \ccode{dptr},
converts them to ASCII text representation, and 
copies them into the buffer \ccode{buf}.

\ccode{buf} must be at least \ccode{L} residues long; \ccode{L+1}, if the
caller needs to NUL-terminate it.

If a sentinel byte is encountered in the digitized
sequence before \ccode{L} residues have been copied, \ccode{buf} is
NUL-terminated there. Otherwise, like \ccode{strncpy()}, \ccode{buf}
will not be NUL-terminated.

Note that because digital sequences are indexed \ccode{1..N},
not \ccode{0..N-1}, the caller must be careful about
off-by-one errors in \ccode{dptr}. For example, to copy from
the first residue of a digital sequence \ccode{dsq}, you must
pass \ccode{dptr=dsq+1}, not \ccode{dptr=dsq}. The text in \ccode{buf}
on the other hand is a normal C string indexed \ccode{0..L-1}.

Returns \ccode{eslOK} on success.


\hypertarget{func:esl_abc_dsqcpy()}
{\item[int esl\_abc\_dsqcpy(const ESL\_DSQ *dsq, int64\_t L, ESL\_DSQ *dcopy)]}

Given a digital sequence \ccode{dsq} of length \ccode{L},
make a copy of it in \ccode{dcopy}. Caller provides
storage in \ccode{dcopy} for at least \ccode{L+2} \ccode{ESL\_DSQ}
residues.

Returns \ccode{eslOK} on success.


\hypertarget{func:esl_abc_dsqdup()}
{\item[int esl\_abc\_dsqdup(const ESL\_DSQ *dsq, int64\_t L, ESL\_DSQ **ret\_dup)]}

Like \ccode{esl\_strdup()}, but for digitized sequences:
make a duplicate of \ccode{dsq} and leave it in \ccode{ret\_dup}.
Caller can pass the string length \ccode{L} if it's known, saving
some overhead; else pass \ccode{-1} and the length will be
determined for you.

Tolerates \ccode{dsq} being \ccode{NULL}; in which case, returns
\ccode{eslOK} with \ccode{*ret\_dup} set to \ccode{NULL}.

Returns \ccode{eslOK} on success, and leaves a pointer in \ccode{ret\_dup}.

Throws \ccode{eslEMEM} on allocation failure.



\hypertarget{func:esl_abc_dsqcat()}
{\item[int esl\_abc\_dsqcat(const ESL\_DSQ *inmap, ESL\_DSQ **dsq, int64\_t *L, const char *s, esl\_pos\_t n)]}

Append the contents of string or memory line \ccode{s} of
length \ccode{n} to a digital sequence, after digitizing  
each input character in \ccode{s} according to an Easel
\ccode{inmap}. The destination sequence and its length
are passed by reference, \ccode{*dsq} and \ccode{*L}, so that
the sequence may be reallocated and the length updated
upon return.

The input map \ccode{inmap} may map characters to
\ccode{eslDSQ\_IGNORED} or \ccode{eslDSQ\_ILLEGAL}, but not to \ccode{eslDSQ\_EOL},
\ccode{eslDSQ\_EOD}, or \ccode{eslDSQ\_SENTINEL} codes. \ccode{inmap[0]} is
special, and must be set to the code for the 'unknown'
residue (such as 'X' for proteins, 'N' for DNA) that
will be used to replace any invalid \ccode{eslDSQ\_ILLEGAL}
characters.

If \ccode{*dsq} is properly terminated digital sequence and
the caller doesn't know its length, \ccode{*L} may be passed
as -1. Providing the length when it's known saves an
\ccode{esl\_abc\_dsqlen()} call. If \ccode{*dsq} is unterminated, \ccode{*L}
is mandatory. Essentially the same goes for \ccode{*s}, which
may be a NUL-terminated string (pass \ccode{n=-1} if length unknown),
or a memory line (\ccode{n} is mandatory). 

\ccode{*dsq} may also be \ccode{NULL}, in which case it is allocated
and initialized here.

Caller should provide an \ccode{s} that is expected to be
essentially all appendable to \ccode{*dsq} except for a small
number of chars that map to \ccode{eslDSQ\_IGNORE}, like an
input sequence data line from a file, for example. We're
going to reallocate \ccode{*dsq} to size \ccode{*L+n}; if \ccode{n} is an
entire large buffer or file, this reallocation will be
inefficient.

Returns \ccode{eslOK} on success; \ccode{*dsq} contains the result of digitizing
and appending \ccode{s} to the original \ccode{*dsq}; and \ccode{*L} contains
the new length of the \ccode{dsq} result in residues.

If any of the characters in \ccode{s} are illegal in the
alphabet \ccode{abc}, these characters are digitized as
unknown residues (using \ccode{inmap[0]}) and
concatenation/digitization proceeds to completion, but
the function returns \ccode{eslEINVAL}. The caller might then
want to call \ccode{esl\_abc\_ValidateSeq()} on \ccode{s} if it wants
to figure out where digitization goes awry and get a
more informative error report. This is a normal error,
because the string \ccode{s} might be user input.

Throws \ccode{eslEMEM} on allocation or reallocation failure;
\ccode{eslEINCONCEIVABLE} on coding error.



\hypertarget{func:esl_abc_dsqcat_noalloc()}
{\item[int esl\_abc\_dsqcat\_noalloc(const ESL\_DSQ *inmap, ESL\_DSQ *dsq, int64\_t *L, const char *s, esl\_pos\_t n)]}

Same as \ccode{esl\_abc\_dsqcat()}, but with no reallocation of
\ccode{dsq}. The pointer to the destination string \ccode{dsq} is 
passed by value not by reference, because it will not
be reallocated or moved. Caller has already allocated 
at least \ccode{*L + n + 2} bytes in \ccode{dsq}. \ccode{*L} and \ccode{n} are
not optional; caller must know (and provide) the lengths
of both the old string and the new source.



\hypertarget{func:esl_abc_dsqlen()}
{\item[int64\_t esl\_abc\_dsqlen(const ESL\_DSQ *dsq)]}

Returns the length of digitized sequence \ccode{dsq} in
positions (including gaps, if any). The \ccode{dsq} must be
properly terminated by a sentinel byte
(\ccode{eslDSQ\_SENTINEL}).  


\hypertarget{func:esl_abc_dsqrlen()}
{\item[int64\_t esl\_abc\_dsqrlen(const ESL\_ALPHABET *abc, const ESL\_DSQ *dsq)]}

Returns the unaligned length of digitized sequence
\ccode{dsq}, in residues, not counting any gaps, nonresidues,
or missing data symbols. 


\hypertarget{func:esl_abc_CDealign()}
{\item[int esl\_abc\_CDealign(const ESL\_ALPHABET *abc, char *s, const ESL\_DSQ *ref\_ax, int64\_t *opt\_rlen)]}

Dealigns \ccode{s} in place by removing characters aligned to
gaps (or missing data symbols) in the reference digital
aligned sequence \ccode{ref\_ax}. Gaps and missing data symbols
in \ccode{ref\_ax} are defined by its digital alphabet \ccode{abc}.

\ccode{s} is typically going to be some kind of textual
annotation string (secondary structure, consensus, or
surface accessibility).

Be supercareful of off-by-one errors here! The \ccode{ref\_ax}
is a digital sequence that is indexed \ccode{1..L}. The
annotation string \ccode{s} is assumed to be \ccode{0..L-1} (a
normal C string), off by one with respect to \ccode{ref\_ax}.
In a sequence object, ss annotation is actually stored
\ccode{1..L} -- so if you're going to \ccode{esl\_abc\_CDealign()} a
\ccode{sq->ss}, pass \ccode{sq->ss+1} as the argument \ccode{s}.

Returns Returns \ccode{eslOK} on success; optionally returns the number
of characters in the dealigned \ccode{s} in \ccode{*opt\_rlen}.

Throws (no abnormal error conditions)


\hypertarget{func:esl_abc_XDealign()}
{\item[int esl\_abc\_XDealign(const ESL\_ALPHABET *abc, ESL\_DSQ *x, const ESL\_DSQ *ref\_ax, int64\_t *opt\_rlen)]}

Dealigns \ccode{x} in place by removing characters aligned to
gaps (or missing data) in the reference digital aligned
sequence \ccode{ref\_ax}. Gaps and missing data symbols in
\ccode{ref\_ax} are defined by its digital alphabet \ccode{abc}.

Returns Returns \ccode{eslOK} on success; optionally returns the number
of characters in the dealigned \ccode{x} in \ccode{*opt\_rlen}.

Throws (no abnormal error conditions)


\hypertarget{func:esl_abc_ConvertDegen2X()}
{\item[int esl\_abc\_ConvertDegen2X(const ESL\_ALPHABET *abc, ESL\_DSQ *dsq)]}

Convert all the degenerate residue codes in digital
sequence \ccode{dsq} to the code for the maximally degenerate 
"unknown residue" code, as specified in digital alphabet
\ccode{abc}. (For example, X for protein, N for nucleic acid.)

This comes in handy when you're dealing with some piece
of software that can't deal with standard residue codes,
and you want to massage your sequences into a form that
can be accepted. For example, WU-BLAST can't deal with O
(pyrrolysine) residues, but UniProt has O codes.

Returns \ccode{eslOK} on success.

Throws (no abnormal error conditions)


\hypertarget{func:esl_abc_revcomp()}
{\item[int esl\_abc\_revcomp(const ESL\_ALPHABET *abc, ESL\_DSQ *dsq, int n)]}

Reverse complement \ccode{dsq}, in place, according to
its digital alphabet \ccode{abc}.

Returns \ccode{eslOK} on success.

Throws \ccode{eslEINCOMPAT} if alphabet \ccode{abc} can't be reverse complemented


\hypertarget{func:esl_abc_ValidateType()}
{\item[int esl\_abc\_ValidateType(int type)]}

Returns \ccode{eslOK} if \ccode{type} is a valid and known Easel
alphabet type code.

Used to validate "user" input, where we're parsing a
file format that has stored an Easel alphabet code.

Returns \ccode{eslFAIL} for the special \ccode{eslUNKNOWN} "unset"
value, even though that is a valid code, because it's
not an alphabet, so shouldn't show up in a file.


\hypertarget{func:esl_abc_GuessAlphabet()}
{\item[int esl\_abc\_GuessAlphabet(const int64\_t *ct, int *ret\_type)]}

Guess the alphabet type from a residue composition.
The input \ccode{ct[0..25]} array contains observed counts of 
the letters A..Z, case-insensitive. 

The composition \ccode{ct} must contain more than 10 residues.

If it contains $\geq$98\% ACGTN and all four of the
residues ACGT occur, call it \ccode{eslDNA}. (Analogously for
ACGUN, ACGU: call \ccode{eslRNA}.)

If it contains any amino-specific residue (EFIJLPQZ),
call it \ccode{eslAMINO}.  

If it consists of $\geq$98\% canonical aa residues or X,
and at least 15 of the different 20 aa residues occur,
and the number of residues that are canonical aa/degenerate
nucleic (DHKMRSVWY) is greater than the number of canonicals
for both amino and nucleic (ACG); then call it \ccode{eslAMINO}.

As a special case, if it consists entirely of N's, and
we have >2000 residues, call it \ccode{eslDNA}. This is a
special case that deals with genome sequence assemblies
that lead with a swath of N's.

We aim to be very conservative, essentially never making
a false call; we err towards calling \ccode{eslUNKNOWN} if
unsure. Our test is to classify every individual
sequence in NCBI NR and NT (or equiv large messy
sequence database) with no false positives, only correct
calls or \ccode{eslUNKNOWN}.

Returns \ccode{eslOK} on success, and \ccode{*ret\_type} is set to
\ccode{eslAMINO}, \ccode{eslRNA}, or \ccode{eslDNA}.

Returns \ccode{eslENOALPHABET} if unable to determine the
alphabet type; in this case, \ccode{*ret\_type} is set to 
\ccode{eslUNKNOWN}.



\hypertarget{func:esl_abc_Match()}
{\item[double esl\_abc\_Match(const ESL\_ALPHABET *abc, ESL\_DSQ x, ESL\_DSQ y, double *p)]}

Given two digital symbols \ccode{x} and \ccode{y} in alphabet
\ccode{abc}, calculate and return the probability that
\ccode{x} and \ccode{y} match, taking degenerate residue codes
into account.

If \ccode{p} residue probability vector is NULL, the
calculation is a simple average. For example, for DNA,
R/A gives 0.5, C/N gives 0.25, N/R gives 0.25, R/R gives
0.5.

If \ccode{p} residue probability vector is non-NULL, it gives
a 0..K-1 array of background frequencies, and the
returned match probability is an expectation (weighted
average) given those residue frequencies.

\ccode{x} and \ccode{y} should only be residue codes. Any other
comparison, including comparisons involving gap or
missing data characters, or even comparisons involving
illegal digital codes, returns 0.0.

Note that comparison of residues from "identical"
sequences (even a self-comparison) will not result in an
identity of 1.0, if the sequence(s) contain degenerate
residue codes.

Returns the probability of an identity (match) between
residues \ccode{x} and \ccode{y}.


\hypertarget{func:esl_abc_IAvgScore()}
{\item[int esl\_abc\_IAvgScore(const ESL\_ALPHABET *a, ESL\_DSQ x, const int *sc)]}

Given a residue code \ccode{x} in alphabet \ccode{a}, and an array of
integer scores \ccode{sc} for the residues in the base
alphabet, calculate and return the average score
(rounded to nearest integer).

\ccode{x} would usually be a degeneracy code, but it
may also be a canonical residue. It must not
be a gap, missing data, or illegal symbol; if it
is, these functions return a score of 0 without
raising an error.

\ccode{esl\_abc\_FAvgScore()} and \ccode{esl\_abc\_DAvgScore()} do the
same, but for float and double scores instead of integers
(and for real-valued scores, no rounding is done).

Returns average score for symbol \ccode{x}          


\hypertarget{func:esl_abc_IExpectScore()}
{\item[int esl\_abc\_IExpectScore(const ESL\_ALPHABET *a, ESL\_DSQ x, const int *sc, const float *p)]}

Given a residue code \ccode{x} in alphabet \ccode{a}, an
array of integer scores \ccode{sc} for the residues in the base
alphabet, and background frequencies \ccode{p} for the
occurrence frequencies of the residues in the base
alphabet, calculate and return the expected score
(weighted by the occurrence frequencies \ccode{p}).

\ccode{x} would usually be a degeneracy code, but it
may also be a canonical residue. It must not
be a gap, missing data, or illegal symbol; if it
is, these functions return a score of 0 without
raising an error.

\ccode{esl\_abc\_FExpectScore()} and \ccode{esl\_abc\_DExpectScore()} do the
same, but for float and double scores instead of integers
(for real-valued scores, no rounding is done).

Returns average score for symbol \ccode{x}          


\hypertarget{func:esl_abc_IAvgScVec()}
{\item[int esl\_abc\_IAvgScVec(const ESL\_ALPHABET *a, int *sc)]}

Given an alphabet \ccode{a} and a score vector \ccode{sc} of length
\ccode{a->Kp} that contains integer scores for the base
alphabet (\ccode{0..a->K-1}), fill out the rest of the score 
vector, calculating average scores for 
degenerate residues using \ccode{esl\_abc\_IAvgScore()}.

The score, if any, for a gap character \ccode{K}, the
nonresidue \ccode{Kp-2}, and the missing data character \ccode{Kp-1}
are untouched by this function. Only the degenerate
scores \ccode{K+1..Kp-3} are filled in.

\ccode{esl\_abc\_FAvgScVec()} and \ccode{esl\_abc\_DAvgScVec()} do
the same, but for score vectors of floats or doubles,
respectively.

Returns \ccode{eslOK} on success.


\hypertarget{func:esl_abc_IExpectScVec()}
{\item[int esl\_abc\_IExpectScVec(const ESL\_ALPHABET *a, int *sc, const float *p)]}

Given an alphabet \ccode{a}, a score vector \ccode{sc} of length
\ccode{a->Kp} that contains integer scores for the base
alphabet (\ccode{0..a->K-1}), and residue occurrence probabilities
\ccode{p[0..a->K-1]}; fill in the scores for the
degenerate residues \ccode{K+1..Kp-3} using \ccode{esl\_abc\_IExpectScore()}.

The score, if any, for a gap character \ccode{K}, the
nonresidue \ccode{Kp-2}, and the missing data character \ccode{Kp-1}
are untouched by this function. Only the degenerate
scores \ccode{K+1..Kp-3} are filled in.

\ccode{esl\_abc\_FExpectScVec()} and \ccode{esl\_abc\_DExpectScVec()} do
the same, but for score vectors of floats or doubles,
respectively. The probabilities \ccode{p} are floats for the
integer and float versions, and doubles for the double
version.

Returns \ccode{eslOK} on success.


\hypertarget{func:esl_abc_FCount()}
{\item[int esl\_abc\_FCount(const ESL\_ALPHABET *abc, float *ct, ESL\_DSQ x, float wt)]}

Count a possibly degenerate digital symbol \ccode{x} (0..Kp-1)
into a counts array \ccode{ct} for base symbols (0..K-1).
Assign the symbol a weight of \ccode{wt} (often just 1.0).
The count weight of a degenerate symbol is divided equally
across the possible base symbols. 

\ccode{x} can be a gap; if so, \ccode{ct} must be allocated 0..K,
not 0..K-1. If \ccode{x} is a missing data symbol, or a nonresidue
data symbol, nothing is done.

\ccode{esl\_abc\_DCount()} does the same, but for double-precision
count vectors and weights.

Returns \ccode{eslOK} on success.


\hypertarget{func:esl_abc_EncodeType()}
{\item[int esl\_abc\_EncodeType(char *type)]}

Convert a string like "amino" or "DNA" to the
corresponding Easel internal alphabet type code
such as \ccode{eslAMINO} or \ccode{eslDNA}; return the code.

Returns the code, such as \ccode{eslAMINO}; if \ccode{type} isn't
recognized, returns \ccode{eslUNKNOWN}.


\hypertarget{func:esl_abc_EncodeTypeMem()}
{\item[int esl\_abc\_EncodeTypeMem(char *type, int n)]}

Same as \ccode{esl\_abc\_EncodeType()}, but for a
non-NUL terminated memory chunk \ccode{type} of
length \ccode{n}.


\hypertarget{func:esl_abc_DecodeType()}
{\item[char * esl\_abc\_DecodeType(int type)]}

For diagnostics and other output: given an internal
alphabet code \ccode{type} (\ccode{eslRNA}, for example), return
pointer to an internal string ("RNA", for example). 


\hypertarget{func:esl_abc_ValidateSeq()}
{\item[int esl\_abc\_ValidateSeq(const ESL\_ALPHABET *a, const char *seq, int64\_t L, char *errbuf)]}

Check that sequence \ccode{seq} of length \ccode{L} can be digitized
without error; all its symbols are valid in alphabet
\ccode{a}. If so, return \ccode{eslOK}. If not, return \ccode{eslEINVAL}.

If \ccode{a} is \ccode{NULL}, we still validate that at least the
\ccode{seq} consists only of ASCII characters.

\ccode{errbuf} is either passed as \ccode{NULL}, or a pointer to an
error string buffer allocated by the caller for
\ccode{eslERRBUFSIZE} characters. If \ccode{errbuf} is non-NULL, and
the sequence is invalid, an error message is placed in
\ccode{errbuf}.

Returns \ccode{eslOK} if \ccode{seq} is valid; \ccode{eslEINVAL} if not.

Throws (no abnormal error conditions).


\end{sreapi}

