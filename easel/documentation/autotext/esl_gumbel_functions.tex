\begin{sreapi}
\hypertarget{func:esl_gumbel_pdf()}
{\item[double esl\_gumbel\_pdf(double x, double mu, double lambda)]}

Calculates the probability density function for the Gumbel,
$P(X=x)$, given quantile \ccode{x} and Gumbel location and
scale parameters \ccode{mu} and \ccode{lambda}.

Let $y = \lambda(x-\mu)$; for 64-bit doubles,
useful dynamic range is about $-6.5 <= y <= 710$.
Returns 0.0 for smaller $y$, 0.0 for larger $y$.


\hypertarget{func:esl_gumbel_logpdf()}
{\item[double esl\_gumbel\_logpdf(double x, double mu, double lambda)]}

Calculates the log probability density function for the Gumbel,
$\log P(X=x)$.

Let $y = \lambda(x-\mu)$; for 64-bit doubles,
useful dynamic range is about $-708 <= y <= \infty$.
Returns $-\infty$ for smaller or larger $y$.


\hypertarget{func:esl_gumbel_cdf()}
{\item[double esl\_gumbel\_cdf(double x, double mu, double lambda)]}

Calculates the cumulative distribution function
for the Gumbel, $P(X \leq x)$.

Let $y = \lambda(x-\mu)$; for 64-bit doubles,
useful dynamic range for $y$ is about $-6.5 <= y <=36$.
Returns 0.0 for smaller $y$, 1.0 for larger $y$.


\hypertarget{func:esl_gumbel_logcdf()}
{\item[double esl\_gumbel\_logcdf(double x, double mu, double lambda)]}

Calculates the log of the cumulative distribution function
for the Gumbel, $\log P(X \leq x)$.

Let $y = \lambda(x-\mu)$; for 64-bit doubles,
useful dynamic range for $y$ is about $-708 <= y <= 708$.
Returns $-\infty$ for smaller $y$, 0.0 for larger $y$.


\hypertarget{func:esl_gumbel_surv()}
{\item[double esl\_gumbel\_surv(double x, double mu, double lambda)]}

Calculates the survivor function, $P(X>x)$ for a Gumbel 
(that is, 1-cdf), the right tail's probability mass.

Let $y=\lambda(x-\mu)$; for 64-bit doubles, 
useful dynamic range for $y$ is $-3.6 <= y <= 708$.
Returns 1.0 for $y$ below lower limit, and 0.0
for $y$ above upper limit.


\hypertarget{func:esl_gumbel_logsurv()}
{\item[double esl\_gumbel\_logsurv(double x, double mu, double lambda)]}

Calculates $\log P(X>x)$ for a Gumbel (that is, $\log$(1-cdf)):
the log of the right tail's probability mass.

Let $y=\lambda(x-\mu)$; for 64-bit doubles, 
useful dynamic range for $y$ is $-6.5 <= y <= \infty$.
Returns 0.0 for smaller $y$.


\hypertarget{func:esl_gumbel_invcdf()}
{\item[double esl\_gumbel\_invcdf(double p, double mu, double lambda)]}

Calculates the inverse CDF for a Gumbel distribution
with parameters \ccode{mu} and \ccode{lambda}. That is, returns
the quantile \ccode{x} at which the CDF is \ccode{p}.


\hypertarget{func:esl_gumbel_invsurv()}
{\item[double esl\_gumbel\_invsurv(double p, double mu, double lambda)]}

Calculates the score at which the right tail's mass
is p, for a Gumbel distribution
with parameters \ccode{mu} and \ccode{lambda}. That is, returns
the quantile \ccode{x} at which 1-CDF is \ccode{p}.


\hypertarget{func:esl_gumbel_generic_pdf()}
{\item[double esl\_gumbel\_generic\_pdf(double p, void *params)]}

Generic-API version of PDF function.


\hypertarget{func:esl_gumbel_generic_cdf()}
{\item[double esl\_gumbel\_generic\_cdf(double x, void *params)]}

Generic-API version of CDF function.


\hypertarget{func:esl_gumbel_generic_surv()}
{\item[double esl\_gumbel\_generic\_surv(double p, void *params)]}

Generic-API version of survival function.


\hypertarget{func:esl_gumbel_generic_invcdf()}
{\item[double esl\_gumbel\_generic\_invcdf(double p, void *params)]}

Generic-API version of inverse CDF.


\hypertarget{func:esl_gumbel_Plot()}
{\item[int esl\_gumbel\_Plot(FILE *fp, double mu, double lambda, 
		double (*func)(double x, double mu, double lambda), 
		double xmin, double xmax, double xstep)]}

Plot a Gumbel function \ccode{func} (for instance,
\ccode{esl\_gumbel\_pdf()}) for parameters \ccode{mu} and \ccode{lambda}, for
a range of quantiles x from \ccode{xmin} to \ccode{xmax} in steps of \ccode{xstep};
output to an open stream \ccode{fp} in xmgrace XY input format.

Returns \ccode{eslOK} on success.

Throws \ccode{eslEWRITE} on any system write error, such as filled disk.


\hypertarget{func:esl_gumbel_Sample()}
{\item[double esl\_gumbel\_Sample(ESL\_RANDOMNESS *r, double mu, double lambda)]}

Sample a Gumbel-distributed random variate
by the transformation method.


\hypertarget{func:esl_gumbel_FitComplete()}
{\item[int esl\_gumbel\_FitComplete(double *x, int n, double *ret\_mu, double *ret\_lambda)]}

Given an array of Gumbel-distributed samples
\ccode{x[0]..x[n-1]}, find maximum likelihood parameters \ccode{mu}
and \ccode{lambda}.

The number of samples \ccode{n} must be reasonably large to get
an accurate fit. \ccode{n=100} suffices to get an accurate
location parameter $\mu$ (to about 1% error), but
\ccode{n~10000} is required to get a similarly accurate
estimate of $\lambda$. It's probably a bad idea to try to
fit a Gumbel to less than about 1000 data points.

On a very small number of samples, the fit can fail
altogether, in which case the routine will return a
\ccode{eslENORESULT} code. Caller must check for this.

Uses approach described in [Lawless82]. Solves for lambda
using Newton/Raphson iterations, then substitutes lambda
into Lawless' equation 4.1.5 to get mu.

Returns \ccode{eslOK} on success.

\ccode{eslEINVAL} if n\ccode{=1.
<eslENORESULT} if the fit fails, likely because the
number of samples is too small. On either error,
\ccode{*ret\_mu} and \ccode{*ret\_lambda} are 0.0.  These are classed
as failures (normal errors) because the data vector may
have been provided by a user.


\hypertarget{func:esl_gumbel_FitCompleteLoc()}
{\item[int esl\_gumbel\_FitCompleteLoc(double *x, int n, double lambda, double *ret\_mu)]}

Given an array of Gumbel-distributed samples 
\ccode{x[0]..x[n-1]} (complete data), and a known
(or otherwise fixed) \ccode{lambda}, find a maximum
likelihood estimate for location parameter \ccode{mu}.

Algorithm is a straightforward simplification of
\ccode{esl\_gumbel\_FitComplete()}.

Returns \ccode{eslOK} on success.

\ccode{eslEINVAL} if n\ccode{=1; on this error, <*ret\_mu} = 0.

Throws (no abnormal error conditions)


\hypertarget{func:esl_gumbel_FitCensored()}
{\item[int esl\_gumbel\_FitCensored(double *x, int n, int z, double phi, double *ret\_mu, double *ret\_lambda)]}

Given a left-censored array of Gumbel-distributed samples
\ccode{x[0]..x[n-1]}, the number of censored samples \ccode{z}, and
the censoring value \ccode{phi} (all \ccode{x[i]} $\geq$ \ccode{phi}).  Find
maximum likelihood parameters \ccode{mu} and \ccode{lambda}.

Returns \ccode{eslOK} on success.

\ccode{eslEINVAL} if n\ccode{=1. 
<eslENORESULT} if the fit fails, likey because the number
of samples is too small.
On either error, \ccode{*ret\_mu} and \ccode{*ret\_lambda} are 0.0.
These are classed as failures (normal errors) because the
data vector may have been provided by a user.


\hypertarget{func:esl_gumbel_FitCensoredLoc()}
{\item[int esl\_gumbel\_FitCensoredLoc(double *x, int n, int z, double phi, double lambda, 
			  double *ret\_mu)]}

Given a left-censored array of Gumbel distributed samples
\ccode{x[0}..x[n-1]>, the number of censored samples \ccode{z}, and the censoring
value \ccode{phi} (where all \ccode{x[i]} $\geq$ \ccode{phi}), and a known
(or at least fixed) \ccode{lambda};
find the maximum likelihood estimate of the location
parameter $\mu$ and return it in \ccode{ret\_mu}.

Returns \ccode{eslOK} on success.

\ccode{eslEINVAL} if n\ccode{=1; on this error, <*ret\_mu} = 0.

Throws (no abnormal error conditions)


\hypertarget{func:esl_gumbel_FitTruncated()}
{\item[int esl\_gumbel\_FitTruncated(double *x, int n, double phi, double *ret\_mu, double *ret\_lambda)]}

Given a left-truncated array of Gumbel-distributed
samples \ccode{x[0]..x[n-1]} and the truncation threshold
\ccode{phi} (such that all \ccode{x[i]} $\geq$ \ccode{phi}).
Find maximum likelihood parameters \ccode{mu} and \ccode{lambda}.

\ccode{phi} should not be much greater than \ccode{mu}, the
mode of the Gumbel, or the fit will become unstable
or may even fail to converge. The problem is
that for \ccode{phi} $>$ \ccode{mu}, the tail of the Gumbel
becomes a scale-free exponential, and \ccode{mu} becomes
undetermined.

Returns \ccode{eslOK} on success.

\ccode{eslEINVAL} if n\ccode{=1.
<eslENORESULT} if the fit fails, likely because the
number of samples \ccode{n} is too small, or because the
truncation threshold is high enough that the tail
looks like a scale-free exponential and we can't
obtain \ccode{mu}.
On either error, \ccode{*ret\_mu} and \ccode{*ret\_lambda} are 
returned as 0.0.
These are "normal" (returned) errors because 
the data might be provided directly by a user.

Throws \ccode{eslEMEM} on allocation error.           


\end{sreapi}

